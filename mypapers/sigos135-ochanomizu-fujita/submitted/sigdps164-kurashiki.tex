\documentclass[submit,techreq,noauthor,dvipdfmx]{ipsj}

\usepackage{graphicx}
\usepackage{latexsym}
\usepackage{url}    % bibtex で url を使う
\urlstyle{same}     % フォントが変わる問題をなくす

%%%%%%%%%%%%%%%%%%%%%%%%%%%%%%%%%%%%%%%%%%%%%%%%%%%%%%%%%%%%%%%%
%% sty/ にある研究室独自のスタイルファイル

\usepackage{jtygm}  % フォントに関する余計な警告を消す
\usepackage{nutils} % insertfigure, figef, tabref マクロ

\def\figdir{./figs} % 図のディレクトリ
\def\figext{pdf}    % 図のファイルの拡張子

%%%%%%%%%%%%%%%%%%%%%%%%%%%%%%%%%%%%%%%%%%%%%%%%%%%%%%%%%%%%%%%%
%% 文字に関するマクロ

\def\DesktopBookmark{\mbox{Desktop} \mbox{Bookmark}}
\def\DTB{\mbox{DTB}}
\def\WorkState{\mbox{Time} \mbox{Entry}}
\def\Task{\mbox{Task}}
\def\Recurrence{\mbox{Recurrence}}
\def\Mission{\mbox{Mission}}
\def\Job{\mbox{Job}}
\def\Duration{\mbox{Duration}}
\def\TimeEntry{\mbox{Time} \mbox{Entry}}
\def\UnifiedHistory{\mbox{Unified} \mbox{History}}
\newcommand{\Info}[1]{\mbox{(情報#1)}}
\newcommand{\Source}[1]{\mbox{(情報源#1)}}
\newcommand{\Step}[2]{\mbox{(手順#1-#2)}}
\newcommand{\Number}[1]{\mbox{(通番#1)}}

%%%%%%%%%%%%%%%%%%%%%%%%%%%%%%%%%%%%%%%%%%%%%%%%%%%%%%%%%%%%%%%%
%% タイトル

\def\Underline{\setbox0\hbox\bgroup\let\\\endUnderline}\def\endUnderline{\vphantom{y}\egroup\smash{\underline{\box0}}\\}

\setcounter{巻数}{1}%vol53=2012
\setcounter{号数}{1}
\setcounter{page}{1}

\begin{document}

\title{再利用情報を利用した\\メールとタスクの関連付けシステムの提案}

\affiliate{OU}{岡山大学大学院自然科学研究科\\
Graduate School of Natural Science and Technology, Okayama University}

\author{小林 寛明}{Hiroaki Kobayashi}{OU}
\author{乃村 能成}{Yoshinari Nomura}{OU}

\begin{abstract}
  メールとカレンダ上のタスクを関連付けて管理することで,
  相互参照が可能となり,仕事の効率化が期待できる.
  しかし,関連付けの手間が大きいと,それ自体がコストとなってしまう.
  そこで,本稿では,関連付けの手間を軽減する手法として,
  タスクやメールが周期的に再利用されている事に着目した.
  具体的には,タスクやメールを新規ではなく,
  再利用(コピー)で作成させることによって,
  その履歴を蓄積し,タスク間やメール間の
  関連性を検出することで,利用者を支援する.
  また,利用者を支援する具体的なシステムを設計し,動作の概要について述べた.
\end{abstract}

\begin{jkeyword}
  メール,再利用情報,カレンダ
\end{jkeyword}
\maketitle

%%%%%%%%%%%%%%%%%%%%%%%%%%%%%%%%%%%%%%%%%%%%%%%%%%%%%%%%%%%%%%%%
%% 1章 はじめに
\section{はじめに}
%%%%%%%%%%%%%%%%%%%%%%%%%%%%%%%%%%%%%%%%%%%%%%%%%%%%%%%%%%%%%%%%

%%%
% 1. メールとタスクの間の関連性
オフィスにおける主要なアプリケーションとして,メールとカレンダシステムがある.
作業の日程管理にはカレンダが利用される.
本稿では,カレンダに登録された作業を特にタスクと呼ぶ.
タスクに関する報告や連絡にはメールが利用される.
両者は,互いに関連する情報を含んでいることが多く,
メールとタスクを関連付けて管理することで,仕事の効率化が期待できる.
また,これらを意識したシステムやツールが多く存在する
\cite{GoogleTasks}\cite{Mac_app}.

%%%
% 3. メールとタスクは扱う情報が異なるから,作業情報を得るには相互に補足する必要あり.
メールとタスクは関連する情報を保持しているものの,扱う情報が異なる.
たとえば,メールとカレンダを利用する例として,「会議」という作業を考える.
まず,タスクとしてカレンダに「会議」が登録される.また,「会議のお知らせ」
というメールが関係者に送信され,その会議の目的や議題,関連する資料などが告知される.
つまり,タスクは,主に作業の実施日時の情報を一覧性の高い方法で提供するのに
対して,メールは,作業に関する詳細な情報,あるいは議論を保持しているといえる.
このため,作業の情報を得るには,メールとタスクのそれぞれがもつ情報を利用して,
作業の情報を補足することが有効である.
また,この際,メールとタスク間の関連を利用して,関連するメールとタスクを同時に
閲覧できれば,作業の情報を補足する手間を軽減できる.

このように,タスクとメールを相互に関連付けることで作業の効率化が期待できるものの,
メールとタスクの関連を発見したり,ユーザに別途関連付の操作をさせることは,難しい.

メールとタスク間の関連性を利用して作業を効率化する関連研究として,
タスクに関連したリソースを自動抽出して提供することによりメール作成を支援する研究
がある\cite{nagai2009}.
しかし,リソースを適切に自動抽出できなかった場合はメール作成の支援が得られないため,
自動抽出の精度に依存しているといえる.
他にも,メールや添付ファイルといった要素をタスクを中心とした集合で管理する
研究がある\cite{Bellotti2003}.
しかし,集合に要素を含める作業はユーザが全て手動で行う必要があり,手間がかかる.
また,多くのメールアプリケーションでは,
メールからタスクを登録する方式を提供しているが,
タスクを先に登録してしまった場合には,後から既存タスクにメールを
関連付けることは難しい.

%%%
% 4. メールの再利用を利用したい.
そこで,本稿では,多くの作業が繰返し発生していることに
着目した関連付け手法を提案する.
多くの繰り返し作業では,タスクの登録やメールの送信時に過去の
同様のタスクやメールの内容をコピーして再利用することがよくある.
この再利用の元と先の関係を蓄積して利用することで,
再利用元のメールに既に関連するタスクがあれば,
再利用先のメールにも同様のタスクがあると推測できる.
同様に,過去のタスクを参考に新たなタスクを作成して登録する際,過去のタスクに関連する
メールから,新たなタスクに関連して送信すべきメールを把握できる.

本稿では,これら一連の再利用の操作をユーザのコピー&ペーストではなく
明示的なユーザインタフェース操作として提供することで,そのユーザインタフェースを
通じて行った再利用を捉えて再利用情報として蓄積し,メールとタスクの関連付けに役立てる
方式について検討する.また,検討に基づいて,
再利用情報を利用したメールとタスクの関連付けシステムを提案する.

%%%
% 5. 調査と分かった問題
以降では,まず,メールとタスクが実際にはどの程度関連するかを調査する.
この調査により,メールとカレンダの連携の利点を得られる機会が多いことを確認する.
次に,既存のアプリケーションでメールとカレンダの連携の利点を得るうえでの問題
とその対処を提案する.
最後に,提案する対処を実現するシステムを設計する.


%%%%%%%%%%%%%%%%%%%%%%%%%%%%%%%%%%%%%%%%%%%%%%%%%%%%%%%%%%%%%%%%
%% 2章 メールとカレンダの連携
\section{メールとカレンダの連携}\label{chap:mail_task_cooperation}
%%%%%%%%%%%%%%%%%%%%%%%%%%%%%%%%%%%%%%%%%%%%%%%%%%%%%%%%%%%%%%%%

%%%%%%%%%%%%%%%%
%% 2.1節 メールとカレンダの連携とは
\subsection{メールとカレンダの連携とは}\label{sec:this_study_cooperation}
%%%%%%%%%%%%%%%%

本稿におけるメールとカレンダの連携とは,以下の3つを指す.
\begin{enumerate}
\item メールとタスクの関連の情報を保持\\
メールとタスク間には関連がある.
この関連を情報として保持する.
具体的には,関連するメールとタスクを一意に特定できる情報の組を関連の情報として
保持する.
この情報により,関連するメールとタスクが分かる.

\item 関連するメールとタスクを参照\\
メールとタスクの関連の情報を利用して,双方から他方を参照可能にする.
これにより,関連するメールとタスクを閲覧できれば,
タスクに関連して送信すべきメールの存在を把握できる.

\item 再利用操作を利用したメールとタスクの関連付け支援\\
再利用とは,過去のメールやタスクを参考にして類似したメールやタスクを作成する
ことである.
メールを再利用した際,再利用元のメールに関連しているタスクから,
再利用先のメールに関連するタスクの存在を予測できる.
同様に,過去のタスクを参考に新たなタスクを作成し登録した場合,
新たなタスクに関連するメールの存在を予測できる.

\end{enumerate}


%%%%%%%%%%%%%%%%
%% 2.2節 メールの再利用と再利用情報
\subsection{メールの再利用と再利用情報}\label{sec:mail_reuse}
%%%%%%%%%%%%%%%%

タスクの中には,繰返し発生するものがある.
このようなタスクに関連するメールは繰返し発生し,
過去のメールと類似した内容のメールが繰返しやりとりされる.
この際,メールの再利用が行われる.
メールの再利用とは,過去のメールの文面をコピーし,修正して新しいメールを作成
することである.
% このメールの再利用には,以下の3つの問題がある.
% \begin{description}
%
% \item[問題1] メールの送信を忘れる
%
% \item[問題2] 再利用するメールを探す作業に手間がかかる
%
% \item[問題3] 再利用するメールの文面修正に手間がかかる
%
% \end{description}
%
% これらの問題に対処するため,
しかしながら,メールの再利用は,その事実が利用者の記憶にしか蓄積
されないため,再利用の情報を有効に活用することは難しい.
あるいは,再利用が容易にできなければ,メール作成の手間も増大する.
これらの問題を解決するために,
メールの再利用を促進するシステムが提案されている\cite{kimuray2014a}.
このシステムは,メールを再利用した履歴の情報を再利用情報として保持する.
再利用情報は,再利用元・再利用先のメールを一意に特定できる識別子や
再利用された日時といった情報で構成される.
この再利用情報を利用して,メールの再利用の提案やテンプレートによるメールの
作成支援をシステムが行い,メールの再利用における問題に対処する.

%%%%%%%%%%%%%%%%
%% 2.3節 タスクの再利用と再利用情報
\subsection{タスクの再利用と再利用情報}\label{sec:task_reuse}
%%%%%%%%%%%%%%%%

多くのタスクは「2週間に1回」や「毎年12月下旬」のように,ある程度決まった周期性に
基づいて繰返し発生する\cite{mihara2011a}.
このため,将来の予定を計画する際には,過去のタスクが再利用されることが多い.
タスクの再利用とは,過去のタスクを参考に同様の新たなタスクを作成することである.
このタスクの再利用を支援するため,
作業間の関係を集合の包含関係でモデル化する手法が提案されている\cite{mihara2013a}.
この手法では,繰返し発生する同様のタスクをリカーレンスという集合にまとめ,
次に発生する同様の予定の推測に利用する.
これにより,同一のリカーレンスに属する過去のタスクを参考に新たなタスクを作成できる.
このため,このリカーレンスがタスクを再利用した履歴の情報,
すなわち再利用情報であると捉えられる.

%%%%%%%%%%%%%%%%
%% 2.4節 メールとカレンダの連携の利点
\subsection{メールとカレンダの連携の利点}\label{sec:merit_of_cooperation}
%%%%%%%%%%%%%%%%

%%%%%%%%%%%%%%%%%%%%%%%%%%%%%
%% タスクの詳細記述欄にメールに含まれる情報を記述する例
\insertfigure[1.0]{post_to_task}{fig1}{タスクの詳細記述欄にメールに含まれる情報を記述する例}{English Caption}
%%%%%%%%%%%%%%%%%%%%%%%%%%%%%

本研究の目的は,
\ref{sec:this_study_cooperation}節で述べた連携を
\ref{sec:mail_reuse}節と\ref{sec:task_reuse}節で述べた再利用情報を
利用して実現することである.これにより,以下の3つの利点が得られる.
\begin{enumerate}

\item 作業の情報を補足する手間の軽減\\
作業の詳細な情報を把握するには,メールがもつ作業の情報を補足する必要がある.
このため,タスクに関連するメールを探し出して参照することが考えられる.
たとえば,「忘年会」というタスクの場合,
過去にやりとりしたすべてのメールの中から,作業の詳細な情報をもつ
「忘年会の詳細と会費について」というメールを探し出して参照することで,
作業の詳細な情報を得られる.
%%
また,メールの文面に含まれる作業の情報をタスクの詳細情報欄に記述しておくことにより,
タスクを見ることで作業の詳細な情報を得る場合もある.
この例を\figref{post_to_task}に示す.
\figref{post_to_task}は,「忘年会」というタスクの詳細情報欄に
「忘年会の詳細と会費について」というメールに含まれる作業の情報を
記述する様子を表している.

これらの場合,メールの探索や詳細情報の記述作業により,手間がかかる.
メールとタスクを関連付け,関連するメールとタスクを同時に閲覧すれば,
メールの探索や詳細情報の記述作業は必要なく,手間を軽減できる.


%%%%%%%%%%%%%%%%%%%%%%%%%%%%%
%% タスクに関連して送信すべきメールを把握する例
\insertfigure[0.7]{merit2}{fig2}{タスクに関連して送信すべきメールを把握する例}{English Caption}
%%%%%%%%%%%%%%%%%%%%%%%%%%%%%

\item タスクに関連して送信すべきメールの把握\\
  メールとタスクが関連付けられていれば,タスクの再利用情報を利用して再利用元と
  なった過去のタスクを参照することで,タスクに関連したメールの送信の必要性,時期,
  および目的を把握できる.
  タスクに関連して送信すべきメールを把握する例を\figref{merit2}に示す.
  \figref{merit2}は,「第2回打合せ」というタスクを登録した際に,再利用元となった
  過去の「第1回打合せ」とこれに関連する「第1回打合せ議事録」というメールを参照し,
  「第2回打合せ」の翌日に「第2回打合せ議事録」というメールを送信すべきと把握する
  様子を表している.


%%%%%%%%%%%%%%%%%%%%%%%%%%%%%
%% メール送信を契機として登録すべきタスクを把握する例
\insertfigure[0.7]{merit3}{fig3}{メール送信を契機として登録すべきタスクを把握する例}{English Caption}
%%%%%%%%%%%%%%%%%%%%%%%%%%%%%

\item メール送信を契機とした登録すべきタスクの把握\\
メールを再利用する際,再利用元となるメールに関連しているタスクがあれば,
再利用先のメールに関連するタスクを把握できる.
メール送信を契機として登録すべきタスクを把握する例を\figref{merit3}に示す.
\figref{merit3}は,「第1回打合せについて」というメールを再利用して
「第2回打合せについて」というメールを作成した際,
「第1回打合せについて」に関連する「第1回打合せ」から,
「第2回打合せについて」に関連して「第2回打合せ」というタスクを登録すべきと
把握する様子を表している.

\end{enumerate}

(1)について,作業の完了後に再び同様の作業が発生した際,メールとタスクは参照される
と考えらえる.
つまり,一度のみのメールとタスクは完了後には参照されないため,あまり利点を得られない.
(2)と(3)についても,メールとタスクが繰返し発生している場合に得られる利点である.
以上より,一度のみのメールとタスクを関連付けても得られる利点は少なく,
有益性が期待できない.
利点が多く得られるのは,繰返し発生するメールとタスクを関連付けた場合であるため,
再利用情報を利用してメールとタスクの連携を図る本方式は,理にかなっているといえる.

%%%%%%%%%%%%%%%%
%% 3章 メールとタスクの関連の調査
\section{メールとタスクの関連の調査}\label{sec:task_research}
%%%%%%%%%%%%%%%%

%%%%%%%%%%%%%
%% 3.1節 調査対象
\subsection{調査対象}\label{sec:research_target}
%%%%%%%%%%%%%

メールとタスクが実際にどの程度関連するかを確かめるため,
著者らが所属する研究グループにおいて,メールとカレンダ上のタスクの
関連について調査した.
具体的には,研究グループでの共有カレンダに登録されているタスクと,
メーリングリストに流れたメールとの結びつきを数え上げた.

調査に用いるメーリングリストと共有カレンダは以下の特徴をもつ.
\begin{description}

\item[特徴1] 多人数で運用される\\
研究グループに所属する40名程度が利用する.

\item[特徴2] メーリングリストは作業の連絡に多用される\\
メーリングリストには,作業の日程連絡や事後報告のメールが送信される.
たとえば,「打合せ」という作業の報告を行う「打合せ議事録」や
「ボウリング大会」という作業の連絡を行う「ボウリング大会のお知らせ」
が送信される.

\item[特徴3] 共有カレンダは定期的に発生する作業を含む\\
著者らが所属する研究グループでは,1年の間に複数回,定期的に発生する作業がある.
共有カレンダにはこのような作業が含まれる.
たとえば,1カ月ごとに発生する「ミーティング」がある.

\end{description}

\begin{enumerate}

\item 調査対象\\
調査対象とするのは,調査対象期間(2013年1月1日〜2013年12月31日)に研究グループの
メーリングリストに流れたメール508通と,
同期間に研究グループの共有カレンダに登録されているタスク53件である.
それぞれの調査対象について,以下で詳しく述べる.

\item メール\\
調査対象期間に研究グループのメーリングリスト宛に送られたメールのうち,
勤怠管理に関するメールを除外したメール508通である.

\item タスク\\
調査対象とするのは,以下の2種類のタスクの合計53件である.
\begin{enumerate}

\item 共有カレンダに登録されたタスク\\
研究グループの共有カレンダに登録されているタスクのうち,
調査対象期間に登録されているタスク26件である.

\item 共有カレンダに後に補完されたタスク\\
調査対象期間に研究グループの共有カレンダにタスクとして本来登録されるべきだが,
登録を忘れたため実際には登録されず,
調査のため調査対象のメールから抽出され,補完されたタスク27件である.
タスクを抽出できるメールの例を\figref{sampling}に示す.
\figref{sampling}のメールの文面から,「新B4歓迎会」が実施されたことが分かる.
このため,「新B4歓迎会」をタスクとして抽出する.
%%%%%%%%%%%%%%%%%%%%%%%%%%%%%
%% タスクを抽出できるメールの例
\insertfigure[0.7]{sampling}{fig4}{タスクを抽出できるメールの例}{English Caption}
%%%%%%%%%%%%%%%%%%%%%%%%%%%%%

\end{enumerate}
%%%%%%%%%%


\end{enumerate}

%%%%%%%%%%%%%
\subsection{関連の基準}\label{sec:reference_point}
%%%%%%%%%%%%%

メールとタスクが同じ作業の情報をもつ場合,メールとタスクは関連するといえる.
また,タスクの実施に向けて必要な別の作業の情報や
タスクの実施後に付随して発生する作業の情報を
メールがもつ場合も,メールとタスクは関連すると考えられる.
そこで,今回の調査では,タスクに対してメールが下記のどちらかの基準を満たす場合,
メールとタスクが関連するものとする.
\begin{description}

\item[基準1] メールの文面にタスクに直接的に関係する情報が含まれる\\
タスクに直接的に関係する情報として,以下の3つが挙げられる.
\begin{enumerate}

\item タスクの実施時期

\item タスクの実施場所

\item タスクの内容

\end{enumerate}


\item[基準2] メールの文面にタスクに間接的に関係する情報が含まれる\\
タスクに間接的に関係する情報として,以下の2つが挙げられる.
\begin{enumerate}

\item タスクの実施前に行う必要のある作業の情報

\item タスクの実施後に行う必要のある作業の情報

\end{enumerate}

\end{description}
%%%%%%%%%%%%%%%%%%%%%%%%%%%%%
%% 関連するメールとタスクの例
\insertfigure[1.0]{cooperation_example}{fig5}{関連するメールとタスクの例}{English Caption}
%%%%%%%%%%%%%%%%%%%%%%%%%%%%

関連するメールとタスクの例を\figref{cooperation_example}に示す.
「研修会」のタスクと,「メール1」と「メール2」の2通のメールがある.
「メール1」は「研修会」というタスクの実施時期と実施場所を文面に含んでおり,
基準1を満たす.また,
「メール2」は「研修会」というタスクの実施前に行う必要がある「食物アレルギーの調査」
という作業の情報を文面に含んでおり,基準2を満たす.

%%%%%%%%%%%%%
\subsection{調査結果}\label{sec:research_result}
%%%%%%%%%%%%%

%%%%%%%%%%%%%%%%%%%%%%%%%%%%%%
%% タスクの結果の表
\begin{table*}[tb]
  \begin{center}
    \caption{メールと関連するタスクの調査結果}
    %\ecaption{Outbreak frequency start time.} 
    \label{tab:result_task}
    \scalebox{1.0}{
      \begin{tabular}{l|r|r|r}
        \hline \hline
        & \multicolumn{1}{l|}{メールと関連する} 
        & \multicolumn{1}{l|}{メールと関連しない} 
        & \multicolumn{1}{c}{総数} \\
        & \multicolumn{1}{l|}{タスク} 
        & \multicolumn{1}{l|}{タスク} & \\
        & \multicolumn{1}{r|}{(割合)} & \multicolumn{1}{r|}{(割合)} & \\
        \hline
        (A)共有カレンダに登録された & 23 & 3 & 26 \\
        タスク & (88\%) & (12\%) & \\
        \hline
        (B)共有カレンダに後に補完された & 50 & 3 & 53 \\
        タスクを含む場合 & (94\%) & (6\%) & \\
        \hline
      \end{tabular}
    }
  \end{center}
\end{table*}
%%%%%%%%%%%%%%%%%%%%%%%%%%%%%%

%%%%%%%%%%%%%%%%%%%%%%%%%%%%%%
%% メールの結果の表
\begin{table*}[tb]
  \begin{center}
    \caption{タスクと関連するメールの調査結果}
    %\ecaption{Outbreak frequency start time.} 
    \label{tab:result_mail}
    \scalebox{1.0}{
      \begin{tabular}{l|r|r|r|r}
        \hline \hline
        & \multicolumn{2}{c|}{タスクと関連するメール} 
        & \multicolumn{1}{l|}{タスクと関連しない} 
        & \multicolumn{1}{c}{総数} \\ \cline{2-3}
        & \multicolumn{1}{l|}{(A)共有カレンダに登録された} 
        & \multicolumn{1}{l|}{(B)共有カレンダに後に補完された} 
        & \multicolumn{1}{l|}{メール} & \\
        & \multicolumn{1}{l|}{タスク} 
        & \multicolumn{1}{l|}{タスクを含む場合}
        & & \\
        & (割合) & (割合) & (割合) &\\
        \hline
        & 75 & 199 & 309 & 508\\
        & (15\%) & (39\%) & (61\%) & \\
        \hline 
      \end{tabular}
    }
  \end{center}
\end{table*}
%%%%%%%%%%%%%%%%%%%%%%%%%%%%%%

調査結果を\tabref{tab:result_task}と\tabref{tab:result_mail}に示す.
\tabref{tab:result_task}はタスクから見たメールとの関連,
\tabref{tab:result_mail}はメールから見たタスクとの関連を示している.
調査結果から,以下の2つのことが分かった.
\begin{enumerate}

\item メールとカレンダの連携の利点を得られる機会が多い\\
\tabref{tab:result_task}より,(A)共有カレンダに登録されたタスクの
88\%にあたる23件のタスクがメールと関連すると分かった.
また,(B)共有カレンダに後に補完されたタスクを含めた場合は
タスク全体の94\%にあたる50件のタスクがメールと関連すると分かった.
これらより,タスクはメールと関連することが多く,\ref{sec:merit_of_cooperation}節で
述べたメールとカレンダの連携の利点を得られる機会が多いと考えられる.

\item タスクと関連するメールの割合は,タスクの登録粒度に依存する\\
\tabref{tab:result_mail}より,
(A)共有カレンダに登録されたタスクに関連するメールは,
調査対象メール全体の15\%にあたる75通であると分かった.
また,(B)共有カレンダに後に補完されたタスクを含めた場合に
関連するメールは,調査対象メール全体の39\%にあたる199通と分かった.
つまり,タスクの数が増えると,タスクと関連するメールの割合が大きくなることが分かる.
よって,タスクと関連するメールの割合はタスクの数に依存するといえる.

カレンダに登録されるタスクの数は,
どの程度の粒度の作業までをタスクとしてカレンダに登録するかによって変化する.
カレンダにタスクとして登録する作業の粒度が細かいほど,
カレンダに登録されるタスクの数は多くなる.
このため,今回の調査に用いた共有カレンダより細かい粒度の作業までをタスクとして
登録しているカレンダを調査に用いた場合,
メールの総数に占めるタスクと関連するメールの割合は
今回の調査の結果より大きくなると考えられる.

\end{enumerate}


%%%%%%%%%%%%%%%%%%%%%%%%%%%%%%%%%%%%%%%%%%%%%%%%%%%%%%%%%%%%%%%%
%% 4章 問題と対処
\section{問題と対処}\label{chap:problem_and_solution}
%%%%%%%%%%%%%%%%%%%%%%%%%%%%%%%%%%%%%%%%%%%%%%%%%%%%%%%%%%%%%%%%

%%%%%%%%%%%%%%%%
\subsection{メールとカレンダの連携における問題}\label{sec:problem}
%%%%%%%%%%%%%%%%
メールとタスクの関連付けを行える既存のアプリケーション
\cite{GoogleTasks}\cite{Mac_app}
を調査し,メールとカレンダの連携の利点を得るうえでの問題を明らかにした.
この結果,既存のアプリケーションを利用したメールとカレンダの連携において,
以下の2つの問題があると判断した.
\begin{description}
  \item[問題1] 関連するタスクをメールから参照できない\\
    \ref{sec:merit_of_cooperation}節で述べたメールとカレンダの連携の利点を得るため
    には,
    メールとタスクの関連の情報を保持できる機能と,
    メールとタスクの双方から他方を参照できる機能が必要である.
    Google Tasks\cite{GoogleTasks}やMacの「メール」と「カレンダー」\cite{Mac_app}
    では,メールからタスクを作成することでメールとタスクを関連付けることが可能である.
    また,関連するメールをタスクから参照することが可能である.
    これは,メールとタスクの関連の情報をタスク側に保持させ,関連するメールの情報を
    タスクがもつことで実現している.
    しかし,メールとタスクの関連の情報をタスク側に保持させているため,
    メールとタスクの関連をメールから把握できない.
    このため,既存のアプリケーションでは,関連するメールをタスクから参照できるが,
    逆に関連するタスクをメールから参照できない.

  \item[問題2] メールの検索に手間がかかる\\
    メールをやり取りした時点ではタスクと関連付けず,後に関連付ける場合,
    タスクと関連付けるメールを探す必要がある.
    メールを探す方法として,多くのメーラにはキーワード検索の機能がある.
    しかし,キーワードを含むメールが多く,タスクと関連付けるメールを見つけづらい
    場合が考えられる.
    また,略称や別名でタスクが登録されており,タスク名から適切なキーワードを
    推測できず,キーワード検索自体が行えない場合も考えられる.
    このような場合,過去にやりとりされたメールの文面を確認し,タスクに関連付ける
    メールであるか判断する,という手順を複数回行う必要があり,手間がかかる.

\end{description}



%%%%%%%%%%%%%%%%
\subsection{メールとカレンダの連携における問題への対処}\label{sec:solution}
%%%%%%%%%%%%%%%%

問題1への対処として,メールとタスクの関連の情報をタスクとメールの両方から参照可能
にする.
これにより,メールから関連するタスクを参照でき,関連するメールとタスクの両方からの
参照を実現できる.

問題2への対処として,メールを探す際に再利用情報を用いた絞込みを可能にする.
\ref{sec:merit_of_cooperation}節で述べたように,
メールとタスクが繰返し発生している場合に,メールとカレンダの連携の利点が得られる.
\ref{sec:mail_reuse}節で述べた再利用情報は,メールの再利用により類似したメールの
送信が繰り返されていることを意味する.
このため,再利用情報をもつメールは,タスクと関連付けた際にメールとカレンダの連携の
利点が得られる有用なメールといえる.
そこで,再利用情報を利用したメールの絞込みを実現する.
絞込みを行うことにより,繰返し送信されている有用なメールを一覧できる.
適切なキーワードが分かる場合,キーワード検索と併用することで,より絞込みの精度
を向上させることができる.
これにより,確認するメールの数を減らせるため,タスクと関連付けるメールを探す際の
手間を軽減できる.

以上の対処を実現するため,再利用情報を利用してメールとタスクの関連付け
を行うシステムを提案する.






%%%%%%%%%%%%%%%%%%%%%%%%%%%%%%%%%%%%%%%%%%%%%%%%%%%%%%%%%%%%%%%%
%% 5章 システムの設計
\section{システムの設計}\label{chap:system}
%%%%%%%%%%%%%%%%%%%%%%%%%%%%%%%%%%%%%%%%%%%%%%%%%%%%%%%%%%%%%%%%

%%%%%%%%%%%%%%%%
\subsection{扱う情報}\label{sec:system_structure}
%%%%%%%%%%%%%%%%

\ref{sec:solution}節で提案した
再利用情報を利用してメールとタスクの関連付けを行うシステム(以下,提案システム)
では,以下の5つの情報を扱う.

\begin{description}
\item[M] メール\\
  過去にやりとりされ,提案システム上にアーカイブされているメールの情報である.
\item[M'] 送信されたメール\\
  ユーザが送信したメールである.
\item[R] 再利用情報\\
  メールを再利用した履歴の情報である.
\item[T] タスク\\
  提案システムと連携するカレンダシステムに登録されたタスクの情報である.
\item[M:T] メールとタスクの組(メール・タスク関連情報)\\
  関連するメールとタスクの組である.
\end{description}

また,再利用情報の構成は,
メールの再利用を促進するシステム\cite{kimuray2014a}と同様の構成とする.

%%%%%%%%%%%%%%%%
\subsection{関連付けを行う際の動作}\label{sec:system_action}
%%%%%%%%%%%%%%%%

%%%%%%%%%%%%%%%%%%%%%%%%%%%%%
%% 提案システムの構成
\insertfigure[1.0]{system_structure}{fig6}{提案システムの構成}{English Caption}
%%%%%%%%%%%%%%%%%%%%%%%%%%%%%

%%%%%%%%%%%%%%%%%%%%%%% 関連付け方法1 %%%%%%%%%%%%%%%%%%%%%%%
\subsubsection{閲覧インタフェースを利用して関連付ける場合}
\label{sec:system_action1}

メール・カレンダ共通閲覧部が提供する閲覧インタフェースを利用して,
ユーザはメールとタスクを選択し関連付けを行える.
この流れについて,\figref{system_structure}を用いて説明する.
\figref{system_structure}は,提案システムの構成を表している.
\begin{description}

\item[(A-1)] メール・タスク関連情報管理部は,各管理部から{\bf タスク(T)},
  {\bf メール(M)},{\bf 再利用情報(R)},および
  {\bf メール・タスク関連情報(M:T)}を集め,これらをメール・カレンダ共通閲覧部で
  ユーザに提示する.

\item[(A-2)] ユーザはメール・カレンダ共通閲覧部が提供する閲覧インタフェースを通じ,
  {\bf メール(M)}と{\bf タスク(T)}を選択して関連付ける.
  または,{\bf メール(M)}から{\bf タスク(T)}を作成することでメールとタスクを
  関連付ける.

\item[(A-3)] メール・タスク関連情報管理部は,メール・カレンダ共通閲覧部から渡された
  {\bf メール(M)}と{\bf タスク(T)}の情報から
  {\bf メール・タスク関連情報(M:T)}を作成し,メール・タスク関連情報DBへ保存する.

\end{description}

%%%%%%%%%%%%%%%%%%%%%%% 関連付け方法2 %%%%%%%%%%%%%%%%%%%%%%%
\subsubsection{メールの再利用により関連付ける場合}
\label{sec:system_action2}

メールの再利用時,再利用元となるメールに関連するタスクから
再利用先のメールに関連するタスクを予測し,ユーザに登録および関連付けを提案し
関連付ける方法が考えられる.
この流れについて,\figref{system_structure}を用いて説明する.

\begin{description}
\item[[B-1\textrm{]}] 再利用提案部は,テンプレートメールの作成に必要な情報を取得する.
  ユーザによる再利用操作の場合,メール・カレンダ共通閲覧部から再利用元となる
  {\bf メール(M)}を取得する.
  システムによる再利用提案の場合,{\bf 再利用情報(R)}を取得する.

\item[[B-2\textrm{]}] 再利用提案部は,再利用元となる{\bf メール(M)}に関連するタスクの有無を
  メール・タスク関連情報管理部に問い合わせる.
  再利用元となる{\bf メール(M)}に関連する{\bf タスク(T)}がある場合,
  メール・タスク関連情報管理部は,再利用先のメールに関連するタスクを予測する.
  たとえば,「第3回打合せ」というタスクから,「第4回打合せ」というタスクを予測した
  場合の動作には以下の2つの場合がある.
\begin{enumerate}

\item 「第4回打合せ」がカレンダに存在する場合\\
  「第4回打合せ」という{\bf タスク(T)}をカレンダシステムから取得し,
  再利用提案部に送信する.

\item 「第4回打合せ」がカレンダに存在しない場合\\
  「第4回打合せ」という擬似的な{\bf タスク(T)}を作成し,再利用提案部に送信する.
  擬似的なタスクは,タスクがカレンダに存在しないことを意味する.

\end{enumerate}

\item[[B-3\textrm{]}] テンプレート作成部は,[B-2]で予測した{\bf タスク(T)}と
  新たな{\bf 再利用情報(R)}を作成したテンプレートメールに付与し,
  ユーザへ送信することでメールの再利用を提案する.

\item[[B-4\textrm{]}] ユーザは受け取ったメールを修正し,修正したメールを送信することで
  メールの再利用をする.

\item[[B-5\textrm{]}] {\bf 送信されたメール(M')}は通常のメール配送とは別にメール解析部に
  渡され,{\bf 再利用情報(R)}と[B-2]で予測した{\bf タスク(T)}の有無を判別される.

\item[[B-6\textrm{]}] {\bf 送信されたメール(M')}に{\bf 再利用情報(R)}が含まれている場合,
  {\bf 再利用情報(R)}を再利用情報DBに保存する.

\item[[B-7\textrm{]}] {\bf 送信されたメール(M')}に[B-2]で予測した{\bf タスク(T)}が
  含まれている場合,閲覧インタフェースでユーザに{\bf 送信されたメール(M')}と
  予測した{\bf タスク(T)}の関連付けを提案する.
  予測した{\bf タスク(T)}が擬似的なタスクの場合,先にタスクの登録を提案する.
  ユーザがタスクの登録の提案を受け入れた場合,擬似的なタスクを{\bf タスク(T)}と
  して登録し,カレンダシステムに反映する.
  ユーザが関連付けの提案を受け入れた場合,{\bf 送信されたメール(M')}と
  {\bf タスク(T)}を関連付け,{\bf メール・タスク関連情報(M':T)}を作成して
  メール・タスク関連情報DBに保存する.

\item[[B-8\textrm{]}] {\bf 送信されたメール(M')}をメールアーカイブに保存する.

\end{description}


%%%%%%%%%%%%%%%%
\subsection{操作例}\label{sec:example_of_use}
%%%%%%%%%%%%%%%%

%%%%%%%%%%%%%%%%%%%%%%% 操作例1 %%%%%%%%%%%%%%%%%%%%%%%
\subsubsection{メールからタスクを作成}\label{sec:use_example1}

%%%%%%%%%%%%%%%%%%%%%%%%%%%%%
%% メールのドラッグ&ドロップによるタスクの作成
\insertfigure[1.0]{using_1_1}{fig7}{メールのドラッグ&ドロップによるタスクの作成}{English Caption}
%%%%%%%%%%%%%%%%%%%%%%%%%%%%%
%%%%%%%%%%%%%%%%%%%%%%%%%%%%%
%% タスクの元となった(関連付けられている)メールの表示
\insertfigure[1.0]{using_1_2}{fig8}{タスクの元となった(関連付けられている)メールの表示}{English Caption}
%%%%%%%%%%%%%%%%%%%%%%%%%%%%%

メールからタスクを作成する流れを以下に示す.

\begin{enumerate}
\item 閲覧インタフェースを表示する.
\item タスクを作成するメールを探し,\figref{using_1_1}のようにカレンダの
  日付のセルにドラッグ&ドロップする.
\item ドロップした日付にタスクが作成される.
\end{enumerate}

上記の流れで作成されたタスクの詳細を表示すると,
\figref{using_1_2}のように,タスクの元となったメールが表示され,
メールとタスクが関連付けられていることが分かる.

%%%%%%%%%%%%%%%%%%%%%%% 操作例2 %%%%%%%%%%%%%%%%%%%%%%%
\subsubsection{過去のメールを再利用}\label{sec:use_example2}
%% 計画立案に過去のタスクとメールを利用

%%%%%%%%%%%%%%%%%%%%%%%%%%%%%
%% メールの操作一覧
\insertfigure[0.7]{using_2_1}{fig9}{メールの操作一覧}{English Caption}
%%%%%%%%%%%%%%%%%%%%%%%%%%%%%

メールを再利用する流れを以下に示す.

\begin{enumerate}
\item キーワード検索と再利用情報を用いた絞込みを利用して再利用するメールを探す.
\item (1)で探し出したメールを選択し,\figref{using_2_1}の操作一覧から「このメールを再利用」を選択する.
\item 提案システムに登録されたメールアドレス宛にテンプレートメールが送信される.
  ユーザは任意のメーラでこれを受信する.
\item (3)で受信したメールの文面を修正し,送信する.
\end{enumerate}

上記の流れでメールを再利用すると,提案システムに(4)で送信したメールが蓄積される.
また,蓄積されたメールに再利用情報が付与される.

%%%%%%%%%%%%%%%%%%%%%%% 操作例3 %%%%%%%%%%%%%%%%%%%%%%%
\subsubsection{過去の予定に関連するメールを参照}\label{sec:use_example3}
%% 計画立案のために過去の予定を参照
%% 計画立案に過去のタスクとメールを利用

%%%%%%%%%%%%%%%%%%%%%%%%%%%%%
%% 再利用の関係にあるメールの一覧表示
\insertfigure[1.0]{using_3_1}{fig10}{再利用の関係にあるメールの一覧表示}{English Caption}
%%%%%%%%%%%%%%%%%%%%%%%%%%%%%

提案システムでは,タスクに関連するメールだけでなく,関連するメールの再利用元となった
メール(それ以前の同様のタスクに関連するメール)を参照できる.
過去の予定に関連するメールを参照する流れを以下に示す.

\begin{enumerate}
\item 過去のタスクの詳細を表示する.
\item \figref{using_3_1}のように,表示しているタスクに関連するメール
  だけでなく,これ以前の同様のタスクに関連するメールも一覧されている.
\item 関連するメールをクリックし,メールの内容を参照する.
\end{enumerate}




%%%%%%%%%%%%%%%%%%%%%%%%%%%%%%%%%%%%%%%%%%%%%%%%%%%%%%%%%%%%%%%%
%% 6章 おわりに
\section{おわりに}\label{chap:conclusion}
%%%%%%%%%%%%%%%%%%%%%%%%%%%%%%%%%%%%%%%%%%%%%%%%%%%%%%%%%%%%%%%%

本稿では,まず,メールとカレンダの連携について検討した.
メールとカレンダを連携させれば,作業の情報を補足する手間の軽減,
タスクに関連して送信すべきメールの把握,および
メール送信を契機とした登録すべきタスクの把握という3つの利点が得られる.

次に,メールとタスクの関連の調査と既存のアプリケーションにおけるメールとタスクの
関連付けの調査を行った.
調査の結果,調査対象のタスク全体の94\%にあたる50件のタスクがメールと関連すると
分かった.
このため,メールとタスクは関連することが多く,メールとカレンダの連携の利点
を得られる機会が多いと分かった.

続いて,既存のアプリケーションにおけるメールとタスクの関連付けの問題とその対処
を検討した.
既存のアプリケーションにおけるメールとタスクの関連付けには,
関連するタスクをメールから参照できない,メールの検索に手間がかかる
という2つの問題があると分かった.
関連するタスクをメールから参照できない問題への対処として,
メールとタスクの双方から他方を参照可能にする.
メールの検索に手間がかかる問題への対処として,
関連付けるメールを探す際に再利用情報を用いた絞込みを可能にする.

最後に,既存のアプリケーションにおけるメールとタスクの関連付けの問題への対処
を実現するシステムを設計した.

今後の課題として,
\ref{sec:task_reuse}節で述べたタスクの再利用情報を利用する手法の検討・実装がある.
現在の提案システムでは,タスクの再利用情報を利用する仕組みを取り入れていないため,
\ref{sec:merit_of_cooperation}節(2)の利点を得るにはユーザがタスク間の関連を
把握しておく必要がある.
タスクの再利用情報を利用する手法を検討・実装し,ユーザがタスク間の関連を
把握してなかったとしても\ref{sec:merit_of_cooperation}節(2)の利点を
得られるようにする.
また,提案システムの有益性の評価がある.
たとえば,再利用情報を用いた絞込みの精度の測定や,提案システムが有用な場面・条件
についてのより詳細な検討を行う.
さらに,再利用情報を最初に付与する際の手間の軽減がある.
現在の提案システムの設計では,再利用情報を最初に付与する際には,
既存のアプリケーションと同じ方法でメールを探す必要があり,手間がかかる.
そこで,関連付けるメールとタスクを抽出して関連付けを自動化する仕組みを実装し,
ユーザの負担のさらなる軽減を試みる.
この仕組みの実装においては,
メールの文書構造からタスクを抽出する手法\cite{hasegawa1999}や
概念学習・関係学習によってメールを自動分類する手法
\cite{okumura2004}\cite{yamaguchi2013}が参考になると考えている.

%%%%%%%%%%%%%%%%%%%%%%%%%%%%%%%%%%%%
% 謝辞
%%%%%%%%%%%%%%%%%%%%%%%%%%%%%%%%%%%%
\begin{acknowledgment}
本研究の一部は,科学研究費補助金・基盤研究(C)(課題番号: 26330224)による研究費を得て実施した.
\end{acknowledgment}

%%%%%%%%%%%%%%%%%%%%%%%%%%%%%%%%%%%%%%%%%%%%%%%%%%%%%%%%%%%%%%%%
%% 参考文献
\bibliographystyle{ipsjunsrt} % 情報処理学会形式(新)
\bibliography{mybibdata}

\end{document}
