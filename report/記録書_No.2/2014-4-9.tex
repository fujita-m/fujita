% Created 2014-04-04 金 09:39
\documentclass[fleqn, 14pt]{extarticle}
                    \usepackage{reportForm}
\usepackage[utf8]{inputenc}
\usepackage[T1]{fontenc}
\usepackage{fixltx2e}
\usepackage{graphicx}
\usepackage{longtable}
\usepackage{float}
\usepackage{wrapfig}
\usepackage[normalem]{ulem}
\usepackage{textcomp}
\usepackage{marvosym}
\usepackage{wasysym}
\usepackage{latexsym}
\usepackage{amssymb}
\usepackage{amstext}
\usepackage{hyperref}
\tolerance=1000
\subtitle{(2014年04月01日$\sim$2014年04月10日)}
\usepackage{strike}
\setcounter{section}{-1}
\author{乃村研究室B4\\藤田 将輝}
\date{2014年04月11日}
\title{記録書 No.1}
\hypersetup{
  pdfkeywords={},
  pdfsubject={},
  pdfcreator={Emacs 24.3.1 (Org mode 8.0.3)}}
\begin{document}

\maketitle
\section{前回ミーティングからの指導・指摘事項}
\label{sec-1}
\begin{enumerate}
\item 他人に対してのご指導やご指摘を自身への事としてとらえる.
\newline
\hfill
[4/2, メール,乃村先生]
\end{enumerate}
\section{実績}
\label{sec-2}
\subsection{研究関連}
\label{sec-2-1}
\begin{enumerate}
\item 2014年度Newグループ新B4課題に関する項目
\hfill
\label{enum-research1}
\begin{enumerate}
\item Fedora14のインストール
\hfill
\label{enum-1-A}
(100%,+100%)
\item Linuxカーネルの再構築
\hfill
\label{enum-1-B}
(100%,+100%)
\item Linuxカーネルへのシステムコールの実装
\hfill
\label{enum-1-C}
(0%,+0%)
\item システムコールの実装の手順書作成
\hfill
\label{enum-1-D}
(0%,+0%)
\item Mintの構築
\hfill
\label{enum-1-E}
(90%,+90%)
\end{enumerate}
\end{enumerate}

\subsection{研究室関連}
\label{sec-2-2}
\begin{enumerate}
\item 新B4歓迎会
\hfill
\label{enum-laboratory1}
(04/01)
\item 平成26年度新B4向けGit勉強会
\hfill
\label{enum-laboratory2}
(04/03)
\item 乃村研究室ミーティング
\hfill
\label{enum-laboratory3}
(04/04)
\item 乃村研お花見
\hfill
\label{enum-laboratory4}
(04/04)
\item 第248回Newグループ打ち合わせ
\hfill
\label{enum-laboratory5}
(04/09)
\end{enumerate}

\subsection{大学・大学院関連}
\label{sec-2-3}
\begin{enumerate}
\item 平成26年度岡山大学・大学院入学式
\hfill
\label{enum-university1}
(04/08)
\end{enumerate}

\section{詳細および反省・感想}
\label{sec-3}
\subsection{研究関連}
\label{sec-3-1}
\begin{itemize}
\item[(\ref{enum-1-A})]
実験用計算機の3つのハードディスクにFedora14をインストールした.
インストールの流れを理解するために,インストールの手順を再確認する.
\item[(\ref{enum-1-B})]
実験用計算機にMintのベースであるLinux-3.0.8をインストールした.
OSへの理解を深めるために,bzImageがOSにおいてどのような
役割を果たすかについて調べる.
\item[(\ref{enum-1-E})]
実験用計算機にMintを構築している.
複数のOSを同時に起動することで課題の達成となる.
現時点では3つ同時に起動ができていない.
3つ目のOSの起動に関するデバイス占有の仕組みを調べることを
今後の課題とする.
\end{itemize}

\subsection{研究室関連}
\label{sec-3-2}
\begin{itemize}
\item[(\ref{enum-laboratory2})]
Git勉強会に参加した.
先輩方の丁寧な説明を聞く事と,
実際に使ってみる事でGitへの理解が深まった.
バージョン管理ツールは開発において重要な役割を果たすため
,資料を読み返し復習する.
\item[(\ref{enum-laboratory3})]
乃村研究室ミーティングに参加した.
初めてのミーティングではB4は記録書は作成せず,口頭で自分の
近況を報告した.
ミーティングの流れを理解した.
\item[(\ref{enum-laboratory4})]
乃村研究室で花見をした.
研究室の方々と親睦を深められた.

\end{itemize}

\section{今後の予定}
\label{sec-4}
\subsection{研究関連}
\label{sec-4-1}
\begin{enumerate}
\item 2014年度Newグループ新B4課題に関する項目
\hfill
\begin{enumerate}

\item Linuxカーネルへのシステムコールの実装
\hfill
(04/18)
\item システムコールの実装手順書の作成
\hfill
(04/18)
\item Mintの構築
\hfill
(04/11)
\end{enumerate}
\end{enumerate}

\subsection{研究室関連}
\label{sec-4-2}
\begin{enumerate}
\item クレオフーガ交流会
\hfill
\label{enum-3}
(04/18)
\item 第249回Newグループ打ち合わせ
\hfill
\label{enum-4}
(04/21)
\end{enumerate}

\subsection{その他}
\label{sec-5}
研究室に配属されたばかりで自分の知識の無さを実感するばかりである.
今後に向けてOSや英語の知識を養っていく.
頼れる先輩方や,同期がいるというとても良い環境に置かれているため積極的に学ぶ.
また,上下関係やマナーについても身に着けていくことを目標とする.
楽しみながら研究を進める.

% Emacs 24.3.1 (Org mode 8.0.3)
\end{document}
