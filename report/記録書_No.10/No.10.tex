\documentclass[fleqn, 14pt]{extarticle}
                    \usepackage{reportForm}
\usepackage[utf8]{inputenc}
\usepackage[T1]{fontenc}
\usepackage{fixltx2e}
\usepackage{graphicx}
\usepackage{longtable}
\usepackage{float}
\usepackage{wrapfig}
\usepackage[normalem]{ulem}
\usepackage{textcomp}
\usepackage{marvosym}
\usepackage{wasysym}
\usepackage{latexsym}
\usepackage{amssymb}
\usepackage{amstext}
\usepackage{hyperref}
\usepackage{comment}
\tolerance=1000
\subtitle{(2014年09月24日$\sim$2014年10月6日)}
\usepackage{strike}
\setcounter{section}{-1}
\author{乃村研究室B4\\藤田 将輝}
\date{2014年10月03日}
\title{記録書 No.10}
\hypersetup{
  pdfkeywords={},
  pdfsubject={},
  pdfcreator={Emacs 24.3.1 (Org mode 8.0.3)}}
\begin{document}

\maketitle




\section{前回ミーティングからの指導・指摘事項}
\label{sec-1}
\begin{enumerate}
\item 特になし
\newline
\hfill

\end{enumerate}




\section{実績}
\label{sec-2}

\subsection{研究関連}
\label{sec-2-1}
\begin{enumerate}
\item 研究テーマに関する項目
\hfill
\label{enum-research1}
\begin{enumerate}

\item 参考文献の読解
\hfill
\label{enum-1-A}
(50%,+0%)
\item 使用する共有メモリ領域の検討
\hfill
\label{enum-1-B}
(55%,+50%)
\item 割り込みハンドラの動作確認
\hfill
\label{enum-1-C}
(100%,+100%)
\end{enumerate}
\item 開発に関する項目
\hfill
\label{enum-research2}
\begin{enumerate}

\item 自動ビルドスクリプトの作成
\hfill
\label{enum-2-A}
(95%,+0%)
\end{enumerate}

\item 第11回New開発打ち合わせ
\hfill
\label{enum-3}
(09/25)
\item 第261回New打ち合わせ
\hfill
\label{enum-4}
(09/30)



\end{enumerate}

\subsection{研究室関連}
\label{sec-2-2}
\begin{enumerate}
\item ノムニチ開発
\hfill
\label{enum-lab1}
(09/26,30,10/01)
\end{enumerate}
\subsection{大学・大学院関連}
\label{sec-2-3}

\begin{enumerate}
\item 特になし
\hfill
\label{enum-univ2}
\end{enumerate}





\section{詳細および反省・感想}
\label{sec-3}
\subsection{研究関連}
\label{sec-3-1}

\begin{itemize}
\item[(\ref{enum-1-C})]
割り込み元OSの占有するコアが割り込み先OSの占有するコアへIPIを送信すると,
割り込み先OSが共有メモリに格納されているデータを取得する割り込みハンドラの動作を確認した.
IPIの送信と割り込みハンドラの登録のシステムコールは山本凌平さんが作成したものであるため,コードを読解することで
これらのシステムコールの流れを理解する.
また,最終目標はNICドライバに割り込みを発生させられるようにするため,今後はNICドライバのコードを読解し,
NICドライバの割り込み発生の流れを理解する.
\end{itemize}

\section{今後の予定}
\label{sec-4}
\subsection{研究関連}
\label{sec-4-1}

\begin{enumerate}
\item 研究テーマに関する項目
\hfill
\begin{enumerate}


\item 参考文献の読解
\hfill
(10/20)

\item 使用する共有メモリ領域の検討
\hfill
(10/16)




\end{enumerate}
\item 開発に関する項目
\hfill
\begin{enumerate}

\item 自動ビルドスクリプトの作成
\hfill
(10/07)

\end{enumerate}
\item 第12回Newグループ開発打ち合わせ
\hfill
\label{enum-7}
(10/07)
\item 第262回New打ち合わせ
\hfill
\label{enum-3}
(10/16)
\end{enumerate}

\subsection{研究室関連}
\label{sec-4-2}

\begin{enumerate}



\item コンピュータセキュリティシンポジウム 2014 発表練習
\hfill
\label{enum-8}
(10/17)

\item 全体ミーティング
\hfill
\label{enum-9}
(10/17)
\item 平成26年度第2回研究室内部屋別対抗ボウリング大会
\hfill
\label{enum-10}
(10/17)
\end{enumerate}


\section{その他}
平成26年度第2回研究室内部屋別対抗ボウリング大会が近付いている.
第1回のボウリング大会では幸運のおかげで良い結果を残せた.
しかし,今回も運に恵まれるとは限らないため,事前にしっかりと練習して良い結果を残せるようにする.







\end{document}
