\documentclass[fleqn, 14pt]{extarticle}
                    \usepackage{reportForm}
\usepackage[utf8]{inputenc}
\usepackage[T1]{fontenc}
\usepackage{fixltx2e}
\usepackage{graphicx}
\usepackage{longtable}
\usepackage{float}
\usepackage{wrapfig}
\usepackage[normalem]{ulem}
\usepackage{textcomp}
\usepackage{marvosym}
\usepackage{wasysym}
\usepackage{latexsym}
\usepackage{amssymb}
\usepackage{amstext}
\usepackage{hyperref}
\usepackage{comment}
\tolerance=1000
\subtitle{(2014年08月25日$\sim$2014年09月07日)}
\usepackage{strike}
\setcounter{section}{-1}
\author{乃村研究室B4\\藤田 将輝}
\date{2014年09月08日}
\title{記録書 No.8}
\hypersetup{
  pdfkeywords={},
  pdfsubject={},
  pdfcreator={Emacs 24.3.1 (Org mode 8.0.3)}}
\begin{document}

\maketitle




\section{前回ミーティングからの指導・指摘事項}
\label{sec-1}
\begin{enumerate}
\item 特になし
\newline
\hfill

\end{enumerate}




\section{実績}
\label{sec-2}

\subsection{研究関連}
\label{sec-2-1}
\begin{enumerate}
\item 研究テーマに関する項目
\hfill
\label{enum-research1}
\begin{enumerate}

\item 参考文献の読解
\hfill
\label{enum-1-A}
(50%,+0%)
\item 使用する共有メモリ領域の検討
\hfill
\label{enum-1-B}
(5%,+5%)
\end{enumerate}
\item 開発に関する項目
\hfill
\label{enum-research2}
\begin{enumerate}

\item 自動ビルドスクリプトの作成
\hfill
\label{enum-2-A}
(90%,+0%)
\end{enumerate}

\item 第9回New開発打ち合わせ
\hfill
\label{enum-3}
(08/26)
\item 第259回New打ち合わせ
\hfill
\label{enum-4}
(09/4)



\end{enumerate}

\subsection{研究室関連}
\label{sec-2-2}
\begin{enumerate}
\item 2014年度研修会
\hfill
\label{enum-lab2}
(09/02,03)
\end{enumerate}
\subsection{大学・大学院関連}
\label{sec-2-3}

\begin{enumerate}
\item 平成27年度岡山大学大学院入学試験合格発表
\hfill
\label{enum-univ2}
(09/05)
\end{enumerate}





\section{詳細および反省・感想}
\label{sec-3}
\subsection{研究関連}
\label{sec-3-1}

\begin{itemize}
\item[(\ref{enum-4})]
2014年度後期研究計画を提出した.
研究テーマは「Mintを用いたNICドライバへの割り込み挿入手法の実現」である.
研究ではNICドライバを改変し,NIC無しで任意に割り込みを発生させる機構を実現させる.
このため,まずNICドライバの構造を理解する必要がある.
NICドライバの割り込みに関するレジスタがマッピングされているメモリのアドレスについて変更を加え,
共有メモリを用いてNICの割り込みに関するデータを受け渡そうと考えているため,
NICドライバが参照するレジスタがマッピングされているメモリのアドレスを調査する.
\end{itemize}
\subsection{研究関連}
\label{sec-3-2}

\begin{itemize}
\item[(\ref{enum-lab2})]
2014年度研修会に参加した.
ディベートとドッヂボールを通じて,SWLABと後藤研究室の皆さんと交流を深めることができた.
私はディベートが初めてであり,うまく自分の意見を話せなかった.
これに対し,先輩方は論理的に堂々と話されていた.自身との大きな差を感じた.
来年のディベートでは後輩を引っ張っていけるよう,論理的且つ堂々と意見を主張する.
\end{itemize}

\subsection{大学関連}
\label{sec-3-3}

\begin{itemize}
\item[(\ref{enum-univ2})]
先日受験した平成27年度岡山大学大学院入学試験の合格発表があった.
無事合格していた.
進路が決まったため,大学院を含めた2年半の計画を立てる.
\end{itemize}
\section{今後の予定}
\label{sec-4}
\subsection{研究関連}
\label{sec-4-1}

\begin{enumerate}
\item 研究テーマに関する項目
\hfill
\begin{enumerate}


\item 参考文献の読解
\hfill
(09/20)

\end{enumerate}
\item 開発に関する項目
\hfill
\begin{enumerate}

\item 自動ビルドスクリプトの作成
\hfill
(09/20)

\end{enumerate}
\item 第10回Newグループ開発打ち合わせ
\hfill
\label{enum-7}
(09/09)
\item 第260回New打ち合わせ
\hfill
\label{enum-3}
(09/17)
\end{enumerate}

\subsection{研究室関連}
\label{sec-4-2}

\begin{enumerate}



\item 第25回乃村杯
\hfill
\label{enum-8}
(09/08)

\end{enumerate}



\end{document}
