\documentclass[fleqn, 14pt]{extarticle}
                    \usepackage{reportForm}
\usepackage[utf8]{inputenc}
\usepackage[T1]{fontenc}
\usepackage{fixltx2e}
\usepackage{graphicx}
\usepackage{longtable}
\usepackage{float}
\usepackage{wrapfig}
\usepackage[normalem]{ulem}
\usepackage{textcomp}
\usepackage{marvosym}
\usepackage{wasysym}
\usepackage{latexsym}
\usepackage{amssymb}
\usepackage{amstext}
\usepackage{hyperref}
\usepackage{comment}
\tolerance=1000
\subtitle{(2014年06月16日$\sim$2014年06月25日)}
\usepackage{strike}
\setcounter{section}{-1}
\author{乃村研究室B4\\藤田 将輝}
\date{2014年06月26日}
\title{記録書 No.6}
\hypersetup{
  pdfkeywords={},
  pdfsubject={},
  pdfcreator={Emacs 24.3.1 (Org mode 8.0.3)}}
\begin{document}

\maketitle




\section{前回ミーティングからの指導・指摘事項}
\label{sec-1}
\begin{enumerate}
\item 相談するときは,誰に相談するかをよく考える.
\newline
\hfill
[6/16, 全体ミーティング,谷口先生]
\end{enumerate}




\section{実績}
\label{sec-2}


\subsection{研究関連}
\label{sec-2-1}
\begin{enumerate}
\item 研究テーマに関する項目
\hfill
\label{enum-research1}
\begin{enumerate}

\item 参考文献の読解
\hfill
\label{enum-1-A}
(50%,+10%)
\end{enumerate}
\item 開発に関する項目
\hfill
\label{enum-research2}
\begin{enumerate}

\item 自動ビルドスクリプトの作成
\hfill
\label{enum-2-A}
(90%,+α%)
\end{enumerate}

\item 第253回New打ち合わせ
\hfill
\label{enum-laboratory2}
(06/18)

\end{enumerate}


\subsection{研究室関連}
\label{sec-2-2}

\begin{enumerate}
\item 平成26年度M2論文紹介
\hfill
\label{enum-laboratory1}
(06/20)

\end{enumerate}

\subsection{大学・大学院関連}
\label{sec-2-3}

\begin{enumerate}
\item 特になし
\hfill
\label{enum-univ2}
\end{enumerate}





\section{詳細および反省・感想}
\label{sec-3}
\subsection{研究関連}
\label{sec-3-1}

\begin{itemize}
\item[(\ref{enum-1-A})]
山本凌平さんの特別研究報告の参考文献の1つである「Debugging operating systems with time-traveling virtual machines」[1]
の読解をしている.
この論文は,VMを用いてOSのデバッグを支援する機構を紹介するものである.
この論文は英語で書かれており,ページ数も多いため,読解がすすんでいない.
英語能力の向上のためにも,読解をすすめる.
\item[(\ref{enum-2-A})]
コードに変更があったカーネルをビルドし,再起動した際にビルドしたカーネ
ルを選択して起動し,起動に失敗すれば,変更前のカーネルにロールバックする
スクリプトを作成している.
現在,fallbackというgrubの機能を使用して,ロールバックするスクリプトを作成中である.
起動失敗後に起動するカーネルを変更できていない.
fallbackの機能と実装方法の説明を読解する.
\end{itemize}

\subsection{研究室関連}
\label{sec-3-2}
\begin{itemize}
\item[(\ref{enum-laboratory1})]
平成26年度M2論文紹介に参加した.
M2の先輩方の論文紹介を聞いた.
自身の研究とは異なる分野の論文も多かったが,わかりやすいスライドで紹介してくださったため,
理解できた.
また,発表の方法や話し方も理解できた.
自分が発表するときの参考にする.

\end{itemize}






\section{今後の予定}
\label{sec-4}
\subsection{研究関連}
\label{sec-4-1}

\begin{enumerate}
\item 研究テーマに関する項目
\hfill
\begin{enumerate}


\item 参考文献の読解
\hfill
(07/03)

\end{enumerate}
\item 開発に関する項目
\hfill
\begin{enumerate}

\item 自動ビルドスクリプトの作成
\hfill
(06/27)

\end{enumerate}
\item 第6回Newグループ開発打ち合わせ
\hfill
\label{enum-7}
(06/27)
\item 第254回New打ち合わせ
\hfill
\label{enum-3}
(07/03)
\end{enumerate}

\subsection{研究室関連}
\label{sec-4-2}

\begin{enumerate}

\item 暑気払い
\hfill
\label{enum-6}
(07/07)

\end{enumerate}
\section{その他}
6月22日にバスケットボールの大会に参加した.
この大会に向けて最近の練習を頑張っていた.
私は,運動量でチームに貢献することしかできないため,体力が続く限り動いた.
結果,私は14得点をあげることができたが,チームは負けてしまった.
原因はこちらのチームのディフェンスの甘さにあったと感じた.
このため,これからの練習ではディフェンスを強化し,次の大会での勝利を目指す.





\section{参考文献}
\renewcommand{\labelenumi}{[\arabic{enumi}]}
\begin{enumerate}
\item Samuel, T.K., George, W.D. and Peter M.C.: Debugging operating systems with time-
travelling virtual machines, Proceedings of The USENIX Annual Technical Conference,
pp.1-15(2005).
\end{enumerate}

\end{document}
