\documentclass[fleqn, 14pt]{extarticle}
                    \usepackage{reportForm}
\usepackage[utf8]{inputenc}
\usepackage[T1]{fontenc}
\usepackage{fixltx2e}
\usepackage{graphicx}
\usepackage{longtable}
\usepackage{float}
\usepackage{wrapfig}
\usepackage[normalem]{ulem}
\usepackage{textcomp}
\usepackage{marvosym}
\usepackage{wasysym}
\usepackage{latexsym}
\usepackage{amssymb}
\usepackage{amstext}
\usepackage{hyperref}
\usepackage{comment}
\tolerance=1000
\subtitle{(2014年04月28日$\sim$2014年05月16日)}
\usepackage{strike}
\setcounter{section}{-1}
\author{乃村研究室B4\\藤田 将輝}
\date{2014年05月19日}
\title{記録書 No.3}
\hypersetup{
  pdfkeywords={},
  pdfsubject={},
  pdfcreator={Emacs 24.3.1 (Org mode 8.0.3)}}
\begin{document}

\maketitle




\section{前回ミーティングからの指導・指摘事項}
\label{sec-1}
\begin{enumerate}
\item 特になし
\end{enumerate}




\section{実績}
\label{sec-2}


\subsection{研究関連}
\label{sec-2-1}
\begin{enumerate}
\item 研究テーマに関する項目
\hfill
\label{enum-research1}
\begin{enumerate}

\item 論文要約
\hfill
\label{enum-1-A}
(100%,+100%)
\item IPI送受信の確認
\hfill
\label{enum-1-B}
(50%,+50%)
\end{enumerate}
\item 開発に関する項目
\hfill
\label{enum-research2}
\begin{enumerate}

\item 自動ビルドスクリプトの作成
\hfill
\label{enum-2-A}
(70%,+70%)
\end{enumerate}
\end{enumerate}


\subsection{研究室関連}
\label{sec-2-2}

\begin{enumerate}
\item 第24回乃村杯
\hfill
\label{enum-laboratory1}
(04/28)
\item 乃村研ミーティング
\hfill
\label{enum-laboratory2}
(04/28)
\item 平成26年度B4英語勉強会
\hfill
\label{enum-laboratory3}
(05/01,08)
\item 第250回New打ち合わせ
\hfill
\label{enum-laboratory4}
(05/07)
\item 第3回Newグループ開発打ち合わせ
\hfill
\label{enum-laboratory5}
(05/13)
\end{enumerate}

\subsection{大学・大学院関連}
\label{sec-2-3}

\begin{enumerate}
\item 特になし
\hfill
\end{enumerate}





\section{詳細および反省・感想}
\label{sec-3}
\subsection{研究関連}
\label{sec-3-1}

\begin{itemize}
\item[(\ref{enum-1-A})]
研究テーマが決定した.
研究テーマは「Mintオペレーティングシステムを用いた割り込み処理のデバッグ支援環境の提案」である.
これは山本さんの研究テーマの引継ぎである.このため,山本さんの特別研究報告書である
「Mintオペレーティングシステムを用いた割り込み処理のデバッグ支援環境の提案」[1]
を要約した.
現在はInter-Processor Interrupt(IPI)を用いて,CPUへ割り込みを発生させられている.
今後の課題は,NICドライバへの割り込みのデバッグ環境の構成である.
割り込み処理における知識が必要であると感じたため,これからは
残された文書を読むことと,割り込み手法を実行することで割り込み処理の理解に努める.

\end{itemize}

\subsection{研究室関連}
\label{sec-3-2}

\begin{itemize}
\item[(\ref{enum-laboratory1})]
第24回乃村杯に参加した.
初参加の乃村杯の競技はビリヤードであった.
研究室の方々との交流を深められた.
ビリヤードは思っていたよりも難しく,結果は13人中の11位であった.
\item[(\ref{enum-laboratory3})]
平成26年度B4英語勉強会に参加した.
最初に参加したときよりもスコアが取れるようになってきた.
公開TOEICが5月25日にあるため,550点を目指して勉強する.
\end{itemize}








\section{今後の予定}
\label{sec-4}
\subsection{研究関連}
\label{sec-4-1}

\begin{enumerate}
\item 研究テーマに関する項目
\hfill
\begin{enumerate}

\item IPI送受信の確認
\hfill
(05/23)

\end{enumerate}
\item 開発に関する項目
\hfill
\begin{enumerate}

\item 自動ビルドスクリプトの作成
\hfill
(05/23)

\end{enumerate}
\end{enumerate}

\subsection{研究室関連}
\label{sec-4-2}

\begin{enumerate}
\item 第4回Newグループ開発打ち合わせ
\hfill
\label{enum-3}
(05/26)
\item 乃村研ミーティング
\hfill
\label{enum-4}
(05/29)
\end{enumerate}
\subsection{大学関連}
\begin{enumerate}
\item 情報化における職業
\hfill
\label{enum-5}
(05/23,30)
\item 公開TOEIC
\hfill
\label{enum-6}
(05/25)
\item カレッジTOEIC
\hfill
\label{enum-7}
(05/31)
\end{enumerate}

\section{参考文献}
\renewcommand{\labelenumi}{[\arabic{enumi}]}
\begin{enumerate}
\item 山本凌平:Mintオペレーティングシステムを用いた割り込み処理のデバッグ支援環境の提案,
岡山大学工学部情報工学科特別研究報告(2014).
\end{enumerate}

\end{document}
