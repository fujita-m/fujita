\documentclass[fleqn, 14pt]{extarticle}
\usepackage{reportForm}
\usepackage[utf8]{inputenc}
\usepackage[T1]{fontenc}
\usepackage{fixltx2e}
\usepackage{graphicx}
\usepackage{longtable}
\usepackage{float}
\usepackage{wrapfig}
\usepackage[normalem]{ulem}
\usepackage{textcomp}
\usepackage{marvosym}
\usepackage{wasysym}
\usepackage{latexsym}
\usepackage{amssymb}
\usepackage{amstext}
\usepackage{hyperref}
\usepackage{comment}
\tolerance=1000
\subtitle{(2014年10月17日$\sim$2014年11月3日)}
\usepackage{strike}
\setcounter{section}{-1}
\author{乃村研究室B4\\藤田 将輝}
\date{2014年11月04日}
\title{記録書 No.12}
\hypersetup{
  pdfkeywords={},
  pdfsubject={},
  pdfcreator={Emacs 24.3.1 (Org mode 8.0.3)}}
\begin{document}

\maketitle




\section{前回ミーティングからの指導・指摘事項}
\label{sec-1}
\begin{enumerate}
\item 特になし
\newline
\hfill

\end{enumerate}




\section{実績}
\label{sec-2}

\subsection{研究関連}
\label{sec-2-1}
\begin{enumerate}
\item 研究テーマに関する項目
\hfill
\label{enum-research1}
\begin{enumerate}

\item 参考文献の読解
\hfill
\label{enum-1-A}
(50%,+0%)
\item 使用する共有メモリ領域の検討
\hfill
\label{enum-1-B}
(55%,+0%)
\item NICのデバイスドライバの改変箇所の調査
\hfill
\label{enum-1-C}
(30%,+10%)
\item NICドライバの改変
\hfill
\label{enum-1-D}
(10%,+10%)
\end{enumerate}
\item 開発に関する項目
\hfill
\label{enum-research2}
\begin{enumerate}

\item 自動ビルドスクリプトの作成
\hfill
\label{enum-2-A}
(95%,+0%)
\item debianでのMintの構築
\hfill
\label{enum-2-A}
(0%,+0%)
\end{enumerate}

\item 第13回New開発打ち合わせ
\hfill
\label{enum-3}
(10/22)
\item 第263回New打ち合わせ
\hfill
\label{enum-4}
(10/31)



\end{enumerate}

\subsection{研究室関連}
\label{sec-2-2}
\begin{enumerate}
\item 平成26年度第2回研究室内部屋別対抗ボウリング大会
\hfill
\label{lab-ive1}
(10/17)
\item 平成26年度M1論文紹介
\hfill
\label{lab-ive1}
(10/30)
\end{enumerate}
\subsection{大学・大学院関連}
\label{sec-2-3}
\begin{enumerate}
\item 岡山大学際
\hfill
\label{enum-univ1}
(11/1,2)
\end{enumerate}





\section{詳細および反省・感想}
\label{sec-3}
\subsection{研究関連}
\label{sec-3-1}

\begin{itemize}
\item[(\ref{enum-1-D})]
NICドライバでのパケットの授受をMintの共有メモリでするため,NICドライバを改変している.
現在はNICドライバの送信処理の関数内でパケットらしきものを共有メモリに格納できることを確かめた.
今後はパケットの構造を調査し,擬似的なパケットを作成できるようにする.
\end{itemize}

\subsection{研究室関連}
\begin{itemize}
\item[(\ref{lab-ive1})]
平成26年度M1論文紹介に参加した.
先輩方の堂々とした態度とわかりやすいスライドの構成が参考になった.
また,限られた時間内で紹介したいことを伝えきることに難しさを感じた.
自身が発表するときには先輩方の発表を参考にし,よく練習して発表に臨む.
\end{itemize}

\section{今後の予定}
\label{sec-4}
\subsection{研究関連}
\label{sec-4-1}

\begin{enumerate}
\item 研究テーマに関する項目
\hfill
\begin{enumerate}


\item 参考文献の読解
\hfill
(11/12)

\item 使用する共有メモリ領域の検討
\hfill
(11/11)

\item NICのデバイスドライバの改変箇所の調査
\hfill
(11/20)

\item NICドライバの改変
\hfill
(11/30)

\end{enumerate}
\item 開発に関する項目
\hfill
\begin{enumerate}

\item 自動ビルドスクリプトの作成
\hfill
(11/30)

\item debianでのMintの構築
\hfill
(11/30)

\end{enumerate}
\item 第14回Newグループ開発打ち合わせ
\hfill
\label{enum-7}
(11/05)
\item 第264New打ち合わせ
\hfill
\label{enum-8}
(11/10)
\end{enumerate}

\subsection{研究室関連}
\label{sec-4-2}

\begin{enumerate}

\item 全体ミーティング
\hfill
\label{enum-10}
(11/13)
\item 特別研究中間報告
\hfill
\label{enum-11}
(11/21)
\end{enumerate}









\end{document}
