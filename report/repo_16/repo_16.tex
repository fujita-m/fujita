\documentclass[fleqn, 14pt]{extarticle}
\usepackage{reportForm}
\usepackage[utf8]{inputenc}
\usepackage[T1]{fontenc}
\usepackage{fixltx2e}
\usepackage{graphicx}
\usepackage{longtable}
\usepackage{float}
\usepackage{wrapfig}
\usepackage[normalem]{ulem}
\usepackage{textcomp}
\usepackage{marvosym}
\usepackage{wasysym}
\usepackage{latexsym}
\usepackage{amssymb}
\usepackage{amstext}
\usepackage{hyperref}
\usepackage{comment}
\tolerance=1000
\subtitle{(2014年12月19日$\sim$2015年1月4日)}
\usepackage{strike}
\setcounter{section}{-1}
\author{乃村研究室B4\\藤田 将輝}
\date{2015年1月5日}
\title{記録書 No.16}
\hypersetup{
  pdfkeywords={},
  pdfsubject={},
  pdfcreator={Emacs 24.3.1 (Org mode 8.0.3)}}
\begin{document}
\maketitle
\section{前回ミーティングからの指導・指摘事項}
\label{sec-1}
\begin{enumerate}
\item 特になし
\newline
\hfill

\end{enumerate}




\section{実績}
\label{sec-2}

\subsection{研究関連}
\label{sec-2-1}
\begin{enumerate}
    \item 研究テーマに関する項目
    \hfill
    \label{enum-research1}
    \begin{enumerate}

        \item 参考文献の読解
        \hfill
        \label{enum-1-A}
        (50%,+0%)
        \item 使用する共有メモリ領域の検討
        \hfill
        \label{enum-1-B}
        (75%,+0%)
        \item NICのデバイスドライバの改変箇所の調査
        \hfill
        \label{enum-1-C}
        (50%,+0%)
        \item NICドライバの改変
        \hfill
        \label{enum-1-D}
        (50%,+0%)
        \item 特別研究報告書
        \hfill
        \label{enum-1-E}
        (20%,+10%)

    \end{enumerate}
    \item 開発に関する項目
    \hfill
    \label{enum-research2}
    \begin{enumerate}

        \item 自動ビルドスクリプトの作成
        \hfill
        \label{enum-2-A}
        (95%,+0%)
        \item debianでのMintの構築
        \hfill
        \label{enum-2-A}
        (50%,+0%)
    \end{enumerate}
    \item 第267回New打ち合わせ 
    \hfill
    \label{enum-research3}
    (12/19)
    \end{enumerate}

\subsection{研究室関連}
\label{sec-2-2}
\begin{enumerate}
\item 全体ミーティング 
\hfill
\label{enum-lab1}
(12/19)
\item 乃村研納会
\hfill
\label{enum-lab3}
(12/25)
\end{enumerate}


\section{詳細および反省・感想}
\label{sec-3}
%\setcounter{subsection}{1}
\subsection{研究関連}
\label{sec-3-1}

\begin{itemize}
\item[(\ref{enum-1-E})]
    特別研究報告書を作成している.
    目次案を完成することができたため,図を作成しその説明を
    考えている.
    分かりやすい構成を意識して作成する.
    これに並行して研究もすすめる.
\end{itemize}


\section{今後の予定}
\label{sec-4}
\subsection{研究関連}
\label{sec-4-1}

\begin{enumerate}
\item 研究テーマに関する項目
\hfill
\begin{enumerate}


\item 参考文献の読解
\hfill
(1月下旬)

\item 使用する共有メモリ領域の検討
\hfill
(1月中旬)

\item NICのデバイスドライバの改変箇所の調査
\hfill
(1月中旬)

\item NICドライバの改変
\hfill
(1月中旬)

\item 特別研究報告書 
\hfill
(2月上旬)


\end{enumerate}
\item 開発に関する項目
\hfill
\begin{enumerate}

\item 自動ビルドスクリプトの作成
\hfill
(2月中旬)

\item debianでのMintの構築
\hfill
(2月中旬)

\end{enumerate}
\item 第268回New打ち合わせ
\hfill
\label{enum-7}
(1/6,7)
\item 第17回Newグループ開発打ち合わせ
\hfill
\label{enum-8}
(2/16)
\end{enumerate}

\subsection{研究室関連}
\label{sec-4-2}

\begin{enumerate}


\item 書初め 
\hfill
\label{enum-16}
(1/05)

\end{enumerate}

\subsection{大学関連}
\begin{enumerate}
\item 特になし
\end{enumerate}

\section{その他}
1月2日に発熱してしまった.
その日に高校のクラスの同窓会があったが欠席してしまった.
とても残念であった.
次の日にはなんとか動ける程度にはなったため,
高校の部活動(バスケットボール部)に参加した.
無理はしないようにしていたがプレーしているうちに熱くなってしまい,全力でプレーした.
結果,少し体調が悪くなってしまった.
卒業論文もあるため,今後は無理をせず体調を最優先する.









\end{document}
