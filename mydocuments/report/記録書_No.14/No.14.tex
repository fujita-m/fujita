\documentclass[fleqn, 14pt]{extarticle}
\usepackage{reportForm}
\usepackage[utf8]{inputenc}
\usepackage[T1]{fontenc}
\usepackage{fixltx2e}
\usepackage{graphicx}
\usepackage{longtable}
\usepackage{float}
\usepackage{wrapfig}
\usepackage[normalem]{ulem}
\usepackage{textcomp}
\usepackage{marvosym}
\usepackage{wasysym}
\usepackage{latexsym}
\usepackage{amssymb}
\usepackage{amstext}
\usepackage{hyperref}
\usepackage{comment}
\tolerance=1000
\subtitle{(2014年11月13日$\sim$2014年11月22日)}
\usepackage{strike}
\setcounter{section}{-1}
\author{乃村研究室B4\\藤田 将輝}
\date{2014年11月23日}
\title{記録書 No.14}
\hypersetup{
  pdfkeywords={},
  pdfsubject={},
  pdfcreator={Emacs 24.3.1 (Org mode 8.0.3)}}
\begin{document}

\maketitle




\section{前回ミーティングからの指導・指摘事項}
\label{sec-1}
\begin{enumerate}
\item 特になし
\newline
\hfill

\end{enumerate}




\section{実績}
\label{sec-2}

\subsection{研究関連}
\label{sec-2-1}
\begin{enumerate}
\item 研究テーマに関する項目
\hfill
\label{enum-research1}
\begin{enumerate}

\item 参考文献の読解
\hfill
\label{enum-1-A}
(50%,+0%)
\item 使用する共有メモリ領域の検討
\hfill
\label{enum-1-B}
(55%,+0%)
\item NICのデバイスドライバの改変箇所の調査
\hfill
\label{enum-1-C}
(30%,+0%)
\item NICドライバの改変
\hfill
\label{enum-1-D}
(30%,+20%)
\item 中間発表スライドの作成
\hfill
\label{enum-1-E}
(100%,+20%)
\end{enumerate}
\item 開発に関する項目
\hfill
\label{enum-research2}
\begin{enumerate}

\item 自動ビルドスクリプトの作成
\hfill
\label{enum-2-A}
(95%,+0%)
\item debianでのMintの構築
\hfill
\label{enum-2-A}
(0%,+0%)
\end{enumerate}




\end{enumerate}

\subsection{研究室関連}
\label{sec-2-2}
\begin{enumerate}
\item 中間発表発表練習
\hfill
\label{enum-lab1}
(11/18)
\item 平成26年度特別研究中間報告会
\hfill
\label{enum-lab2}
(11/21)
\item 平成26年度M2研究進捗報告会
\hfill
\label{enum-11}
(11/21)
\end{enumerate}





\section{詳細および反省・感想}
\label{sec-3}
\setcounter{subsection}{1}
\subsection{研究室関連}
\label{sec-3-1}

\begin{itemize}
\item[(\ref{enum-lab2})]
平成26年度特別研究中間報告会(以下,中間発表)に参加した.
題目は「Mintオペレーティングシステムを用いたNICドライバの割り込みデバッグ手法の実現」である.
初めての発表であったため,非常に緊張した.
しかし,中間発表前に何度も練習したことと,先生と先輩方のご指導のおかげで時間内に発表できた.
ご指導をくださった乃村先生,先輩の皆さん,ありがとうございました.
反省としては,質疑の際に,自信をもって答えられなかったため,今後は質疑を意識した練習をする.
\end{itemize}


\section{今後の予定}
\label{sec-4}
\subsection{研究関連}
\label{sec-4-1}

\begin{enumerate}
\item 研究テーマに関する項目
\hfill
\begin{enumerate}


\item 参考文献の読解
\hfill
(11/30)

\item 使用する共有メモリ領域の検討
\hfill
(11/30)

\item NICのデバイスドライバの改変箇所の調査
\hfill
(11/30)

\item NICドライバの改変
\hfill
(11/30)

\end{enumerate}
\item 開発に関する項目
\hfill
\begin{enumerate}

\item 自動ビルドスクリプトの作成
\hfill
(11/30)

\item debianでのMintの構築
\hfill
(11/30)

\end{enumerate}
\item 第265回New打ち合わせ
\hfill
\label{enum-7}
(11/25)
\item 第15回Newグループ開発打ち合わせ
\hfill
\label{enum-8}
(12/1)
\end{enumerate}

\subsection{研究室関連}
\label{sec-4-2}

\begin{enumerate}


\item 乃村研忘年会
\hfill
\label{enum-12}
(11/23)

\item SWLAB忘年会
\hfill
\label{enum-13}
(12/5)
\end{enumerate}

\subsection{大学関連}
\begin{enumerate}
\item 特になし
\end{enumerate}











\end{document}
