\documentclass[fleqn, 14pt]{extarticle}
\usepackage{reportForm}
\usepackage[utf8]{inputenc}
\usepackage[T1]{fontenc}
\usepackage{fixltx2e}
\usepackage{graphicx}
\usepackage{longtable}
\usepackage{float}
\usepackage{wrapfig}
\usepackage[normalem]{ulem}
\usepackage{textcomp}
\usepackage{marvosym}
\usepackage{setspace}
\usepackage{wasysym}
\usepackage{latexsym}
\usepackage{amssymb}
\usepackage{amstext}
\usepackage{hyperref}
\usepackage{comment}
\tolerance=1000
\subtitle{(2016年01月18日$\sim$2016年02月03日)}
\usepackage{strike}
\setcounter{section}{-1}
\author{乃村研究室M1\\藤田 将輝}
\date{2016年02月04日}
\title{記録書 No.44}
\hypersetup{
    pdfkeywords={},
    pdfsubject={},
    pdfcreator={Emacs 24.3.1 (Org mode 8.0.3)}}
    \begin{document}
    \maketitle

    \section{前回ミーティングからの指導・指摘事項}
    \label{sec-1}
    \begin{enumerate}

        \item 主語を明確にする.
            \hfill
            [02/01,メール,谷口先生]

        \item 的や性など,曖昧な表現は書かない.
            \hfill
            [02/01,メール,谷口先生]

        \item 行ったことを書くのではなく,行ったことで何が分かったかを書く.
            \newline
            \hfill
            [02/02,203,谷口先生]


    \end{enumerate}

    \section{実績}
    \label{sec-2}

    \subsection{研究関連}
    \label{sec-2-1}
    \begin{enumerate}

        \item 研究テーマに関する項目
            \hfill
            \label{enum-research1}
            \begin{enumerate}

                \item 参考文献の読解
                    \hfill
                    \label{enum-1-A}
                    (50%,+0%)

                \item バグの再現
                    \hfill
                    \label{enum-1-B}
                    (0%,+0%)

                \item 第136回システムソフトウェアとオペレーティング・システム研究会原稿執筆
                    \hfill
                    \label{enum-1-C}
                    (100%,+70%)

                \item 第136回システムソフトウェアとオペレーティング・システム研究会スライド作成
                    \hfill
                    \label{enum-1-D}
                    (10%,+0%)

            \end{enumerate}

        \item 開発に関する項目
            \hfill
            \label{enum-research2}
            \begin{enumerate}

                \item 自動ビルドスクリプトの作成
                    \hfill
                    \label{enum-2-A}
                    (95%,+0%)

            \end{enumerate}

        \item 第294回New打ち合わせ
            \hfill
            \label{enum-research3}
            (01/27)

    \end{enumerate}
    \subsection{研究室関連}
    \label{sec-2-2}
    \begin{enumerate}

        \item 乃村研高校生訪問
            \hfill
            \label{enum-18}
            (01/26)

    \end{enumerate}

    \subsection{大学院関連}
    \label{sec2-3}
    \begin{enumerate}

        \item 特になし.
            \hfill
            \label{enum-univ1}

    \end{enumerate}

    \section{詳細および反省・感想}
    \label{sec-3}

    %\setcounter{subsection}{1}
    \subsection{研究関連}
    \label{sec-3-2}

    \begin{itemize}

        \item[(\ref{enum-1-C})]
            \begin{spacing}{1.2}
            第136回システムソフトウェアとオペレーティング・システム研究会の
            原稿を執筆を完了し,提出した.原稿執筆にあたって,ご指導いただいた
            谷口先生,乃村先生,ならびに増田さん,ありがとうございました.
            引き続き,スライドを作成し,発表練習をする.
            今回の原稿執筆で頂いたご指導を今後の文章作成に活かす.
            \end{spacing}

    \end{itemize}

    \section{今後の予定}
    \label{sec-4}

    \subsection{研究関連}
    \label{sec-4-1}

    \begin{enumerate}

        \item 研究テーマに関する項目
            \hfill
            \begin{enumerate}

                \item 参考文献の読解
                    \hfill
                    (2月中旬)

                \item バグの再現
                    \hfill
                    (3月上旬)

                \item 第136回システムソフトウェアとオペレーティング・システム研究会スライド作成
                    \hfill
                    (2月29日)


            \end{enumerate}

        \item 開発に関する項目
            \hfill
            \begin{enumerate}

                \item 自動ビルドスクリプトの作成
                    \hfill
                    (3月中旬)

            \end{enumerate}

        \item 第295回New打ち合わせ
            \hfill
            \label{enum-7}
            (02/08)

    \end{enumerate}

    \subsection{研究室関連}
    \label{sec-4-2}

    \begin{enumerate}

        \item 乃村研ミーティング
            \hfill
            \label{enum-18}
            (02/05)

        \item 研究室説明会
            \hfill
            \label{enum-18}
            (02/16)
    \end{enumerate}

    \subsection{大学院関連}
    \begin{enumerate}

        \item 特になし
            \hfill
            \label{enum-17}

    \end{enumerate}
    \subsection{学会情報} 
    \begin{enumerate}
        \item 第136回システムソフトウェアとオペレーティングシステム研究会\\
            開催日時:2016年2月29日(月)\\
            開催場所:理化学研究所計算科学研究機構\\
            申込締切:\sout{2016年1月12日(火)}\\
            原稿締め切り:\sout{2016年2月2日(火)}
    \end{enumerate}

    \end{document}
