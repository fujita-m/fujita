\documentclass[fleqn, 12pt]{extarticle}
\usepackage{reportForm}
\usepackage[utf8]{inputenc}
\usepackage[T1]{fontenc}
\usepackage{fixltx2e}
\usepackage{graphicx}
\usepackage{longtable}
\usepackage{float}
\usepackage{wrapfig}
\usepackage[normalem]{ulem}
\usepackage{textcomp}
\usepackage{marvosym}
\usepackage{setspace}
\usepackage{wasysym}
\usepackage{latexsym}
\usepackage{amssymb}
\usepackage{amstext}
\usepackage{hyperref}
\usepackage{comment}
\tolerance=1000
\subtitle{(2016年06月01日$\sim$2016年06月28日)}
\usepackage{strike}
\setcounter{section}{-1}
\author{乃村研究室M2\\藤田 将輝}
\date{2016年06月29日}
\title{記録書 No.48}
\hypersetup{
    pdfkeywords={},
    pdfsubject={},
    pdfcreator={Emacs 24.3.1 (Org mode 8.0.3)}}
    \begin{document}
    \maketitle

    \section{前回ミーティングからの指導・指摘事項}
    \label{sec-1}
    \begin{enumerate}

        \item 特になし.

    \end{enumerate}

    \section{実績}
    \label{sec-2}

    \subsection{研究関連}
    \label{sec-2-1}
    \begin{enumerate}

        \item 研究テーマに関する項目
            \hfill
            \label{enum-research1}
            \begin{enumerate}

                \item 参考文献の読解
                    \hfill
                    \label{enum-1-A}
                    (50%,+0%)

                \item バグの再現
                    \hfill
                    \label{enum-1-B}
                    (0%,+0%)

                \item バグの調査
                    \hfill
                    \label{enum-1-B}
                    (30%,+30%)

            \end{enumerate}

        \item 開発に関する項目
            \hfill
            \label{enum-research2}
            \begin{enumerate}

                \item 自動ビルドスクリプトの作成
                    \hfill
                    \label{enum-2-A}
                    (95%,+0%)

            \end{enumerate}

        \item 第303回New打ち合わせ
            \hfill
            \label{enum-research3}
            (06/15)

        \item 平成28年度第1回New輪講(1/4)
            \hfill
            \label{enum-research4}
            (06/20)

        \item 平成28年度第1回New輪講(2/4)
            \hfill
            \label{enum-research5}
            (06/23)

    \end{enumerate}
    \subsection{研究室関連}
    \label{sec-2-2}
    \begin{enumerate}

        \item 平成28年度部屋別対抗ボウリング大会
            \hfill
            \label{enum-18}
            (06/07)

        \item 乃村研お楽しみ会
            \hfill
            \label{enum-18}
            (06/15)

    \end{enumerate}

    \subsection{大学院関連}
    \label{sec2-3}
    \begin{enumerate}

        \item 特になし.
            \hfill
            \label{enum-univ1}

    \end{enumerate}

    \subsection{就職活動関連}
    \label{sec2-3}
    \begin{enumerate}

        \item 三菱電機最終面接
            \hfill
            \label{enum-syukatsu}
            (06/01)

    \end{enumerate}

    \section{詳細および反省・感想}
    \label{sec-3}

    %\setcounter{subsection}{1}
    \subsection{研究関連}
    \label{sec-3-2}

    \begin{itemize}

        \item[(\ref{enum-research4})]
            \begin{spacing}{1.2}
                平成28年度第1回New輪講に参加し,発表した.輪講では,IntelのプロセッサマニュアルをNewグループの学生で読解し,
                各々が担当した章について,スライドを用いて紹介する.私が担当したのは,Intelプロセッサにおけるデバッグ機能の
                章である.具体的には,チェックポイント機能と最新分岐記録機能について紹介した.
                チェックポイント機能とは特定の条件を満たすとソフトウェアが動作を止め,その時点でのレジスタやメモリの値を確認できる機能である.
                最新分岐記録機能とは分岐が発生した際の,分岐元と分岐先を記録しておくことで,分岐から分岐までのシングルステップ実行を可能にし,
                バグの発生源の特定を補助する機能である.
                各機能についてどのようなレジスタを
                用いるのか,どのフラグがどのような操作を許可するか等を紹介した.紹介したハードウェア機能,OS,およびデバッガの
                関連を示すと分かりやすいとご指導を頂いた.これについて,デバッガのソースコードを調査し,どのような処理を呼び出しているかを
                特定しようと考えている.
            \end{spacing}

    \end{itemize}

    \setcounter{subsection}{3}
    \subsection{就職活動関連}
    \begin{itemize}

        \item[(\ref{enum-syukatsu})]
            \begin{spacing}{1.2}
                三菱電機の最終面接を受けた.結果,内々定を頂き,就職活動を終了した.
            \end{spacing}

    \end{itemize}


    \section{今後の予定}
    \label{sec-4}

    \subsection{研究関連}
    \label{sec-4-1}

    \begin{enumerate}

        \item 研究テーマに関する項目
            \hfill
            \begin{enumerate}

                \item 参考文献の読解
                    \hfill
                    (7月中旬)

                \item バグの再現
                    \hfill
                    (7月下旬)

                \item バグの調査
                    \hfill
                    (7月中旬)

            \end{enumerate}

        \item 開発に関する項目
            \hfill
            \begin{enumerate}

                \item 自動ビルドスクリプトの作成
                    \hfill
                    (7月中旬)

            \end{enumerate}

        \item 第304回New打ち合わせ
            \hfill
            \label{enum-7}
            (06/30)

        \item 平成28年度第1回New輪講(3/4)
            \hfill
            \label{enum-7}
            (07/01)

        \item 平成28年度第1回New輪講(4/4)
            \hfill
            \label{enum-7}
            (07/11)

    \end{enumerate}

    \subsection{研究室関連}
    \label{sec-4-2}

    \begin{enumerate}

        \item 乃村研ミーティング
            \hfill
            \label{enum-18}
            (07/04)

        \item 香川大学訪問
            \hfill
            \label{enum-18}
            (07/06)

        \item 暑気払い
            \hfill
            \label{enum-18}
            (07/12)

    \end{enumerate}

    \subsection{大学院関連}
    \begin{enumerate}

        \item 特になし
            \hfill
            \label{enum-17}

    \end{enumerate}

    \end{document}
