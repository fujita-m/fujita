\documentclass[fleqn, 14pt]{extarticlej}
\usepackage{sty/reportForm}
\usepackage{hyperref}
\tolerance=1000
\subtitle{(2015年06月25日$\sim$2015年07月21日)}
\usepackage{sty/strike}
\usepackage{nutils}
\setcounter{section}{0}
\author{乃村研究室M1\\藤田 将輝}
\date{2015年07月22日}
\title{記録書 No.29}
\begin{document}

\maketitle
\section{実績,詳細,および反省・感想}
\subsection{研究関連}
\label{sec-2-1}
\begin{enumerate}
    \item 研究テーマに関する項目
    \hfill
    \label{enum-research1}
    \begin{enumerate}
        \item 参考文献の読解
        \hfill
        \label{enum-1-A}
        (50%,+0%)
        \item 使用する共有メモリ領域の検討
        \hfill
        \label{enum-1-B}
        (75%,+0%)
        \item NICのデバイスドライバの改変箇所の調査
        \hfill
        \label{enum-1-C}
        (50%,+0%)
        \item パケットの作成
        \hfill
        \label{enum-1-D}
        (91%,+1%)\\
        パケットを作成するアプリケーションを作成している.
        libnetというライブラリを用いることで,パケットを作成できる.
        作成したパケットがデバッグ支援環境において正常に処理されるかどうかを
        実験してみた所,正常に処理されていないことがわかった.
        この要因として,IPv4ヘッダの構成に不備があることがわかった.
        このため,IPv4ヘッダを作成する関数について,ドキュメントを読むことで
        理解を深める.
        正常に処理されるパケットを作成できたら,連続でパケットを送信した際に,
        どの程度連続でパケットが処理されるかを調査する.
\end{enumerate}
    \item 開発に関する項目
        \hfill
        \label{enum-research2}
        \begin{enumerate}
            \item 自動ビルドスクリプトの作成
                \hfill
                \label{enum-2-A}
                (95%,+0%)
        \end{enumerate}
    \item 第23回New開発打ち合わせ
        \hfill
        \label{enum-research3}
        (7/01)
    \item 第279回New打ち合わせ
        \hfill
        \label{enum-research3}
        (7/09)
    \item 第280回New打ち合わせ
        \hfill
        \label{enum-research3}
        (7/16)

\end{enumerate}
\subsection{研究室関連}
\begin{enumerate}
    \item M2論文紹介
        \hfill
        (06/26)
    \item 高校生訪問
        \hfill
        (07/02)
    \item 乃村研BBQ
        \hfill
        (07/05)
    \item 暑気払い
        \hfill
        (07/06)
    \item 全体ミーティング
        \hfill
        (07/10)
\end{enumerate}

\subsection{大学院関連}
\begin{enumerate}
    \item 特になし
\end{enumerate}

\section{KPT}
\subsection{GOOD}
\begin{enumerate}
    \item 論文紹介をする講義が全て終わった.
    \item 海の日に久々に会う先輩たちと遊んだ.
\end{enumerate}

\subsection{BAD}
\begin{enumerate}
    \item 海の日に昼から飲んだ.
    \item 台風の日にダラダラ過ごしてしまった.
\end{enumerate}

\section{近況報告}
最近,家ですることが動画の視聴程度になってしまっている.
それ以外は,睡眠しかしていない.
なにか趣味が欲しいと思っている.
短い時間でもできるような趣味を知っている方は,
教えてください.

\end{document}
