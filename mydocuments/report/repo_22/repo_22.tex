\documentclass[fleqn, 14pt]{extarticle}
\usepackage{reportForm}
\usepackage[utf8]{inputenc}
\usepackage[T1]{fontenc}
\usepackage{fixltx2e}
\usepackage{graphicx}
\usepackage{longtable}
\usepackage{float}
\usepackage{wrapfig}
\usepackage[normalem]{ulem}
\usepackage{textcomp}
\usepackage{marvosym}
\usepackage{wasysym}
\usepackage{latexsym}
\usepackage{amssymb}
\usepackage{amstext}
\usepackage{hyperref}
\usepackage{comment}
\tolerance=1000
\subtitle{(2015年3月27日$\sim$2015年4月15日)}
\usepackage{strike}
\setcounter{section}{-1}
\author{乃村研究室B4\\藤田 将輝}
\date{2015年4月16日}
\title{記録書 No.22}
\hypersetup{
  pdfkeywords={},
  pdfsubject={},
  pdfcreator={Emacs 24.3.1 (Org mode 8.0.3)}}
\begin{document}
\maketitle
\section{前回ミーティングからの指導・指摘事項}
\label{sec-1}
\begin{enumerate}
\item 特になし
\newline
\hfill

\end{enumerate}




\section{実績}
\label{sec-2}

\subsection{研究関連}
\label{sec-2-1}
\begin{enumerate}
    \item 研究テーマに関する項目
    \hfill
    \label{enum-research1}
    \begin{enumerate}

        \item 参考文献の読解
        \hfill
        \label{enum-1-A}
        (50%,+0%)
        \item 使用する共有メモリ領域の検討
        \hfill
        \label{enum-1-B}
        (75%,+0%)
        \item NICのデバイスドライバの改変箇所の調査
        \hfill
        \label{enum-1-C}
        (50%,+0%)
        \item パケット受信処理の実装
        \hfill
        \label{enum-1-C}
        (20%,+20%)

    \end{enumerate}
    \item 開発に関する項目
    \hfill
    \label{enum-research2}
    \begin{enumerate}

        \item 自動ビルドスクリプトの作成
        \hfill
        \label{enum-2-A}
        (95%,+0%)
        \item debianでのMintの構築
        \hfill
        \label{enum-2-A}
        (95%,+45%)
    \end{enumerate}
    \item 第271回New打ち合わせ 
    \hfill
    \label{enum-research3}
    (3/30)
    \item 第272回New打ち合わせ 
    \hfill
    \label{enum-research3}
    (4/6)
    \item 第18回New開発打ち合わせ
    \hfill
    \label{enum-research3}
    (4/13)
    \end{enumerate}

\subsection{研究室関連}
\label{sec-2-2}
\begin{enumerate}
\item 全体ミーティング 
\hfill
\label{enum-lab1}
(3/17)

\label{sec-2-2}
\item 平成27年度SWLAB新B4ガイダンス
\hfill
\label{enum-lab1}
(4/1)

\label{sec-2-2}
\item 平成27年度新B4歓迎会 
\hfill
\label{enum-lab1}
(4/1)

\label{sec-2-2}
\item 平成27年度新B4向けGit勉強会
\hfill
\label{enum-lab12}
(4/2)

\item 乃村研お花見
\hfill
\label{enum-18}
(4/2)

\item 乃村研ミーティング
\hfill
\label{enum-18}
(4/3)

\end{enumerate}

\subsection{大学院関連}
\label{sec2-3}
\begin{enumerate}
    \item 新M1向けオリエンテーション
    \hfill
    \label{enum-univ2}
    (4/1)

    \item 平成27年度岡山大学入学式,大学院入学式
    \hfill
    \label{enum-univ2}
    (4/8)

    \item プロセッサ工学特論
    \hfill
    \label{enum-univ2}
    (4/9)
    \item システムプログラム特論
    \hfill
    \label{enum-univ2}
    (4/14)
    \item ソフトウェア開発法
    \hfill
    \label{enum-univ2}
    (4/14)
\end{enumerate}

\section{詳細および反省・感想}
\label{sec-3}
\setcounter{subsection}{1}
\subsection{研究室関連}
\label{sec-3-1}
\begin{itemize}
\item[(\ref{enum-lab12})]
    平成27年度新B4向けGit勉強会を行った.
    Gitを使ったことのない人にどのように教えると,わかりやすく伝わるか
    を新M1で話し合った.
    言い回しと,説明の順序を少し変えるだけで相手の理解度が
    大きく変わると感じた.
\end{itemize}

\section{今後の予定}
\label{sec-4}
\subsection{研究関連}
\label{sec-4-1}

\begin{enumerate}
\item 研究テーマに関する項目
\hfill
\begin{enumerate}


\item 参考文献の読解
\hfill
(5月上旬)

\item 使用する共有メモリ領域の検討
\hfill
(4月下旬)

\item NICのデバイスドライバの改変箇所の調査
\hfill
(4月下旬)

\item パケット受信処理の実装
\hfill
(4月下旬)
\end{enumerate}
\item 開発に関する項目
\hfill
\begin{enumerate}

\item 自動ビルドスクリプトの作成
\hfill
(4月下旬)

\end{enumerate}
\item 第273回New打ち合わせ
\hfill
\label{enum-7}
(4/20)
\item 第18回New開発打ち合わせ
\hfill
\label{enum-7}
(4/27)
\end{enumerate}
\subsection{研究室関連}
\label{sec-4-2}

\begin{enumerate}

\item 乃村研ミーティング 
\hfill
\label{enum-18}
(4/20)
\end{enumerate}

\subsection{大学院関連}
\begin{enumerate}

\item システムプログラム特論
\hfill
\label{enum-17}
(4/21)
\item ソフトウェア開発法
\hfill
\label{enum-17}
(4/21)
\item プログラミング方法論
\hfill
\label{enum-17}
(4/22)

\end{enumerate}










\end{document}
