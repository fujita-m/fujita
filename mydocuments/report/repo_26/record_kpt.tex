\documentclass[fleqn, 14pt]{extarticlej}
\usepackage{sty/reportForm}
\usepackage{hyperref}
\tolerance=1000
\subtitle{(2015年05月11日$\sim$2015年05月31日)}
\usepackage{sty/strike}
\usepackage{nutils}
\setcounter{section}{0}
\author{乃村研究室M1\\藤田 将輝}
\date{2015年06月01日}
\title{記録書 No.26}
\begin{document}

\maketitle
\section{実績,詳細,および反省・感想}
\subsection{研究関連}
\label{sec-2-1}
\begin{enumerate}
    \item 研究テーマに関する項目
    \hfill
    \label{enum-research1}
    \begin{enumerate}
        \item 参考文献の読解
        \hfill
        \label{enum-1-A}
        (50%,+0%)
        \item 使用する共有メモリ領域の検討
        \hfill
        \label{enum-1-B}
        (75%,+0%)
        \item NICのデバイスドライバの改変箇所の調査
        \hfill
        \label{enum-1-C}
        (50%,+0%)
        \item パケットの作成
        \hfill
        \label{enum-1-D}
        (70%,+0%)\\
  
    \end{enumerate}
    \item 開発に関する項目
    \hfill
    \label{enum-research2}
    \begin{enumerate}

        \item 自動ビルドスクリプトの作成
        \hfill
        \label{enum-2-A}
        (95%,+0%)
    \end{enumerate}
    \item 第275回New打ち合わせ 
    \hfill
    \label{enum-research3}
    (5/14)
    \item 第276回New打ち合わせ 
    \hfill
    \label{enum-research3}
    (5/26)
    \item 第20回New開発打ち合わせ 
    \hfill
    \label{enum-research3}
    (5/15)

\end{enumerate}
  \subsection{研究室関連}
  \begin{enumerate}
   \item 乃村研ミーティング
         \hfill
         (05/11)
   \item 乃村杯
         \hfill
         (05/11)
   \item 全体ミーティング
         \hfill
         (05/15)
   \item 平成27年度第1回研究室内部屋別対抗ボウリング大会
         \hfill
         (05/15)
  \end{enumerate}

  \subsection{大学院関連}
  \begin{enumerate}
      \item 特になし
  \end{enumerate}

\section{KPT}
  \subsection{GOOD}
  \begin{enumerate}
   \item バスケットボールの県リーグで勝ち続けている
   \item 乃村杯で優勝できた
  \end{enumerate}

  \subsection{BAD}
  \begin{enumerate}
   \item お金がない
   \item 集中力が続かない
  \end{enumerate}

\section{近況報告}
先日,研究室の学生で焼肉を食べに行った.
久々の焼肉であり,食べ放題であったため,非常に楽しかった.
食べ放題ではいつも無理をしてでも多くの肉を食べてしまう.
今回もいつものように無理をして食べ過ぎてしまった.
しかし,辛くはなく,非常に幸せを感じた.
今までに行ったことのない店であったが,これからも
定期的に通いたい.
\end{document}
