\documentclass[fleqn, 14pt]{extarticle}
\usepackage{reportForm}
\usepackage[utf8]{inputenc}
\usepackage[T1]{fontenc}
\usepackage{fixltx2e}
\usepackage{graphicx}
\usepackage{longtable}
\usepackage{float}
\usepackage{wrapfig}
\usepackage[normalem]{ulem}
\usepackage{textcomp}
\usepackage{marvosym}
\usepackage{wasysym}
\usepackage{latexsym}
\usepackage{amssymb}
\usepackage{amstext}
\usepackage{hyperref}
\usepackage{comment}
\tolerance=1000
\subtitle{(2015年03月27日$\sim$2015年04月02日)}
\usepackage{strike}
\setcounter{section}{0}
\author{乃村研究室M1\\藤田 将輝}
\date{2015年04月03日}
\title{記録書 No.21}
\begin{document}

\maketitle
\section{実績,詳細,および反省・感想}
\subsection{研究関連}
\label{sec-2-1}
\begin{enumerate}
    \item 研究テーマに関する項目
    \hfill
    \label{enum-research1}
    \begin{enumerate}
        \item 参考文献の読解
        \hfill
        \label{enum-1-A}
        (50%,+0%)
        \item 使用する共有メモリ領域の検討
        \hfill
        \label{enum-1-B}
        (75%,+0%)
        \item NICのデバイスドライバの改変箇所の調査
        \hfill
        \label{enum-1-C}
        (50%,+0%)
        \item パケットの作成
            \hfill
            \label{enum-1-D}
            (0%,+0%)
    \end{enumerate}
    \item 開発に関する項目
    \hfill
    \label{enum-research2}
    \begin{enumerate}

        \item 自動ビルドスクリプトの作成
        \hfill
        \label{enum-2-A}
        (95%,+0%)
        \item debianでのMintの構築
        \hfill
        \label{enum-2-A}
        (95%,+0%)
    \end{enumerate}
    \item 新B4課題に関する項目
    \hfill
    \label{enum-research3}
    \begin{enumerate}
        \item Mint構築手順作成
        \hfill
        \label{enum-3-A}
        (95%,+0%)
        
    \end{enumerate}
    \item 第271回New打ち合わせ 
    \hfill
    \label{enum-research3}
    (3/30)

    \end{enumerate}
  \subsection{研究室関連}
  \begin{enumerate}
   \item 全体ミーティング
         \hfill
         (03/27)
   \item 新B4向けオリエンテーション
         \hfill
         (04/01)
     \item 新B4歓迎会
         \hfill
         (04/01)
     \item Git勉強会
         \hfill
         (04/02)
     \item 乃村研お花見
         \hfill
         (04/02)\\
         乃村研お花見に参加した.昨年は寒さから体調不良の者が出るほど
         であったが,今年は暖かく,お花見に適した環境であったと感じた.
         OBの方も来てくださり,非常に楽しいお花見であった.
         新B4も徐々に研究室の空気に慣れてきたようで,安心した.

  \end{enumerate}

  \subsection{大学院関連}
  \begin{enumerate}
   \item 新M1向けオリエンテーション
         \hfill
         (04/01)
  \end{enumerate}

\section{KPT}
  \subsection{GOOD}
  \begin{enumerate}
   \item するべきタスクを整理した
   \item 早起きができた
   \item 目覚めてからの準備が早くなった
  \end{enumerate}

  \subsection{BAD}
  \begin{enumerate}
   \item 研究が進められていない
   \item 集中力が続かない
   \item 浪費癖が発覚した
  \end{enumerate}

\section{近況報告}
4月からあまりバイトに入られないことをバイト先の店長に伝えた.
自身はキッチンを担当しているが,他のキッチン担当者はほとんどが
新4年生であり,就職活動をはじめる.
このため,常にキッチンが人出不足になっている.
具体的には,金曜日にキッチンの担当人数が1人という状況である.
なお,普段であれば3人〜4人は入っている.
バイトに入らないと言ったものの,このような状況が続いていて,
非常に気まずい.
\end{document}
