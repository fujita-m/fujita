\documentclass[fleqn, 14pt]{extarticle}
\usepackage{reportForm}
\usepackage[utf8]{inputenc}
\usepackage[T1]{fontenc}
\usepackage{fixltx2e}
\usepackage{graphicx}
\usepackage{longtable}
\usepackage{float}
\usepackage{wrapfig}
\usepackage[normalem]{ulem}
\usepackage{textcomp}
\usepackage{marvosym}
\usepackage{wasysym}
\usepackage{latexsym}
\usepackage{amssymb}
\usepackage{amstext}
\usepackage{hyperref}
\usepackage{comment}
\tolerance=1000
\subtitle{(2015年2月17日$\sim$2015年3月26日)}
\usepackage{strike}
\setcounter{section}{-1}
\author{乃村研究室B4\\藤田 将輝}
\date{2015年3月27日}
\title{記録書 No.20}
\hypersetup{
  pdfkeywords={},
  pdfsubject={},
  pdfcreator={Emacs 24.3.1 (Org mode 8.0.3)}}
\begin{document}
\maketitle
\section{前回ミーティングからの指導・指摘事項}
\label{sec-1}
\begin{enumerate}
\item 特になし
\newline
\hfill

\end{enumerate}




\section{実績}
\label{sec-2}

\subsection{研究関連}
\label{sec-2-1}
\begin{enumerate}
    \item 研究テーマに関する項目
    \hfill
    \label{enum-research1}
    \begin{enumerate}

        \item 参考文献の読解
        \hfill
        \label{enum-1-A}
        (50%,+0%)
        \item 使用する共有メモリ領域の検討
        \hfill
        \label{enum-1-B}
        (75%,+0%)
        \item NICのデバイスドライバの改変箇所の調査
        \hfill
        \label{enum-1-C}
        (50%,+0%)
    \end{enumerate}
    \item 開発に関する項目
    \hfill
    \label{enum-research2}
    \begin{enumerate}

        \item 自動ビルドスクリプトの作成
        \hfill
        \label{enum-2-A}
        (95%,+0%)
        \item debianでのMintの構築
        \hfill
        \label{enum-2-A}
        (95%,+45%)
    \end{enumerate}
    \item 新B4課題に関する項目
    \hfill
    \label{enum-research3}
    \begin{enumerate}
        \item Mint構築手順作成
        \hfill
        \label{enum-3-A}
        (95%,+95%)
        
    \end{enumerate}
    \item 第270回New打ち合わせ 
    \hfill
    \label{enum-research3}
    (3/3)

    \item Newコードレビュー 
    \hfill
    \label{enum-research3}
    (3/24)
    \end{enumerate}

\subsection{研究室関連}
\label{sec-2-2}
\begin{enumerate}
\item 全体ミーティング 
\hfill
\label{enum-lab1}
(2/17)

\label{sec-2-2}
\item 部屋別対抗ボウリング大会 
\hfill
\label{enum-lab1}
(2/17)

\label{sec-2-2}
\item 乃村研円卓会議 
\hfill
\label{enum-lab1}
(2/19)

\label{sec-2-2}
\item nomod 
\hfill
\label{enum-lab1}
(2/23)

\item 研究室配属説明会 
\hfill
\label{enum-18}
(3/2)

\item 乃村杯 
\hfill
\label{enum-18}
(3/23)

\item 乃村研送別会 
\hfill
\label{enum-18}
(3/23)

\item 大掃除 
\hfill
\label{enum-18}
(3/26)
\end{enumerate}

\subsection{大学関連}
\label{sec2-3}
\begin{enumerate}
    \item 平成26年度岡山大学学位記等授与式
    \hfill
    \label{enum-univ2}
    (3/25)

    \item 謝恩会
    \hfill
    \label{enum-univ2}
    (3/25)

\end{enumerate}

\section{詳細および反省・感想}
\label{sec-3}
\setcounter{subsection}{2}
\subsection{大学関連}
\label{sec-3-1}
\begin{itemize}
\item[(\ref{enum-3-A})]
    新B4研修課題を整理している.
    B4が読んで理解できるように,丁寧に書いているつもりだが,
    読みなおしてみると,わかりにくいところが多くあった.
    B4にとって最初の課題になり,さらに自身にとっても最初の
    指導になるため,丁寧に取り組む.
    これからは指導する立場になるという事を意識して行動する.

\end{itemize}

\section{今後の予定}
\label{sec-4}
\subsection{研究関連}
\label{sec-4-1}

\begin{enumerate}
\item 研究テーマに関する項目
\hfill
\begin{enumerate}


\item 参考文献の読解
\hfill
(4月上旬)

\item 使用する共有メモリ領域の検討
\hfill
(4月下旬)

\item NICのデバイスドライバの改変箇所の調査
\hfill
(4月下旬)

\end{enumerate}
\item 開発に関する項目
\hfill
\begin{enumerate}

\item 自動ビルドスクリプトの作成
\hfill
(4月中旬)

\end{enumerate}
\item 第271回New打ち合わせ
\hfill
\label{enum-7}
(3/3)
\end{enumerate}

\subsection{研究室関連}
\label{sec-4-2}

\begin{enumerate}

\item SWLAB新B4ガイダンス
\hfill
\label{enum-18}
(4/1)

\item 新B4歓迎会 
\hfill
\label{enum-18}
(4/1)

\item 乃村研ミーティング 
\hfill
\label{enum-18}
(4/3)


\item 乃村研お花見 
\hfill
\label{enum-18}
(4/3)
\end{enumerate}

\subsection{大学関連}
\begin{enumerate}

\item 平成27年度岡山大学・大学院入学式
\hfill
\label{enum-17}
(4/8)


\end{enumerate}










\end{document}
