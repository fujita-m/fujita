\documentclass[fleqn, 12pt]{extarticle}
\usepackage{reportForm}
\usepackage[utf8]{inputenc}
\usepackage[T1]{fontenc}
\usepackage{fixltx2e}
\usepackage{graphicx}
\usepackage{longtable}
\usepackage{float}
\usepackage{wrapfig}
\usepackage[normalem]{ulem}
\usepackage{textcomp}
\usepackage{marvosym}
\usepackage{setspace}
\usepackage{wasysym}
\usepackage{latexsym}
\usepackage{amssymb}
\usepackage{amstext}
\usepackage{hyperref}
\usepackage{comment}
\tolerance=1000
\subtitle{(2016年02月04日$\sim$2016年03月04日)}
\usepackage{strike}
\setcounter{section}{-1}
\author{乃村研究室M1\\藤田 将輝}
\date{2016年03月07日}
\title{記録書 No.45}
\hypersetup{
    pdfkeywords={},
    pdfsubject={},
    pdfcreator={Emacs 24.3.1 (Org mode 8.0.3)}}
    \begin{document}
    \maketitle

    \section{前回ミーティングからの指導・指摘事項}
    \label{sec-1}
    \begin{enumerate}

        \item 特になし.

    \end{enumerate}

    \section{実績}
    \label{sec-2}

    \subsection{研究関連}
    \label{sec-2-1}
    \begin{enumerate}

        \item 研究テーマに関する項目
            \hfill
            \label{enum-research1}
            \begin{enumerate}

                \item 参考文献の読解
                    \hfill
                    \label{enum-1-A}
                    (50%,+0%)

                \item バグの再現
                    \hfill
                    \label{enum-1-B}
                    (0%,+0%)

                \item 第136回システムソフトウェアとオペレーティング・システム研究会スライド作成
                    \label{enum-1-D}
                    \newline
                    \hfill
                    (100%,+90%)

            \end{enumerate}

        \item 開発に関する項目
            \hfill
            \label{enum-research2}
            \begin{enumerate}

                \item 自動ビルドスクリプトの作成
                    \hfill
                    \label{enum-2-A}
                    (95%,+0%)

            \end{enumerate}

        \item 第295回New打ち合わせ
            \hfill
            \label{enum-research3}
            (02/08)

        \item 第296回New打ち合わせ
            \hfill
            \label{enum-research3}
            (02/22)

    \end{enumerate}
    \subsection{研究室関連}
    \label{sec-2-2}
    \begin{enumerate}

        \item 平成27年度研究室説明会
            \hfill
            \label{enum-18}
            (02/16)

        \item 第136回システムソフトウェアとオペレーティング・システム研究会発表練習
            \hfill
            \label{enum-19}
            (02/19)

        \item 第136回システムソフトウェアとオペレーティング・システム研究会
            \hfill
            \label{enum-20}
            (02/29,03/01)

    \end{enumerate}

    \subsection{大学院関連}
    \label{sec2-3}
    \begin{enumerate}

        \item 特になし.
            \hfill
            \label{enum-univ1}

    \end{enumerate}

    \subsection{就職活動関連}
    \label{sec2-3}
    \begin{enumerate}

        \item アカリクITイベント in 大阪
            \hfill
            \label{enum-univ1}
            (02/13)

        \item NEC OB訪問
            \hfill
            \label{enum-univ1}
            (02/26)

        \item 三菱電機 会社説明会
            \hfill
            \label{enum-univ1}
            (03/02)

        \item NTTデータ OB訪問
            \hfill
            \label{enum-univ1}
            (03/03)

    \end{enumerate}

    \section{詳細および反省・感想}
    \label{sec-3}

    \setcounter{subsection}{1}
    \subsection{研究室関連}
    \label{sec-3-2}

    \begin{itemize}

        \item[(\ref{enum-20})]
            \begin{spacing}{1.2}
            第136回システムソフトウェアとオペレーティング・システム研究会に発表者として参加した.
            自身にとって初めての発表であり,自身の発表開始直前までとても緊張していた.
            自身の発表が始まり,話し始めると緊張が和らぎ,落ち着いて発表できたように感じた.
            質疑では,たくさんの意見を頂き,活発に議論できたと感じた.
            また,発表会場が広く,指示棒が使用できなかったため,レーザーポインタを使用した.
            レーザーポインタの扱いに不慣れだったためか,途中でどこを指しているか分からなくなることがあった.
            このため,今後はレーザーポインタを用いての発表練習も行う.
            研究会に関してご指導していただいた先生方,ならびに先輩方ありがとうございました.
            \end{spacing}

    \end{itemize}

    \section{今後の予定}
    \label{sec-4}

    \subsection{研究関連}
    \label{sec-4-1}

    \begin{enumerate}

        \item 研究テーマに関する項目
            \hfill
            \begin{enumerate}

                \item 参考文献の読解
                    \hfill
                    (3月中旬)

                \item バグの再現
                    \hfill
                    (4月上旬)

            \end{enumerate}

        \item 開発に関する項目
            \hfill
            \begin{enumerate}

                \item 自動ビルドスクリプトの作成
                    \hfill
                    (4月中旬)

            \end{enumerate}

        \item 第297回New打ち合わせ
            \hfill
            \label{enum-7}
            (03/07)

        \item Mintソースコードレビュー
            \hfill
            \label{enum-18}
            (03/08)

    \end{enumerate}

    \subsection{研究室関連}
    \label{sec-4-2}

    \begin{enumerate}

        \item 第29回乃村杯
            \hfill
            \label{enum-18}
            (03/09)

        \item 平成27年度乃村研究室送別会
            \hfill
            \label{enum-18}
            (03/09)

    \end{enumerate}

    \subsection{大学院関連}
    \begin{enumerate}

        \item 特になし
            \hfill
            \label{enum-17}

    \end{enumerate}

    \subsection{就職活動関連}
    \label{sec2-3}
    \begin{enumerate}

        \item 日立製作所 OB訪問
            \hfill
            \label{enum-univ1}
            (03/07)

        \item 三菱電機関連会社 会社説明会
            \hfill
            \label{enum-univ1}
            (03/08)

        \item OB/OGフォーラム
            \hfill
            \label{enum-univ1}
            (03/14 - 03/17)

        \item NEC 会社説明会
            \hfill
            \label{enum-univ1}
            (03/16)

    \end{enumerate}

    \end{document}
