\documentclass[fleqn, 14pt]{extarticle}
\usepackage{reportForm}
\usepackage[utf8]{inputenc}
\usepackage[T1]{fontenc}
\usepackage{fixltx2e}
\usepackage{graphicx}
\usepackage{longtable}
\usepackage{float}
\usepackage{wrapfig}
\usepackage[normalem]{ulem}
\usepackage{textcomp}
\usepackage{marvosym}
\usepackage{wasysym}
\usepackage{latexsym}
\usepackage{amssymb}
\usepackage{amstext}
\usepackage{hyperref}
\usepackage{comment}
\tolerance=1000
\subtitle{(2015年1月19日$\sim$2015年2月5日)}
\usepackage{strike}
\setcounter{section}{-1}
\author{乃村研究室B4\\藤田 将輝}
\date{2015年2月6日}
\title{記録書 No.18}
\hypersetup{
  pdfkeywords={},
  pdfsubject={},
  pdfcreator={Emacs 24.3.1 (Org mode 8.0.3)}}
\begin{document}
\maketitle
\section{前回ミーティングからの指導・指摘事項}
\label{sec-1}
\begin{enumerate}
\item 特になし
\newline
\hfill

\end{enumerate}




\section{実績}
\label{sec-2}

\subsection{研究関連}
\label{sec-2-1}
\begin{enumerate}
    \item 研究テーマに関する項目
    \hfill
    \label{enum-research1}
    \begin{enumerate}

        \item 参考文献の読解
        \hfill
        \label{enum-1-A}
        (50%,+0%)
        \item 使用する共有メモリ領域の検討
        \hfill
        \label{enum-1-B}
        (75%,+0%)
        \item NICのデバイスドライバの改変箇所の調査
        \hfill
        \label{enum-1-C}
        (50%,+0%)
        \item NICドライバの改変
        \hfill
        \label{enum-1-D}
        (100%,+20%)
        \item 特別研究報告書
        \hfill
        \label{enum-1-E}
        (100%,+40%)
        \item 発表スライドの作成 
        \hfill
        \label{enum-1-F}
        (50%,+50%)
    \end{enumerate}
    \item 開発に関する項目
    \hfill
    \label{enum-research2}
    \begin{enumerate}

        \item 自動ビルドスクリプトの作成
        \hfill
        \label{enum-2-A}
        (95%,+0%)
        \item debianでのMintの構築
        \hfill
        \label{enum-2-A}
        (50%,+0%)
    \end{enumerate}
    \item New打ち合わせ 
    \hfill
    \label{enum-research3}
    (1/20,2/5)
    \end{enumerate}

\subsection{研究室関連}
\label{sec-2-2}
\begin{enumerate}
\item 全体ミーティング 
\hfill
\label{enum-lab1}
(1/19)
\end{enumerate}


\section{詳細および反省・感想}
\label{sec-3}
%\setcounter{subsection}{1}
\subsection{研究関連}
\label{sec-3-1}

\begin{itemize}
\item[(\ref{enum-1-E})]
    特別研究報告書を作成し,提出した.
    特別研究報告書作成にあたって,ご指導してくださった乃村先生,
    先輩の皆様,本当にありがとうございました.
    今は,スライドの作成をしている.
    自身の研究を理解してもらえるように,話の流れを意識して
    作成する.
    
\end{itemize}

\section{今後の予定}
\label{sec-4}
\subsection{研究関連}
\label{sec-4-1}

\begin{enumerate}
\item 研究テーマに関する項目
\hfill
\begin{enumerate}


\item 参考文献の読解
\hfill
(2月下旬)

\item 使用する共有メモリ領域の検討
\hfill
(2月中旬)

\item NICのデバイスドライバの改変箇所の調査
\hfill
(2月中旬)

\item 発表スライドの作成 
\hfill
(2/10)

\end{enumerate}
\item 開発に関する項目
\hfill
\begin{enumerate}

\item 自動ビルドスクリプトの作成
\hfill
(2月中旬)

\item debianでのMintの構築
\hfill
(2月中旬)

\end{enumerate}
\item 第270回New打ち合わせ
\hfill
\label{enum-7}
(3/3)
\item 第17回Newグループ開発打ち合わせ
\hfill
\label{enum-8}
(2/16)
\end{enumerate}

\subsection{研究室関連}
\label{sec-4-2}

\begin{enumerate}


\item 乃村研ミーティング 
\hfill
\label{enum-16}
(2/6)

\item 全体ミーティング 
\hfill
\label{enum-18}
(2/17)

\item 研究室配属説明会 
\hfill
\label{enum-18}
(3/2)
\end{enumerate}

\subsection{大学関連}
\begin{enumerate}
\item 特別研究報告書締切 
\hfill
\label{enum-17}
(2/6)

\item 平成26年度特別研究報告会 
\hfill
\label{enum-17}
(2/13)


\end{enumerate}










\section{その他}
先日,約3年間続けたアルバイトである個別指導塾の講師を2月末で辞めることを
責任者に伝えた.
3年間継続して教えていた生徒は,出会った頃は中学3年生だったが,今は
高校3年生である.
その生徒は進学先が決まり,下宿先も決めたと言っていた.
さらに,その生徒は卒業まで引越しのアルバイトをするとも言っていた.
立派に成長していて誇らしい気持ちになった.
同時に,もう生徒と関わることがなくなるため,寂しい気持ちにもなった.
子を持ったことはないが,親心のようなものを感じた.
また,人の成長に携われたことは大きな財産であると感じた.
私が塾を辞めるまで,残された授業に精一杯取り組む.
\end{document}
