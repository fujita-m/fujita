\documentclass[fleqn, 14pt]{extarticle}
\usepackage{reportForm}
\usepackage[utf8]{inputenc}
\usepackage[T1]{fontenc}
\usepackage{fixltx2e}
\usepackage{graphicx}
\usepackage{longtable}
\usepackage{float}
\usepackage{wrapfig}
\usepackage[normalem]{ulem}
\usepackage{textcomp}
\usepackage{marvosym}
\usepackage{wasysym}
\usepackage{latexsym}
\usepackage{amssymb}
\usepackage{amstext}
\usepackage{hyperref}
\usepackage{comment}
\tolerance=1000
\subtitle{(2014年11月4日$\sim$2014年11月12日)}
\usepackage{strike}
\setcounter{section}{-1}
\author{乃村研究室B4\\藤田 将輝}
\date{2014年11月13日}
\title{記録書 No.13}
\hypersetup{
  pdfkeywords={},
  pdfsubject={},
  pdfcreator={Emacs 24.3.1 (Org mode 8.0.3)}}
\begin{document}

\maketitle




\section{前回ミーティングからの指導・指摘事項}
\label{sec-1}
\begin{enumerate}
\item 特になし
\newline
\hfill

\end{enumerate}




\section{実績}
\label{sec-2}

\subsection{研究関連}
\label{sec-2-1}
\begin{enumerate}
\item 研究テーマに関する項目
\hfill
\label{enum-research1}
\begin{enumerate}

\item 参考文献の読解
\hfill
\label{enum-1-A}
(50%,+0%)
\item 使用する共有メモリ領域の検討
\hfill
\label{enum-1-B}
(55%,+0%)
\item NICのデバイスドライバの改変箇所の調査
\hfill
\label{enum-1-C}
(30%,+0%)
\item NICドライバの改変
\hfill
\label{enum-1-D}
(10%,+0%)
\item 中間発表スライドの作成
\hfill
\label{enum-1-E}
(80%,+80%)
\end{enumerate}
\item 開発に関する項目
\hfill
\label{enum-research2}
\begin{enumerate}

\item 自動ビルドスクリプトの作成
\hfill
\label{enum-2-A}
(95%,+0%)
\item debianでのMintの構築
\hfill
\label{enum-2-A}
(0%,+0%)
\end{enumerate}

\item 第14回New開発打ち合わせ
\hfill
\label{enum-3}
(11/5)
\item 第264回New打ち合わせ
\hfill
\label{enum-4}
(11/10)



\end{enumerate}

\subsection{研究室関連}
\label{sec-2-2}
\begin{enumerate}
\item 中間発表発表練習
\hfill
\label{enum-lab1}
(11/10)
\end{enumerate}





\section{詳細および反省・感想}
\label{sec-3}
\subsection{研究関連}
\label{sec-3-1}

\begin{itemize}
\item[(\ref{enum-1-E})]
特別研究中間報告のためのスライドを作成をしている.
自分の研究をスライド3枚にまとめて3分以内で説明することに難しさを感じている.
分かりやすい図を描くことにも苦労したが,先輩方の指導のおかげで
見やすくわかりやすい図を作成することができた.
また,発表練習で説明の仕方やスライドの細かい体裁もご指導いただき,
完成に近づいた.
現在は3分を少し過ぎてしまっているため,今後は冗長な表現を削減し,
3分以内に収める.
\end{itemize}


\section{今後の予定}
\label{sec-4}
\subsection{研究関連}
\label{sec-4-1}

\begin{enumerate}
\item 研究テーマに関する項目
\hfill
\begin{enumerate}


\item 参考文献の読解
\hfill
(11/20)

\item 使用する共有メモリ領域の検討
\hfill
(11/22)

\item NICのデバイスドライバの改変箇所の調査
\hfill
(11/20)

\item NICドライバの改変
\hfill
(11/30)

\end{enumerate}
\item 開発に関する項目
\hfill
\begin{enumerate}

\item 自動ビルドスクリプトの作成
\hfill
(11/30)

\item debianでのMintの構築
\hfill
(11/30)

\end{enumerate}
\item 第265回New打ち合わせ
\hfill
\label{enum-7}
(11/25)
\item 第15回Newグループ開発打ち合わせ
\hfill
\label{enum-8}
(12/1)
\end{enumerate}

\subsection{研究室関連}
\label{sec-4-2}

\begin{enumerate}


\item 特別研究中間報告 発表練習
\hfill
\label{enum-9}
(11/18)

\item 特別研究中間報告
\hfill
\label{enum-10}
(11/21)

\item 平成26年度M2研究進捗報告会
\hfill
\label{enum-11}
(11/21)


\item 乃村研ミーティング
\hfill
\label{enum-11}
(11/23)

\item 乃村研忘年会
\hfill
\label{enum-12}
(11/23)

\item SWLAB忘年会
\hfill
\label{enum-13}
(12/5)
\end{enumerate}

\subsection{大学関連}
\begin{enumerate}
\item 防災訓練
\hfill
\label{enum-13}
(11/20)
\end{enumerate}











\end{document}
