\documentclass[fleqn, 14pt]{extarticlej}
\usepackage{sty/reportForm}
\usepackage{hyperref}
\tolerance=1000
\subtitle{(2015年04月20日$\sim$2015年05月10日)}
\usepackage{sty/strike}
\usepackage{nutils}
\setcounter{section}{0}
\author{乃村研究室M1\\藤田 将輝}
\date{2015年05月11日}
\title{記録書 No.24}
\begin{document}

\maketitle
\section{実績,詳細,および反省・感想}
\subsection{研究関連}
\label{sec-2-1}
\begin{enumerate}
    \item 研究テーマに関する項目
    \hfill
    \label{enum-research1}
    \begin{enumerate}
        \item 参考文献の読解
        \hfill
        \label{enum-1-A}
        (50%,+0%)
        \item 使用する共有メモリ領域の検討
        \hfill
        \label{enum-1-B}
        (75%,+0%)
        \item NICのデバイスドライバの改変箇所の調査
        \hfill
        \label{enum-1-C}
        (50%,+0%)
        \item パケットの作成
        \hfill
        \label{enum-1-D}
        (70%,+30%)\\
        正常な処理をするパケットをキャプチャし,その内容を複写した.
        さらに,複写したパケットを共有メモリに配置し,IPIにより
        NICドライバの割り込みハンドラを動作させ,パケットの受信処理を
        再現した.
        今後はどの程度の間隔ならば連続で受信処理を再現できるかを調査する.
        また,パケットのサイズを任意に指定可能にすることで,
        柔軟な通信量の変化を可能にする.
  
    \end{enumerate}
    \item 開発に関する項目
    \hfill
    \label{enum-research2}
    \begin{enumerate}

        \item 自動ビルドスクリプトの作成
        \hfill
        \label{enum-2-A}
        (95%,+0%)
    \end{enumerate}
    \item 第273回New打ち合わせ 
    \hfill
    \label{enum-research3}
    (4/20)
    \item 第274回New打ち合わせ 
    \hfill
    \label{enum-research3}
    (5/07)
    \item 第19回New開発打ち合わせ 
    \hfill
    \label{enum-research3}
    (4/27)

\end{enumerate}
  \subsection{研究室関連}
  \begin{enumerate}
   \item 乃村研ミーティング
         \hfill
         (04/20)
  \end{enumerate}

  \subsection{大学院関連}
  \begin{enumerate}
    \item プロセッサ工学特論
    \hfill
    \label{enum-univ2}
    (4/23,4/30)
    \item システムプログラム特論
    \hfill
    \label{enum-univ2}
    (4/21)
    \item ソフトウェア開発法
    \hfill
    \label{enum-univ2}
    (4/21,28)
    \item プログラミング方法論
    \hfill
    \label{enum-univ2}
    (5/08)
    \item ヒューマンコンピュータインタラクション
    \hfill
    \label{enum-univ2}
    (4/24)


  \end{enumerate}

\section{KPT}
  \subsection{GOOD}
  \begin{enumerate}
   \item 今後の計画を立てられた
  \end{enumerate}

  \subsection{BAD}
  \begin{enumerate}
   \item 計画どおり作業が進まない.
   \item 浪費癖が一向に治らない.
  \end{enumerate}

\section{近況報告}
5月10日(日)にバスケットボールの県リーグの初戦が行われ,これに参加した.
新チームになって初の試合であったため,チーム全体に非常に気合が入っていた.
そのおかげか,全員がよく動くことができ,結果として約30点差をあけて勝つこと
ができた.自身は新チームのキャプテンである.このため,
チームのモチベーションを高く維持できるよう,チーム全体で試合を意識して
練習に取り組む.
\end{document}
