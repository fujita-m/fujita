\documentclass[fleqn, 14pt]{extarticle}
\usepackage{reportForm}
\usepackage[utf8]{inputenc}
\usepackage[T1]{fontenc}
\usepackage{fixltx2e}
\usepackage{graphicx}
\usepackage{longtable}
\usepackage{float}
\usepackage{wrapfig}
\usepackage[normalem]{ulem}
\usepackage{textcomp}
\usepackage{marvosym}
\usepackage{wasysym}
\usepackage{latexsym}
\usepackage{amssymb}
\usepackage{amstext}
\usepackage{hyperref}
\usepackage{comment}
\tolerance=1000
\subtitle{(2015年8月25日$\sim$2015年9月24日)}
\usepackage{strike}
\setcounter{section}{-1}
\author{乃村研究室M1\\藤田 将輝}
\date{2015年9月25日}
\title{記録書 No.35}
\hypersetup{
    pdfkeywords={},
    pdfsubject={},
    pdfcreator={Emacs 24.3.1 (Org mode 8.0.3)}}
    \begin{document}
    \maketitle

    \section{前回ミーティングからの指導・指摘事項}
    \label{sec-1}
    \begin{enumerate}
        \item イベントを途中退席する際は,周囲に報告する.
            \hfill
            [9/14, 106,乃村先生]
            \newline

    \end{enumerate}

    \section{実績}
    \label{sec-2}

    \subsection{研究関連}
    \label{sec-2-1}
    \begin{enumerate}

        \item 研究テーマに関する項目
            \hfill
            \label{enum-research1}
            \begin{enumerate}

                \item 参考文献の読解
                    \hfill
                    \label{enum-1-A}
                    (50%,+0%)

                \item 正常に処理できる通信量の測定
                    \hfill
                    \label{enum-1-B}
                    (100%,+70%)

                \item バグの再現
                    \hfill
                    \label{enum-1-C}
                    (0%,+0%)

            \end{enumerate}

        \item 開発に関する項目
            \hfill
            \label{enum-research2}
            \begin{enumerate}

                \item 自動ビルドスクリプトの作成
                    \hfill
                    \label{enum-2-A}
                    (95%,+0%)

            \end{enumerate}

        \item 第283回New打ち合わせ
            \hfill
            \label{enum-research3}
            (8/25)

        \item 第284回New打ち合わせ
            \hfill
            \label{enum-research3}
            (9/8)

        \item 第25回New開発打ち合わせ 
            \hfill
            \label{enum-research3}
            (9/17)

        \item 第285回New打ち合わせ
            \hfill
            \label{enum-research3}
            (9/18)

    \end{enumerate}

    \subsection{研究室関連}
    \label{sec-2-2}
    \begin{enumerate}

        \item 全体ミーティング
            \hfill
            \label{enum-lab1}
            (8/25)

        \item 乃村研ミーティング
            \hfill
            \label{enum-lab1}
            (8/31)

        \item 2015年度香川大学岡山大学合同研究会
            \hfill
            \label{enum-lab1}
            (9/7)

        \item 乃村研ミーティング
            \hfill
            \label{enum-lab1}
            (9/18)

    \end{enumerate}

    \subsection{大学院関連}
    \label{sec2-3}
    \begin{enumerate}

        \item 特になし
            \hfill
            \label{enum-univ1}

    \end{enumerate}

    \section{詳細および反省・感想}
    \label{sec-3}

    %\setcounter{subsection}{1}
    \subsection{研究関連}
    \label{sec-3-2}

    \begin{itemize}

        \item[(\ref{enum-1-B})]
            本デバッグ支援環境の動作を確認した.具体的にはデバッグ支援OSか
            らデバッグ対象OSへパケットを送信した際のスループットについて測
            定することで,実際のNICの性能を再現することに十分な性能を実現できてい
            るかを調査した.測定結果は,NICドライバで処理できる最大のスル
            ープットが10Gbpsであり,デバッグ対象OS上で動作するUDPの受信プ
            ログラムが処理できる最大のスループットが7.7Gbpsであった.実際
            のNICは2.5Gbps相当であるため,十分にNICを再現できる性能を持っ
            ていることを示せた.

    \end{itemize}

    \section{今後の予定}
    \label{sec-4}

    \subsection{研究関連}
    \label{sec-4-1}

    \begin{enumerate}

        \item 研究テーマに関する項目
            \hfill
            \begin{enumerate}

                \item 参考文献の読解
                    \hfill
                    (10月中旬)

                \item バグの再現
                    \hfill
                    (10月上旬)

            \end{enumerate}

        \item 開発に関する項目
            \hfill
            \begin{enumerate}

                \item 自動ビルドスクリプトの作成
                    \hfill
                    (10月中旬)

            \end{enumerate}

        \item 第286回New打ち合わせ
            \hfill
            \label{enum-7}
            (10/2)

        \item 第26回New開発打ち合わせ
            \hfill
            \label{enum-7}
            (10/8)

    \end{enumerate}

    \subsection{研究室関連}
    \label{sec-4-2}

    \begin{enumerate}

        \item 全体ミーティング
            \hfill
            \label{enum-18}
            (9/25)

        \item 乃村研究室研修会
            \hfill
            \label{enum-18}
            (9/29,30)

        \item 乃村研ミーティング
            \hfill
            \label{enum-18}
            (10/9)

    \end{enumerate}

    \subsection{大学院関連}
    \begin{enumerate}

        \item 特になし
            \hfill
            \label{enum-17}

    \end{enumerate}

    \subsection{学会情報} 
    \begin{enumerate}
        \item 第135回システムソフトウェアとオペレーティング・システム研究会\\
            開催日時:平成27年11月24日(火)\\
            開催場所:お茶の水女子大学
            申込締切:平成27年10月16日(金)\\
            原稿締め切り:平成27年10月30日(金)\\
    \end{enumerate}

    \end{document}
