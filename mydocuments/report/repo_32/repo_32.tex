\documentclass[fleqn, 14pt]{extarticle}
\usepackage{reportForm}
\usepackage[utf8]{inputenc}
\usepackage[T1]{fontenc}
\usepackage{fixltx2e}
\usepackage{graphicx}
\usepackage{longtable}
\usepackage{float}
\usepackage{wrapfig}
\usepackage[normalem]{ulem}
\usepackage{textcomp}
\usepackage{marvosym}
\usepackage{wasysym}
\usepackage{latexsym}
\usepackage{amssymb}
\usepackage{amstext}
\usepackage{hyperref}
\usepackage{comment}
\tolerance=1000
\subtitle{(2015年8月3日$\sim$2015年8月24日)}
\usepackage{strike}
\setcounter{section}{-1}
\author{乃村研究室M1\\藤田 将輝}
\date{2015年8月25日}
\title{記録書 No.32}
\hypersetup{
    pdfkeywords={},
    pdfsubject={},
    pdfcreator={Emacs 24.3.1 (Org mode 8.0.3)}}
    \begin{document}
    \maketitle

    \section{前回ミーティングからの指導・指摘事項}
    \label{sec-1}
    \begin{enumerate}
        \item 特になし
            \newline
            \hfill

    \end{enumerate}

    \section{実績}
    \label{sec-2}

    \subsection{研究関連}
    \label{sec-2-1}
    \begin{enumerate}

        \item 研究テーマに関する項目
            \hfill
            \label{enum-research1}
            \begin{enumerate}

                \item 参考文献の読解
                    \hfill
                    \label{enum-1-A}
                    (50%,+0%)

                \item 使用する共有メモリ領域の検討
                    \hfill
                    \label{enum-1-B}
                    (100%,+25%)

                \item NICのデバイスドライバの改変箇所の調査
                    \hfill
                    \label{enum-1-C}
                    (100%,+50%)

                \item 正常に処理できる通信量の測定
                    \hfill
                    \label{enum-1-D}
                    (30%,+30%)

            \end{enumerate}

        \item 開発に関する項目
            \hfill
            \label{enum-research2}
            \begin{enumerate}

                \item 自動ビルドスクリプトの作成
                    \hfill
                    \label{enum-2-A}
                    (95%,+0%)

            \end{enumerate}

        \item 第282回New打ち合わせ
            \hfill
            \label{enum-research3}
            (8/3)

        \item 第24回New開発打ち合わせ 
            \hfill
            \label{enum-research3}
            (8/20)

    \end{enumerate}

    \subsection{研究室関連}
    \label{sec-2-2}
    \begin{enumerate}

        \item 全体ミーティング
            \hfill
            \label{enum-lab1}
            (8/3)

    \end{enumerate}

    \subsection{大学院関連}
    \label{sec2-3}
    \begin{enumerate}

        \item オープンキャンパス
            \hfill
            \label{enum-univ1}
            (8/7,8)

        \item 平成28年度岡山大学大学院自然科学研究科入学試験
            \hfill
            \label{enum-univ2}
            (8/20)
        
    \end{enumerate}

    \section{詳細および反省・感想}
    \label{sec-3}

    \setcounter{subsection}{1}
    \subsection{研究室関連}
    \label{sec-3-2}

    \begin{itemize}

        \item[(\ref{enum-univ1})]
            オープンキャンパスに参加し,高校生に向けて研究室と研究内容の説明をした.
            スライド作成や発表を考える際,高校生に向けてわかりやすく研究内容を説明することの難しさを感じた.
            また,実際に発表してみると,頷きながら聞いてくれている人が多く,少しでも理解してくれていると感じ,安心した.
            つまらなそうな顔をしている人もいたため,参加者に問いかけたり,質問したりする
            工夫も必要だと感じた.
            今後,高校生に発表する機会があれば,面白いと感じてくれるような発表を考える.

    \end{itemize}

    \section{今後の予定}
    \label{sec-4}

    \subsection{研究関連}
    \label{sec-4-1}

    \begin{enumerate}

        \item 研究テーマに関する項目
            \hfill
            \begin{enumerate}

                \item 参考文献の読解
                    \hfill
                    (9月中旬)

                \item 正常に処理できる通信量の測定
                    \hfill
                    (9月上旬)

            \end{enumerate}

        \item 開発に関する項目
            \hfill
            \begin{enumerate}

                \item 自動ビルドスクリプトの作成
                    \hfill
                    (9月中旬)

            \end{enumerate}

        \item 第283回New打ち合わせ
            \hfill
            \label{enum-7}
            (8/25)

    \end{enumerate}

    \subsection{研究室関連}
    \label{sec-4-2}

    \begin{enumerate}

        \item 全体ミーティング
            \hfill
            \label{enum-18}
            (8/25)

        \item 2015年度研修会
            \hfill
            \label{enum-18}
            (8/27,28)

        \item 乃村研ミーティング 
            \hfill
            \label{enum-18}
            (8/31)

        \item 平成27年度香川大学岡山大学合同研究会
            \hfill
            \label{enum-18}
            (9/7)
    \end{enumerate}

    \subsection{大学院関連}
    \begin{enumerate}

        \item 平成28年度岡山大学大学院自然科学研究科入学試験合格者発表
            \hfill
            \label{enum-17}
            (9/4)

    \end{enumerate}

    \end{document}
