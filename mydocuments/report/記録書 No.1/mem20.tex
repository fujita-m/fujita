% Created 2014-04-04 金 09:39
\documentclass[fleqn, 14pt]{extarticle}
                    \usepackage{reportForm}
\usepackage[utf8]{inputenc}
\usepackage[T1]{fontenc}
\usepackage{fixltx2e}
\usepackage{graphicx}
\usepackage{longtable}
\usepackage{float}
\usepackage{wrapfig}
\usepackage[normalem]{ulem}
\usepackage{textcomp}
\usepackage{marvosym}
\usepackage{wasysym}
\usepackage{latexsym}
\usepackage{amssymb}
\usepackage{amstext}
\usepackage{hyperref}
\tolerance=1000
\subtitle{(2014年02月24日$\sim$2014年04月03日)}
\usepackage{strike}
\setcounter{section}{-1}
\author{乃村研究室M1\\岡田 卓也}
\date{2014年04月04日}
\title{記録書 No.20}
\hypersetup{
  pdfkeywords={},
  pdfsubject={},
  pdfcreator={Emacs 24.3.1 (Org mode 8.0.3)}}
\begin{document}

\maketitle
\section{前回ミーティングからの指導・指摘事項}
\label{sec-1}
\begin{enumerate}
\item 比較をするときは,比較観点がわかるように表を書く.
\newline
\hfill
[2/28, 第145回GN検討打合せ,乃村先生]
\item 他人に対してのご指導やご指摘を自身への事としてとらえる.
\newline
\hfill
[4/2, メール,乃村先生]
\end{enumerate}
\section{実績}
\label{sec-2}
\subsection{研究関連}
\label{sec-2-1}
\begin{enumerate}
\item 研究テーマに関する項目
\hfill
\label{enum-research1}
\begin{enumerate}
\item Mac対応のDTBの試作
\hfill
\label{enum-1-A}
(50%,+α%)
\item Web閲覧履歴取得方法の調査
\hfill
\label{enum-1-B}
(95%,+15%)
\end{enumerate}
\item LastNoteの開発に関する項目
\hfill
\label{enum-research2}
\begin{enumerate}
\item MACアドレスが表示されない問題への対処
\hfill
\label{enum-2-A}
(100%,+100%)
\item 文献にDOIを追加
\hfill
\label{enum-2-B}
(100%,+100%)
\item 文献ページのリンクなしHTMLの不具合への対処
\hfill
\label{enum-2-C}
(100%,+100%)
\end{enumerate}
\end{enumerate}

\subsection{研究室関連}
\label{sec-2-2}
\begin{enumerate}
\item 部屋別対抗ボウリング
\hfill
\label{enum-laboratory1}
(02/24)
\item 全体ミーティング
\hfill
\label{enum-laboratory2}
(02/24)
\item 平成25年度 追い出し会
\hfill
\label{enum-laboratory3}
(02/24)
\item 第98回計算機幹事打ち合わせ
\hfill
\label{enum-laboratory4}
(02/26)
\item 第145回GN検討打合せ
\hfill
\label{enum-laboratory5}
(02/28)
\item 計画停電
\hfill
\label{enum-laboratory6}
(03/02)
\item 第23回乃村杯
\hfill
\label{enum-laboratory7}
(03/03)
\item 第85回GN談話会
\hfill
\label{enum-laboratory8}
(03/04)
\item 第146回GN検討打合せ
\hfill
\label{enum-laboratory9}
(03/13)
\item もくもく会
\hfill
\label{enum-laboratory10}
(03/15)
\item 第99回計算機幹事打ち合わせ
\hfill
\label{enum-laboratory11}
(03/18)
\item 第86回GN談話会
\hfill
\label{enum-laboratory12}
(03/18)
\item nomod \#7
\hfill
\label{enum-laboratory13}
(03/24)
\item 大掃除
\hfill
\label{enum-laboratory14}
(03/27)
\item 第147回GN検討打合せ
\hfill
\label{enum-laboratory15}
(03/28)
\item 平成26年度新B4向けGit勉強会
\hfill
\label{enum-laboratory16}
(04/03)
\end{enumerate}

\subsection{大学・大学院関連}
\label{sec-2-3}
\begin{enumerate}
\item 岡山大学 前期入試
\hfill
\label{enum-university1}
(02/25)
\item 計画停電
\hfill
\label{enum-university2}
(03/02)
\item 岡山大学 後期入試
\hfill
\label{enum-university3}
(03/12)
\item 平成25年度 情報系学科 謝恩会
\hfill
\label{enum-university4}
(03/25)
\item 平成25年度 春期岡山大学学位授与式
\hfill
\label{enum-university5}
(03/25)
\end{enumerate}

\section{詳細および反省・感想}
\label{sec-3}
\subsection{研究室関連}
\label{sec-3-1}
\begin{itemize}
\item[(\ref{enum-laboratory7})]
第23回乃村杯に参加した.今回の競技はビーズアクセサリ作りであった.
私は,白い花のモチーフがついた指輪を作成した.
なかなか綺麗に作成できたと感じたが,順位は最下位であった.
このため,次回の乃村杯の幹事は私である.
皆さんが楽しんで腕を競い合えるような乃村杯にする.
\end{itemize}

\section{今後の予定}
\label{sec-4}
\subsection{研究関連}
\label{sec-4-1}
\begin{enumerate}
\item 研究テーマに関する項目
\hfill
\begin{enumerate}
\item Mac対応のDTBの試作
\hfill
(04/31)
\item Web閲覧履歴取得方法の調査
\hfill
(04/13)
\end{enumerate}
\end{enumerate}

\subsection{研究室関連}
\label{sec-4-2}
\begin{enumerate}
\item 第87回GN談話会
\hfill
\label{enum-3}
(04/04)
\item 乃村研お花見
\hfill
\label{enum-1}
(04/04)
\item 第148回GN検討打合せ
\hfill
\label{enum-4}
(04/11)
\end{enumerate}
% Emacs 24.3.1 (Org mode 8.0.3)
\end{document}