\documentclass[fleqn, 14pt]{extarticle}
\usepackage{reportForm}
\usepackage[utf8]{inputenc}
\usepackage[T1]{fontenc}
\usepackage{fixltx2e}
\usepackage{graphicx}
\usepackage{longtable}
\usepackage{float}
\usepackage{wrapfig}
\usepackage[normalem]{ulem}
\usepackage{textcomp}
\usepackage{marvosym}
\usepackage{wasysym}
\usepackage{latexsym}
\usepackage{amssymb}
\usepackage{amstext}
\usepackage{hyperref}
\usepackage{comment}
\tolerance=1000
\subtitle{(2015年6月12日$\sim$2015年7月9日)}
\usepackage{strike}
\setcounter{section}{-1}
\author{乃村研究室M1\\藤田 将輝}
\date{2015年7月10日}
\title{記録書 No.29}
\hypersetup{
  pdfkeywords={},
  pdfsubject={},
  pdfcreator={Emacs 24.3.1 (Org mode 8.0.3)}}
\begin{document}
\maketitle
\section{前回ミーティングからの指導・指摘事項}
\label{sec-1}
\begin{enumerate}
\item 特になし
\newline
\hfill

\end{enumerate}

\section{実績}
\label{sec-2}

\subsection{研究関連}
\label{sec-2-1}
\begin{enumerate}
    \item 研究テーマに関する項目
    \hfill
    \label{enum-research1}
    \begin{enumerate}

        \item 参考文献の読解
        \hfill
        \label{enum-1-A}
        (50%,+0%)
        \item 使用する共有メモリ領域の検討
        \hfill
        \label{enum-1-B}
        (75%,+0%)
        \item NICのデバイスドライバの改変箇所の調査
        \hfill
        \label{enum-1-C}
        (50%,+0%)
        \item パケット受信処理の実装
        \hfill
        \label{enum-1-D}
        (99%,+9%)

    \end{enumerate}
    \item 開発に関する項目
    \hfill
    \label{enum-research2}
    \begin{enumerate}

        \item 自動ビルドスクリプトの作成
        \hfill
        \label{enum-2-A}
        (95%,+0%)
    \end{enumerate}

    \item 第22回New開発打ち合わせ
    \hfill
    \label{enum-research3}
    (6/15)
    \item 第278回New打ち合わせ 
    \hfill
    \label{enum-research3}
    (6/19)
    \item 第23回New開発打ち合わせ 
    \hfill
    \label{enum-research3}
    (7/2)
    \item 第279回New開発打ち合わせ
    \hfill
    \label{enum-research3}
    (7/3)
    \end{enumerate}

\subsection{研究室関連}
\label{sec-2-2}
\begin{enumerate}

\item 乃村研ミーティング
\hfill
\label{enum-lab1}
(6/25)


\item M2論文紹介
\hfill
\label{enum-lab2}
(6/26)

\item 高校生訪問
\hfill
\label{enum-lab2}
(7/2)

\item 暑気払い 
\hfill
\label{enum-lab2}
(7/6)
\end{enumerate}

\subsection{大学院関連}
\label{sec2-3}
\begin{enumerate}

    \item 特になし
    \hfill
    \label{enum-univ2}

\end{enumerate}

\section{詳細および反省・感想}
\label{sec-3}
%\setcounter{subsection}{1}
\subsection{研究関連}
\label{sec-3-2}
\begin{itemize}
\item[(\ref{enum-1-D})]
    Mintを用いたデバッグ支援環境の作成をしている.
    デバッグ支援環境に実装予定の複数の機能の内,パケットを作成する機能を作成している.
    この機能に関して,パケットを作成するライブラリを調査し,これを使用することで,
    パケットを作成できることを確認した.
    作成したパケットが本デバッグ支援環境において正常に処理されるかどうかはまだ確認できていない.
    このため,今後は作成したパケットが正常に処理されるかを確認する.
    その後,本デバッグ支援環境でどの程度の間隔ならば連続でパケットを処理できるかを測定する.

\end{itemize}

\section{今後の予定}
\label{sec-4}
\subsection{研究関連}
\label{sec-4-1}

\begin{enumerate}

\item 研究テーマに関する項目
\hfill
\begin{enumerate}


\item 参考文献の読解
\hfill
(7月下旬)

\item 使用する共有メモリ領域の検討
\hfill
(7月下旬)

\item NICのデバイスドライバの改変箇所の調査
\hfill
(7月中旬)

\item パケット受信処理の実装
\hfill
(7月中旬)
\end{enumerate}
\item 開発に関する項目
\hfill
\begin{enumerate}

\item 自動ビルドスクリプトの作成
\hfill
(7月中旬)

\end{enumerate}

\item 第280回New打ち合わせ
\hfill
\label{enum-7}
(7/16)

\item 第24回New開発打ち合わせ
\hfill
\label{enum-7}
(7/23)

\end{enumerate}
\subsection{研究室関連}
\label{sec-4-2}

\begin{enumerate}

\item 全体ミーティング
\hfill
\label{enum-18}
(7/10)

\item 乃村研ミーティング 
\hfill
\label{enum-18}
(7/23)

\end{enumerate}

\subsection{大学院関連}
\begin{enumerate}

\item 特になし
\hfill
\label{enum-17}


\end{enumerate}

\end{document}
