\documentclass[fleqn, 12pt]{extarticlej}
\usepackage{sty/reportForm}
\usepackage{hyperref}
\tolerance=1000
\subtitle{(2015年04月3日$\sim$2015年04月19日)}
\usepackage{sty/strike}
\usepackage{nutils}
\setcounter{section}{0}
\author{乃村研究室M1\\藤田 将輝}
\date{2015年04月20日}
\title{記録書 No.23}
\begin{document}

\maketitle
\section{実績,詳細,および反省・感想}
\subsection{研究関連}
\label{sec-2-1}
\begin{enumerate}
    \item 研究テーマに関する項目
    \hfill
    \label{enum-research1}
    \begin{enumerate}
        \item 参考文献の読解
        \hfill
        \label{enum-1-A}
        (50%,+0%)
        \item 使用する共有メモリ領域の検討
        \hfill
        \label{enum-1-B}
        (75%,+0%)
        \item NICのデバイスドライバの改変箇所の調査
        \hfill
        \label{enum-1-C}
        (50%,+0%)
        \item パケットの作成
        \hfill
        \label{enum-1-D}
        (40%,+20%)\\
        NICのパケット受信処理を擬似するため,処理をさせる
        パケットを作成している.
        現在は,パケットの構成を調査し,UDPパケットを構成
        している.
        Etherヘッダに宛先のMACアドレスを直接指定し,
        NICドライバで処理させているが,NICドライバで
        指定したMACアドレスを確認できていない.
        バイトオーダが関係しているのではないかと考えている.
        今後はまず,正常なパケットをキャプチャし,このパケットを
        任意のタイミングで処理させることを目標とする.

    \end{enumerate}
    \item 開発に関する項目
    \hfill
    \label{enum-research2}
    \begin{enumerate}

        \item 自動ビルドスクリプトの作成
        \hfill
        \label{enum-2-A}
        (95%,+0%)
    \end{enumerate}
    \item 第272回New打ち合わせ 
    \hfill
    \label{enum-research3}
    (4/6)
    \item 第18回New開発打ち合わせ 
    \hfill
    \label{enum-research3}
    (4/13)

\end{enumerate}
  \subsection{研究室関連}
  \begin{enumerate}
   \item 全体ミーティング
         \hfill
         (04/16)
  \end{enumerate}

  \subsection{大学院関連}
  \begin{enumerate}
    \item 平成27年度岡山大学入学式,大学院入学式
    \hfill
    \label{enum-univ2}
    (4/8)

    \item プロセッサ工学特論
    \hfill
    \label{enum-univ2}
    (4/9,16)
    \item システムプログラム特論
    \hfill
    \label{enum-univ2}
    (4/14)
    \item ソフトウェア開発法
    \hfill
    \label{enum-univ2}
    (4/14)
    \item プログラミング方法論
    \hfill
    \label{enum-univ2}
    (4/15)
    \item ヒューマンコンピュータインタラクション
    \hfill
    \label{enum-univ2}
    (4/17)


  \end{enumerate}

\section{KPT}
  \subsection{GOOD}
  \begin{enumerate}
   \item 運動ができた
   \item 掃除できた
   \item 23歳になった
  \end{enumerate}

  \subsection{BAD}
  \begin{enumerate}
   \item ダラダラと作業してしまう
   \item 浪費癖が治らない
  \end{enumerate}

\section{近況報告}
本日(4月20日)は自身の誕生日である.
今年で23歳となった.
23歳の抱負は,何事も中途半端にしないこととする.
これまでは中途半端に終えてしまったことが多かったため,
これからは最後までしっかりと考え,やり遂げる.
\section{学会情報}
\begin{enumerate}
    \item 2015年並列/分散/協調処理に関する『別府』サマー・ワークショップ (SWoPP2015)
        \begin{description}
            \item[開催期間:]平成27年8月4日(火)〜8月6日(木)
            \item[開催場所:] ビーコンプラザ 別府国際コンベンションセンター (大分県別府) 
            \item[原稿締め切り:]未定(6月末)
        \end{description}

\end{enumerate}

\end{document}
