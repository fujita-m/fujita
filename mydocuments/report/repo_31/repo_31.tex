\documentclass[fleqn, 14pt]{extarticle}
\usepackage{reportForm}
\usepackage[utf8]{inputenc}
\usepackage[T1]{fontenc}
\usepackage{fixltx2e}
\usepackage{graphicx}
\usepackage{longtable}
\usepackage{float}
\usepackage{wrapfig}
\usepackage[normalem]{ulem}
\usepackage{textcomp}
\usepackage{marvosym}
\usepackage{wasysym}
\usepackage{latexsym}
\usepackage{amssymb}
\usepackage{amstext}
\usepackage{hyperref}
\usepackage{comment}
\tolerance=1000
\subtitle{(2015年7月10日$\sim$2015年8月2日)}
\usepackage{strike}
\setcounter{section}{-1}
\author{乃村研究室M1\\藤田 将輝}
\date{2015年8月3日}
\title{記録書 No.31}
\hypersetup{
  pdfkeywords={},
  pdfsubject={},
  pdfcreator={Emacs 24.3.1 (Org mode 8.0.3)}}
\begin{document}
\maketitle
\section{前回ミーティングからの指導・指摘事項}
\label{sec-1}
\begin{enumerate}
\item 特になし
\newline
\hfill

\end{enumerate}

\section{実績}
\label{sec-2}

\subsection{研究関連}
\label{sec-2-1}
\begin{enumerate}
    \item 研究テーマに関する項目
    \hfill
    \label{enum-research1}
    \begin{enumerate}

        \item 参考文献の読解
        \hfill
        \label{enum-1-A}
        (50%,+0%)
        \item 使用する共有メモリ領域の検討
        \hfill
        \label{enum-1-B}
        (75%,+0%)
        \item NICのデバイスドライバの改変箇所の調査
        \hfill
        \label{enum-1-C}
        (50%,+0%)
        \item パケット受信処理の実装
        \hfill
        \label{enum-1-D}
        (100%,+1%)

    \end{enumerate}
    \item 開発に関する項目
    \hfill
    \label{enum-research2}
    \begin{enumerate}

        \item 自動ビルドスクリプトの作成
        \hfill
        \label{enum-2-A}
        (95%,+0%)
    \end{enumerate}

    \item 第280回New打ち合わせ
    \hfill
    \label{enum-research3}
    (7/16)
    \item 第281回New打ち合わせ 
    \hfill
    \label{enum-research3}
    (7/27)
    \end{enumerate}

\subsection{研究室関連}
\label{sec-2-2}
\begin{enumerate}

\item 全体ミーティング
\hfill
\label{enum-lab1}
(7/10)

\item 乃村研ミーティング
\hfill
\label{enum-lab2}
(7/22)

\item クラウド勉強会
\hfill
\label{enum-lab2}
(7/23)

\end{enumerate}

\subsection{大学院関連}
\label{sec2-3}
\begin{enumerate}

    \item 特になし
    \hfill
    \label{enum-univ2}

\end{enumerate}

\section{詳細および反省・感想}
\label{sec-3}
%\setcounter{subsection}{1}
\subsection{研究関連}
\label{sec-3-2}
\begin{itemize}
\item[(\ref{enum-1-D})]
\end{itemize}
デバッグ支援環境においてパケットを作成する機能を作成した.
libnetというパケットを作成するライブラリを用いてこの機能を実現した.
デバッグ対象OS上で動作する,受信したUDPのメッセージを表示するアプリケーション
で,作成したパケットが正常に処理できていることを確認した.
今後は,作成したパケットを連続で処理させた際に,どの程度
正常に処理できるかを測定する.

\section{今後の予定}
\label{sec-4}
\subsection{研究関連}
\label{sec-4-1}

\begin{enumerate}

\item 研究テーマに関する項目
\hfill
\begin{enumerate}


\item 参考文献の読解
\hfill
(8月下旬)

\item 使用する共有メモリ領域の検討
\hfill
(8月中旬)

\item NICのデバイスドライバの改変箇所の調査
\hfill
(8月中旬)

\end{enumerate}
\item 開発に関する項目
\hfill
\begin{enumerate}

\item 自動ビルドスクリプトの作成
\hfill
(8月中旬)

\end{enumerate}

\item 第282回New打ち合わせ
\hfill
\label{enum-7}
(8/3)

\end{enumerate}
\subsection{研究室関連}
\label{sec-4-2}

\begin{enumerate}

\item 全体ミーティング
\hfill
\label{enum-18}
(8/3)

\item 乃村研ミーティング 
\hfill
\label{enum-18}
(8/25)

\end{enumerate}

\subsection{大学院関連}
\begin{enumerate}

\item 特になし
\hfill
\label{enum-17}


\end{enumerate}

\subsection{学会情報} 
\begin{enumerate}
    \item 電気・情報関連学会中国支部大会\\
        開催日時:平成27年度10月17日(土)\\
        開催場所:山口県 山口大学工学部 常磐キャンパス\\
        申込締切:8月25日(火)\\
        原稿締め切り:8月25日(火)\\
\end{enumerate}
\end{document}
