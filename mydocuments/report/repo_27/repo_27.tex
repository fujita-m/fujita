\documentclass[fleqn, 14pt]{extarticle}
\usepackage{reportForm}
\usepackage[utf8]{inputenc}
\usepackage[T1]{fontenc}
\usepackage{fixltx2e}
\usepackage{graphicx}
\usepackage{longtable}
\usepackage{float}
\usepackage{wrapfig}
\usepackage[normalem]{ulem}
\usepackage{textcomp}
\usepackage{marvosym}
\usepackage{wasysym}
\usepackage{latexsym}
\usepackage{amssymb}
\usepackage{amstext}
\usepackage{hyperref}
\usepackage{comment}
\tolerance=1000
\subtitle{(2015年5月15日$\sim$2015年6月11日)}
\usepackage{strike}
\setcounter{section}{-1}
\author{乃村研究室M1\\藤田 将輝}
\date{2015年6月12日}
\title{記録書 No.27}
\hypersetup{
  pdfkeywords={},
  pdfsubject={},
  pdfcreator={Emacs 24.3.1 (Org mode 8.0.3)}}
\begin{document}
\maketitle
\section{前回ミーティングからの指導・指摘事項}
\label{sec-1}
\begin{enumerate}
\item 特になし
\newline
\hfill

\end{enumerate}

\section{実績}
\label{sec-2}

\subsection{研究関連}
\label{sec-2-1}
\begin{enumerate}
    \item 研究テーマに関する項目
    \hfill
    \label{enum-research1}
    \begin{enumerate}

        \item 参考文献の読解
        \hfill
        \label{enum-1-A}
        (50%,+0%)
        \item 使用する共有メモリ領域の検討
        \hfill
        \label{enum-1-B}
        (75%,+0%)
        \item NICのデバイスドライバの改変箇所の調査
        \hfill
        \label{enum-1-C}
        (50%,+0%)
        \item パケット受信処理の実装
        \hfill
        \label{enum-1-D}
        (65%,+5%)

    \end{enumerate}
    \item 開発に関する項目
    \hfill
    \label{enum-research2}
    \begin{enumerate}

        \item 自動ビルドスクリプトの作成
        \hfill
        \label{enum-2-A}
        (95%,+0%)
    \end{enumerate}

    \item 第20回New開発打ち合わせ
    \hfill
    \label{enum-research3}
    (5/15)
    \item 第276回New打ち合わせ 
    \hfill
    \label{enum-research3}
    (5/26)
    \item 第21回New開発打ち合わせ 
    \hfill
    \label{enum-research3}
    (6/01)
    \item 第277回New開発打ち合わせ
    \hfill
    \label{enum-research3}
    (6/10)
    \end{enumerate}

\subsection{研究室関連}
\label{sec-2-2}
\begin{enumerate}

\item 全体ミーティング 
\hfill
\label{enum-lab1}
(5/15)


\item 平成27年度第1回研究室内部屋別対抗ボウリング大会
\hfill
\label{enum-lab2}
(5/15)

\item 乃村研ミーティング
\hfill
\label{enum-lab2}
(6/1)

\end{enumerate}

\subsection{大学院関連}
\label{sec2-3}
\begin{enumerate}

    \item 特になし
    \hfill
    \label{enum-univ2}

\end{enumerate}

\section{詳細および反省・感想}
\label{sec-3}
%\setcounter{subsection}{1}
\subsection{研究関連}
\label{sec-3-2}
\begin{itemize}
\item[(\ref{enum-1-D})]
    NICの動作の再現する機構を実現している.この機構の中のパケットを
    作成する機能を作成している.Etherフレームが作成される
    処理流れを調査し,調査した処理流れを参考にパケットを作成する.
    作成したパケットをMintの共有メモリに配置し,一方のOSから
    他方のOSへ任意のタイミングで割り込みを発生させることにより,
    作成したパケットを処理させる.

\end{itemize}

\section{今後の予定}
\label{sec-4}
\subsection{研究関連}
\label{sec-4-1}

\begin{enumerate}

\item 研究テーマに関する項目
\hfill
\begin{enumerate}


\item 参考文献の読解
\hfill
(6月下旬)

\item 使用する共有メモリ領域の検討
\hfill
(6月下旬)

\item NICのデバイスドライバの改変箇所の調査
\hfill
(7月中旬)

\item パケット受信処理の実装
\hfill
(6月下旬)
\end{enumerate}
\item 開発に関する項目
\hfill
\begin{enumerate}

\item 自動ビルドスクリプトの作成
\hfill
(7月中旬)

\end{enumerate}

\item 第22回New開発打ち合わせ
\hfill
\label{enum-7}
(6/15)

\item 第278回New開発打ち合わせ
\hfill
\label{enum-7}
(6/22)
\end{enumerate}
\subsection{研究室関連}
\label{sec-4-2}

\begin{enumerate}

\item 全体ミーティング
\hfill
\label{enum-18}
(6/12)

\item 乃村研ミーティング 
\hfill
\label{enum-18}
(6/22)

\item M2論文紹介
\hfill
\label{enum-18}
(6/26)

\end{enumerate}

\subsection{大学院関連}
\begin{enumerate}

\item 特になし
\hfill
\label{enum-17}


\end{enumerate}










\end{document}
