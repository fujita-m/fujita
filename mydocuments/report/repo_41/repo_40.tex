\documentclass[fleqn, 14pt]{extarticle}
\usepackage{reportForm}
\usepackage[utf8]{inputenc}
\usepackage[T1]{fontenc}
\usepackage{fixltx2e}
\usepackage{graphicx}
\usepackage{longtable}
\usepackage{float}
\usepackage{wrapfig}
\usepackage[normalem]{ulem}
\usepackage{textcomp}
\usepackage{marvosym}
\usepackage{wasysym}
\usepackage{latexsym}
\usepackage{amssymb}
\usepackage{amstext}
\usepackage{hyperref}
\usepackage{comment}
\tolerance=1000
\subtitle{(2015年11月19日$\sim$2015年12月24日)}
\usepackage{strike}
\setcounter{section}{-1}
\author{乃村研究室M1\\藤田 将輝}
\date{2015年12月25日}
\title{記録書 No.41}
\hypersetup{
    pdfkeywords={},
    pdfsubject={},
    pdfcreator={Emacs 24.3.1 (Org mode 8.0.3)}}
    \begin{document}
    \maketitle

    \section{前回ミーティングからの指導・指摘事項}
    \label{sec-1}
    \begin{enumerate}
        \item 特になし
    \end{enumerate}

    \section{実績}
    \label{sec-2}

    \subsection{研究関連}
    \label{sec-2-1}
    \begin{enumerate}

        \item 研究テーマに関する項目
            \hfill
            \label{enum-research1}
            \begin{enumerate}

                \item 参考文献の読解
                    \hfill
                    \label{enum-1-A}
                    (50%,+0%)

                \item バグの再現
                    \hfill
                    \label{enum-1-B}
                    (0%,+0%)

                \item 第136回OS研原稿執筆
                    \hfill
                    \label{enum-1-C}
                    (10%,+10%)

                \item 第136回OS研スライド作成
                    \hfill
                    \label{enum-1-D}
                    (10%,+10%)

            \end{enumerate}

        \item 開発に関する項目
            \hfill
            \label{enum-research2}
            \begin{enumerate}

                \item 自動ビルドスクリプトの作成
                    \hfill
                    \label{enum-2-A}
                    (95%,+0%)

            \end{enumerate}

        \item 第290回New打ち合わせ
            \hfill
            \label{enum-research3}
            (11/25)

        \item Newもくもく会
            \hfill
            \label{enum-research3}
            (11/30, 12/15)

        \item 第291回New打ち合わせ
            \hfill
            \label{enum-research3}
            (12/08)

        \item 第292回New打ち合わせ
            \hfill
            \label{enum-research3}
            (12/17)

    \end{enumerate}
    \subsection{研究室関連}
    \label{sec-2-2}
    \begin{enumerate}

        \item 全体ミーティング
            \hfill
            \label{enum-18}
            (11/19)

        \item SWLAB忘年会
            \hfill
            \label{enum-18}
            (12/01)

        \item M2中間発表
            \hfill
            \label{enum-18}
            (12/01)

    \end{enumerate}

    \subsection{大学院関連}
    \label{sec2-3}
    \begin{enumerate}

        \item 特になし
            \hfill
            \label{enum-univ1}

    \end{enumerate}

    \section{詳細および反省・感想}
    \label{sec-3}

    %\setcounter{subsection}{1}
    \subsection{研究関連}
    \label{sec-3-2}

    \begin{itemize}

        \item[(\ref{enum-1-C})]
            第136回OS研の原稿を執筆している.主に評価の章を執筆している.
            評価の中で,NICドライバの性能測定を行なっており,どの程度の
            間隔ならば正常にパケットを処理できるかについて測定をしている.
            結果はグラフとして示しており,このグラフが正しいことを検証する
            ため,ソースコードを解読し,パケットの処理流れを再確認している.
            どの処理にどれだけの時間がかかっているかを細かく調査し,結果が
            妥当であることを示す.

    \end{itemize}

    \section{今後の予定}
    \label{sec-4}

    \subsection{研究関連}
    \label{sec-4-1}

    \begin{enumerate}

        \item 研究テーマに関する項目
            \hfill
            \begin{enumerate}

                \item 参考文献の読解
                    \hfill
                    (1月中旬)

                \item バグの再現
                    \hfill
                    (2月上旬)

            \end{enumerate}

        \item 開発に関する項目
            \hfill
            \begin{enumerate}

                \item 自動ビルドスクリプトの作成
                    \hfill
                    (12月中旬)

            \end{enumerate}

        \item 第293回New打ち合わせ
            \hfill
            \label{enum-7}
            (12/06)

    \end{enumerate}

    \subsection{研究室関連}
    \label{sec-4-2}

    \begin{enumerate}

        \item 全体ミーティング
            \hfill
            \label{enum-18}
            (12/25)

        \item 乃村研書初め
            \hfill
            \label{enum-18}
            (01/06)

        \item 乃村研ミーティング
            \hfill
            \label{enum-18}
            (01/13)

    \end{enumerate}

    \subsection{大学院関連}
    \begin{enumerate}

        \item 特になし
            \hfill
            \label{enum-17}

    \end{enumerate}
    \subsection{学会情報} 
    \begin{enumerate}
        \item 第136回システムソフトウェアとオペレーティングシステム研究会\\
            開催日時:2016年2月29日(月)\\
            開催場所:理化学研究所計算科学研究機構\\
            申込締切:1月12日(火)\\
            原稿締め切り:2月2日(火)\\
    \end{enumerate}

    \end{document}
