\documentclass[fleqn, 14pt]{extarticlej}
\usepackage{sty/reportForm}
\usepackage{hyperref}
\tolerance=1000
\subtitle{(2015年06月01日$\sim$2015年06月24日)}
\usepackage{sty/strike}
\usepackage{nutils}
\setcounter{section}{0}
\author{乃村研究室M1\\藤田 将輝}
\date{2015年06月25日}
\title{記録書 No.28}
\begin{document}

\maketitle
\section{実績,詳細,および反省・感想}
\subsection{研究関連}
\label{sec-2-1}
\begin{enumerate}
    \item 研究テーマに関する項目
    \hfill
    \label{enum-research1}
    \begin{enumerate}
        \item 参考文献の読解
        \hfill
        \label{enum-1-A}
        (50%,+0%)
        \item 使用する共有メモリ領域の検討
        \hfill
        \label{enum-1-B}
        (75%,+0%)
        \item NICのデバイスドライバの改変箇所の調査
        \hfill
        \label{enum-1-C}
        (50%,+0%)
        \item パケットの作成
        \hfill
        \label{enum-1-D}
        (90%,+20%)\\
        パケットのジェネレータを作成している.現在は,任意のメッセージの
        UDPパケットを作成できている.作成したパケットを用いることで,
        OSノード0からOSノード1へNICを用いずにメッセージを送信できることを
        確認した.パケットの作成時に,各種のヘッダの情報はキャプチャした
        パケットの情報をそのまま用いている.このため,サイズやチェックサム
        等の情報をメッセージによって変動させることができない.したがって,
        今後は,IPパケットの作成用のライブラリを調査することで,
        パケットの作成のプロセスを理解し,ジェネレータの実装をすすめる.
  
    \end{enumerate}
    \item 開発に関する項目
    \hfill
    \label{enum-research2}
    \begin{enumerate}

        \item 自動ビルドスクリプトの作成
        \hfill
        \label{enum-2-A}
        (95%,+0%)
    \end{enumerate}
    \item 第21回New開発打ち合わせ
    \hfill
    \label{enum-research3}
    (6/01)
    \item 第277回New打ち合わせ
    \hfill
    \label{enum-research3}
    (6/10)
    \item 第22回New開発打ち合わせ
    \hfill
    \label{enum-research3}
    (6/15)
    \item 第278回New打ち合わせ
    \hfill
    \label{enum-research3}
    (6/19)

\end{enumerate}
  \subsection{研究室関連}
  \begin{enumerate}
   \item 乃村研ミーティング
         \hfill
         (06/01)
   \item 全体ミーティング
         \hfill
         (06/12)
   \item Newもくもく会
         \hfill
         (06/22)
  \end{enumerate}

  \subsection{大学院関連}
  \begin{enumerate}
      \item 特になし
  \end{enumerate}

\section{KPT}
  \subsection{GOOD}
  \begin{enumerate}
   \item もくもく会で目標を達成できた.
   \item 焼肉が美味しかった.
   \item 米を炊いて持ってくることで節約できている.
  \end{enumerate}

  \subsection{BAD}
  \begin{enumerate}
   \item 蒸し暑いせいか,気分が晴れない.
   \item 下宿先ですでにクーラーをガンガン使っている.
   \item 寝坊が多い.
  \end{enumerate}

\section{近況報告}
自身は,週に1度程度バスケットボールをしている.
最近暑くなってきたせいか,すぐにバテてしまい,足が止まってしまう.
自身はパス,シュート,およびドリブルが上手くはないため,
人より走ることでなんとかチームに貢献できていた.
しかし,今は人並みかそれ以下の運動量である.
対策として,体重を減らすことが効果的ではないかと考えている.
この他にも普段の生活で体力の維持ができるトレーニング法を知っている方がいれば
教えてください.

\end{document}
