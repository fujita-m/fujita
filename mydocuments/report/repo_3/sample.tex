\documentclass[fleqn, 14pt]{extarticle}
                    \usepackage{reportForm}
\usepackage[utf8]{inputenc}
\usepackage[T1]{fontenc}
\usepackage{fixltx2e}
\usepackage{graphicx}
\usepackage{longtable}
\usepackage{float}
\usepackage{wrapfig}
\usepackage[normalem]{ulem}
\usepackage{textcomp}
\usepackage{marvosym}
\usepackage{wasysym}
\usepackage{latexsym}
\usepackage{amssymb}
\usepackage{amstext}
\usepackage{hyperref}
\tolerance=1000
\subtitle{(2014年04月11日$\sim$2014年04月27日)}
\usepackage{strike}
\setcounter{section}{-1}
\author{乃村研究室B4\\藤田 将輝}
\date{2014年04月28日}
\title{記録書 No.2}
\hypersetup{
  pdfkeywords={},
  pdfsubject={},
  pdfcreator={Emacs 24.3.1 (Org mode 8.0.3)}}
\begin{document}

\maketitle




\section{前回ミーティングからの指導・指摘事項}
\label{sec-1}
\begin{enumerate}
\item 特になし
\end{enumerate}




\section{実績}
\label{sec-2}


\subsection{研究関連}
\label{sec-2-1}
\begin{enumerate}
\item 研究テーマに関する項目
\hfill
\label{enum-research1}
\begin{enumerate}

\item 「Mintオペレーティングシステムを用いた割り込み処理のデバッグ支援環境の提案」の要約
\hfill
\label{enum-1-A}
(100%,+100%)
\end{enumerate}
\end{enumerate}


\subsection{研究室関連}
\label{sec-2-2}

\begin{enumerate}
\item 第24回乃村杯
\hfill
\label{enum-laboratory1}
(04/28)
\item 乃村研ミーティング
\hfill
\label{enum-laboratory2}
(04/28)
\item 平成26年度B4英語勉強会
\hfill
\label{enum-laboratory3}
(05/01,08)
\item 第250回New打ち合わせ
\hfill
\label{enum-laboratory4}
(05/07)
\item New開発打ち合わせ
\hfill
\label{enum-laboratory5}
(05/13)
\end{enumerate}

\subsection{大学・大学院関連}
\label{sec-2-3}

\begin{enumerate}
\item 特になし
\hfill
\end{enumerate}





\section{詳細および反省・感想}
\label{sec-3}
\subsection{研究関連}
\label{sec-3-1}

\begin{itemize}
\item[(\ref{enum-1-A})]
研究テーマが決定した.
研究テーマは「Mintオペレーティングシステムを用いた割り込み処理のデバッグ支援環境の提案」である.
そのため,山本の特別研究報告書である「Mintオペレーティングシステムを用いた割り込み処理のデバッグ支援環境の提案」
を要約した.
今できていることとできていないことを確認し,整理できた.
割り込み処理における知識が必要であると感じたため,これからは
残された文書や方法を確認して割り込み処理の理解に努める.

\end{itemize}

\subsection{研究室関連}
\label{sec-3-2}

\begin{itemize}
\item[(\ref{enum-laboratory1})]
第24回乃村杯に参加した.
初の乃村杯はビリヤードであった.
研究室の方々との交流を深められた.
ビリヤードは思っていたよりも難しく,結果は11位であった.
\item[(\ref{enum-laboratory3})]
平成26年度B4英語勉強会に参加した.
最初に勉強したときよりもスコアが取れるようになってきたと感じる.
公開TOEICが5月25日にあるので550点を目指して勉強する.
\end{itemize}








\section{今後の予定}
\label{sec-4}
\subsection{研究関連}
\label{sec-4-1}

\begin{enumerate}
\item 研究テーマに関する項目
\hfill
\begin{enumerate}

\item 研究テーマの決定
\hfill
(05/07)

\end{enumerate}
\end{enumerate}

\subsection{研究室関連}
\label{sec-4-2}

\begin{enumerate}
\item New開発打ち合わせ
\hfill
\label{enum-3}
(05/26)
\item 乃村研ミーティング
\hfill
\label{enum-4}
(05/29)
\end{enumerate}
\subsection{大学関連}
\begin{enumerate}
\item 情報化における職業
\hfill
\label{enum-5}
(05/23,30)
\item 公開TOEIC
\hfill
\label{enum-6}
(05/25)
\item カレッジTOEIC
\hfill
\label{enum-7}
(05/31)
\end{enumerate}





\end{document}
