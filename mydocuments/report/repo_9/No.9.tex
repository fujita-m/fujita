\documentclass[fleqn, 14pt]{extarticle}
                    \usepackage{reportForm}
\usepackage[utf8]{inputenc}
\usepackage[T1]{fontenc}
\usepackage{fixltx2e}
\usepackage{graphicx}
\usepackage{longtable}
\usepackage{float}
\usepackage{wrapfig}
\usepackage[normalem]{ulem}
\usepackage{textcomp}
\usepackage{marvosym}
\usepackage{wasysym}
\usepackage{latexsym}
\usepackage{amssymb}
\usepackage{amstext}
\usepackage{hyperref}
\usepackage{comment}
\tolerance=1000
\subtitle{(2014年09月08日$\sim$2014年09月22日)}
\usepackage{strike}
\setcounter{section}{-1}
\author{乃村研究室B4\\藤田 将輝}
\date{2014年09月24日}
\title{記録書 No.9}
\hypersetup{
  pdfkeywords={},
  pdfsubject={},
  pdfcreator={Emacs 24.3.1 (Org mode 8.0.3)}}
\begin{document}

\maketitle




\section{前回ミーティングからの指導・指摘事項}
\label{sec-1}
\begin{enumerate}
\item 特になし
\newline
\hfill

\end{enumerate}




\section{実績}
\label{sec-2}

\subsection{研究関連}
\label{sec-2-1}
\begin{enumerate}
\item 研究テーマに関する項目
\hfill
\label{enum-research1}
\begin{enumerate}

\item 参考文献の読解
\hfill
\label{enum-1-A}
(50%,+0%)
\item 使用する共有メモリ領域の検討
\hfill
\label{enum-1-B}
(55%,+50%)
\end{enumerate}
\item 開発に関する項目
\hfill
\label{enum-research2}
\begin{enumerate}

\item 自動ビルドスクリプトの作成
\hfill
\label{enum-2-A}
(95%,+5%)
\end{enumerate}

\item 第10回New開発打ち合わせ
\hfill
\label{enum-3}
(09/09)
\item 第260回New打ち合わせ
\hfill
\label{enum-4}
(09/17)



\end{enumerate}

\subsection{研究室関連}
\label{sec-2-2}
\begin{enumerate}
\item 乃村研ミーティング
\hfill
\label{enum-lab1}
(09/08)
\item 第25回乃村杯
\hfill
\label{enum-lab2}
(09/08)
\item 岡山大学香川大学合同研究会
\hfill
\label{enum-lab3}
(09/16)
\item Haskell勉強会
\hfill
\label{enum-lab4}
(09/18,19)
\end{enumerate}
\subsection{大学・大学院関連}
\label{sec-2-3}

\begin{enumerate}
\item 特になし
\hfill
\label{enum-univ2}
\end{enumerate}





\section{詳細および反省・感想}
\label{sec-3}
\subsection{研究室関連}
\label{sec-3-2}

\begin{itemize}
\item[(\ref{enum-lab2})]
第25回乃村杯に参加した.種目はレーシングゲーム(マリオカート)だった.
最終結果は8位だった.
密かに1位を狙っていたが,練習不足からか,思うように操作ができず残念な結果になってしまった.
今後はどのような競技でも事前に練習を重ねることでイメージ通りの動きができるようにする.

\item[(\ref{enum-lab3})]
岡山大学香川大学合同研究会に参加した.
香川大学の方々がどのような研究をされているのかを知ることができた.
また,岡山大学の先輩方の発表を聞くことができた.
堂々とした態度を自身の発表の参考にする.
夜に開かれた懇親会にも参加した.
懇親会では香川大学の方々と交流を深めることができた.
\end{itemize}

\section{今後の予定}
\label{sec-4}
\subsection{研究関連}
\label{sec-4-1}

\begin{enumerate}
\item 研究テーマに関する項目
\hfill
\begin{enumerate}


\item 参考文献の読解
\hfill
(09/29)

\item 使用する共有メモリ領域の検討
\hfill
(09/30)


\end{enumerate}
\item 開発に関する項目
\hfill
\begin{enumerate}

\item 自動ビルドスクリプトの作成
\hfill
(09/24)

\end{enumerate}
\item 第11回Newグループ開発打ち合わせ
\hfill
\label{enum-7}
(09/25)
\item 第261回New打ち合わせ
\hfill
\label{enum-3}
(09/30)
\end{enumerate}

\subsection{研究室関連}
\label{sec-4-2}

\begin{enumerate}



\item 乃村研ミーティング
\hfill
\label{enum-8}
(10/06)
\end{enumerate}


\section{その他}
9月15日に友人とMAZDA Zoom-Zoom スタジアムで広島東洋カープと読売ジャイアンツの試合を観戦した.
私はカープを応援した.
初めてのプロ野球観戦だったため,非常に興奮した.
私が座っていた席の周りの方々とその興奮を分かち合うことができた.
野球観戦の面白さと,人と楽しさを共有する面白さを感じられた.







\end{document}
