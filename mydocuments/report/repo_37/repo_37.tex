\documentclass[fleqn, 14pt]{extarticle}
\usepackage{reportForm}
\usepackage[utf8]{inputenc}
\usepackage[T1]{fontenc}
\usepackage{fixltx2e}
\usepackage{graphicx}
\usepackage{longtable}
\usepackage{float}
\usepackage{wrapfig}
\usepackage[normalem]{ulem}
\usepackage{textcomp}
\usepackage{marvosym}
\usepackage{wasysym}
\usepackage{latexsym}
\usepackage{amssymb}
\usepackage{amstext}
\usepackage{hyperref}
\usepackage{comment}
\tolerance=1000
\subtitle{(2015年9月25日$\sim$2015年10月27日)}
\usepackage{strike}
\setcounter{section}{-1}
\author{乃村研究室M1\\藤田 将輝}
\date{2015年10月27日}
\title{記録書 No.37}
\hypersetup{
    pdfkeywords={},
    pdfsubject={},
    pdfcreator={Emacs 24.3.1 (Org mode 8.0.3)}}
    \begin{document}
    \maketitle

    \section{前回ミーティングからの指導・指摘事項}
    \label{sec-1}
    \begin{enumerate}
        \item 失敗をした際は具体的な対策を検討し,実行する.
            \hfill
            [10/16, メール,乃村先生]
            \newline

    \end{enumerate}

    \section{実績}
    \label{sec-2}

    \subsection{研究関連}
    \label{sec-2-1}
    \begin{enumerate}

        \item 研究テーマに関する項目
            \hfill
            \label{enum-research1}
            \begin{enumerate}

                \item 参考文献の読解
                    \hfill
                    \label{enum-1-A}
                    (50%,+0%)

                \item 正常に処理できる通信量の測定
                    \hfill
                    \label{enum-1-B}
                    (90%,+20%)

                \item バグの再現
                    \hfill
                    \label{enum-1-C}
                    (0%,+0%)

            \end{enumerate}

        \item 開発に関する項目
            \hfill
            \label{enum-research2}
            \begin{enumerate}

                \item 自動ビルドスクリプトの作成
                    \hfill
                    \label{enum-2-A}
                    (95%,+0%)

                \item MintのGRUB2への対応
                    \hfill
                    \label{enum-2-B}
                    (90%,+90%)

            \end{enumerate}

        \item 第286回New打ち合わせ
            \hfill
            \label{enum-research3}
            (10/02)

        \item 第27回New開発打ち合わせ 
            \hfill
            \label{enum-research3}
            (10/08)

        \item 第287回New打ち合わせ
            \hfill
            \label{enum-research3}
            (10/13)

        \item 第28回New開発打ち合わせ 
            \hfill
            \label{enum-research3}
            (10/22)

    \end{enumerate}

    \subsection{研究室関連}
    \label{sec-2-2}
    \begin{enumerate}

        \item 全体ミーティング
            \hfill
            \label{enum-lab1}
            (9/25)

        \item Rails勉強会
            \hfill
            \label{enum-lab1}
            (9/28)

        \item 乃村研究室研修会
            \hfill
            \label{enum-lab1}
            (9/29,30)

        \item 乃村研ミーティング
            \hfill
            \label{enum-lab1}
            (10/09)

        \item 乃村杯
            \hfill
            \label{enum-lab1}
            (10/09)

    \end{enumerate}

    \subsection{大学院関連}
    \label{sec2-3}
    \begin{enumerate}

        \item 特になし
            \hfill
            \label{enum-univ1}

    \end{enumerate}

    \section{詳細および反省・感想}
    \label{sec-3}

    %\setcounter{subsection}{1}
    \subsection{研究関連}
    \label{sec-3-2}

    \begin{itemize}

        \item[(\ref{enum-2-B})]
            GRUB2を用いてMintのOSノードを起動した際,起動対象のOSが指定したメモリ位置に
            カーネルを配置できていないことが分かった.これは改変したセットアップルーチンを
            GRUB2では通らないことが原因であった.このため,起動したOSノードをKexecを用いて
            再起動することにより,指定したメモリ位置にカーネルを配置できた.今後,これらの操作による
            不具合の有無を調査する.

    \end{itemize}

    \section{今後の予定}
    \label{sec-4}

    \subsection{研究関連}
    \label{sec-4-1}

    \begin{enumerate}

        \item 研究テーマに関する項目
            \hfill
            \begin{enumerate}

                \item 参考文献の読解
                    \hfill
                    (11月中旬)

                \item バグの再現
                    \hfill
                    (11月上旬)

            \end{enumerate}

        \item 開発に関する項目
            \hfill
            \begin{enumerate}

                \item 自動ビルドスクリプトの作成
                    \hfill
                    (11月中旬)

            \end{enumerate}

        \item 第288回New打ち合わせ
            \hfill
            \label{enum-7}
            (10/29)

        \item 第28回New開発打ち合わせ
            \hfill
            \label{enum-7}
            (11/06)

    \end{enumerate}

    \subsection{研究室関連}
    \label{sec-4-2}

    \begin{enumerate}

        \item 全体ミーティング
            \hfill
            \label{enum-18}
            (10/27)

        \item 平成27年度第2回部屋別対抗ボウリング大会
            \hfill
            \label{enum-18}
            (10/27)

        \item 平成27年度M2およびM1論文紹介
            \hfill
            \label{enum-18}
            (10/30)

        \item 乃村研究室忘年会
            \hfill
            \label{enum-18}
            (11/14)

        \item SWLAB忘年会
            \hfill
            \label{enum-18}
            (12/01)

    \end{enumerate}

    \subsection{大学院関連}
    \begin{enumerate}

        \item 特になし
            \hfill
            \label{enum-17}

    \end{enumerate}

    \end{document}
