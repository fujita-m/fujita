\documentclass[fleqn, 14pt]{extarticle}
                    \usepackage{reportForm}
\usepackage[utf8]{inputenc}
\usepackage[T1]{fontenc}
\usepackage{fixltx2e}
\usepackage{graphicx}
\usepackage{longtable}
\usepackage{float}
\usepackage{wrapfig}
\usepackage[normalem]{ulem}
\usepackage{textcomp}
\usepackage{marvosym}
\usepackage{wasysym}
\usepackage{latexsym}
\usepackage{amssymb}
\usepackage{amstext}
\usepackage{hyperref}
\usepackage{comment}
\tolerance=1000
\subtitle{(2014年05月19日$\sim$2014年05月28日)}
\usepackage{strike}
\setcounter{section}{-1}
\author{乃村研究室B4\\藤田 将輝}
\date{2014年05月29日}
\title{記録書 No.4}
\hypersetup{
  pdfkeywords={},
  pdfsubject={},
  pdfcreator={Emacs 24.3.1 (Org mode 8.0.3)}}
\begin{document}

\maketitle




\section{前回ミーティングからの指導・指摘事項}
\label{sec-1}
\begin{enumerate}
\item 自分がほしいと思う機能はすでに実装されている可能性が高い.
このため,その機能が存在していると思って調べる.
\newline
\hfill
[5/26, 201,乃村先生]
\end{enumerate}




\section{実績}
\label{sec-2}


\subsection{研究関連}
\label{sec-2-1}
\begin{enumerate}
\item 研究テーマに関する項目
\hfill
\label{enum-research1}
\begin{enumerate}

\item IPI送受信の確認
\hfill
\label{enum-1-A}
(80%,+30%)
\item 参考文献の読解
\hfill
\label{enum-1-B}
(20%,+20%)
\end{enumerate}
\item 開発に関する項目
\hfill
\label{enum-research2}
\begin{enumerate}

\item 自動ビルドスクリプトの作成
\hfill
\label{enum-2-A}
(90%,+20%)
\end{enumerate}
\end{enumerate}


\subsection{研究室関連}
\label{sec-2-2}

\begin{enumerate}
\item 全体ミーティング
\hfill
\label{enum-laboratory1}
(05/19)
\item 第251回New打ち合わせ
\hfill
\label{enum-laboratory2}
(05/19)
\item 平成26年度第1回研究室内部屋別対抗ボウリング大会
\hfill
\label{enum-laboratory3}
(05/19)
\item 第4回Newグループ開発打ち合わせ
\hfill
\label{enum-laboratory4}
(05/26)
\end{enumerate}

\subsection{大学・大学院関連}
\label{sec-2-3}

\begin{enumerate}
\item 情報化における職業
\hfill
\label{enum-univ1}
(05/23)
\item 公開TOEIC
\hfill
\label{enum-univ2}
(05/25)
\end{enumerate}





\section{詳細および反省・感想}
\label{sec-3}
\subsection{研究関連}
\label{sec-3-1}

\begin{itemize}
\item[(\ref{enum-1-A})]
山本さんが作成した,IPIを用いてCPUへ割り込みを発生させるプログラムを実行した.
エラー処理に誤りがあったため,実行させるのに苦労した.
エラー処理の分岐条件を修正し,
実行に成功した.今後は,プログラムの実行に使われたシステムコールについて,残された文書
を読むことで調査する.

\item[(\ref{enum-1-B})]
山本さんの特別研究報告書の参考文献の1つである「仮想マシンモニタを用いた割り込み処理のデバッグ手法」[1]を読んだ.
割り込みにおけるバグの種類と,仮想マシンを用いたそのデバッグ方法を理解した.
山本さんの特別研究報告書の参考文献はあと5つある.これらの中には英語の論文もあるため,
英語を勉強する.
\end{itemize}


\begin{comment}
\subsection{研究室関連}
\label{sec-3-2}

\begin{itemize}
\item[(\ref{enum-laboratory3})]
平成26年度第1回研究室内部屋別対抗ボウリング大会に参加した.
乃村研の方達だけでなく,ほかの研究室の方達とも交流を深められた.
楽しい雰囲気で投げられたためか,個人で優勝することができた.
普段の力では出せない結果であるため,うれしかった.

\end{itemize}
\end{comment}

\subsection{大学関連}
\label{sec-3-3}
\begin{itemize}
\item[(\ref{enum-univ2})]
公開TOEICを受験した.
会場が岡山大学であったため,落ち着いて取り組むことができた.
リスニングの問題文がいつもよりもよく聞きとれた.
05/31にカレッジTOEICがあるため,これにむけて勉強する.
\end{itemize}








\section{今後の予定}
\label{sec-4}
\subsection{研究関連}
\label{sec-4-1}

\begin{enumerate}
\item 研究テーマに関する項目
\hfill
\begin{enumerate}

\item IPI送受信の確認
\hfill
(06/05)
\item 参考文献の読解
\hfill
(06/13)

\end{enumerate}
\item 開発に関する項目
\hfill
\begin{enumerate}

\item 自動ビルドスクリプトの作成
\hfill
(06/10)

\end{enumerate}
\end{enumerate}

\subsection{研究室関連}
\label{sec-4-2}

\begin{enumerate}
\item 第252回New打ち合わせ
\hfill
\label{enum-3}
(06/06)
\item 第5回Newグループ開発打ち合わせ
\hfill
\label{enum-4}
(06/11)
\item 全体ミーティング
\hfill
\label{enum-5}
(06/16)
\item 平成26年度M2論文紹介
\hfill
\label{enum-6}
(06/20)
\end{enumerate}
\subsection{大学関連}
\begin{enumerate}
\item カレッジTOEIC
\hfill
\label{enum-8}
(05/31)
\end{enumerate}

\section{参考文献}
\renewcommand{\labelenumi}{[\arabic{enumi}]}
\begin{enumerate}
\item 宮原俊介,吉村剛,山田浩史,河野健二:仮想マシンモニタを用いた割り込み処理のデバッグ手法,
情報処理学会研究報告,Vol.2013-OS-124,No.6,pp.1-8(2013).
\end{enumerate}

\end{document}
