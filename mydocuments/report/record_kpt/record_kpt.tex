\documentclass[fleqn, 12pt]{extarticlej}
\usepackage{sty/reportForm}
\usepackage{hyperref}
\tolerance=1000
\subtitle{(2015年02月17日$\sim$2015年03月10日)}
\usepackage{sty/strike}
\usepackage{nutils}
\setcounter{section}{0}
\author{乃村研究室M1\\岡田 卓也}
\date{2015年03月11日}
\title{記録書 No.42}
\begin{document}

\maketitle
\section{実績,詳細,および反省・感想}
  \subsection{研究関連}
  \begin{enumerate}
   \item Mac対応DTB試作
         \hfill
         (85%,+5%)(03/31)\\
         学会でDTBをデモするにあたり,パフォーマンスを改善した.
         また,履歴情報の参照時間と作業時間を視覚的に確認できるTimeline機能を実装した.
         しかし,肝心の履歴情報蓄積機能は実装が進んでいない.
         今後は実際に履歴情報を収集した上で情報をどう扱っていくかという検討を進めたい.
         このため,なんとかして履歴情報蓄積機能の実装を完了したい.
   \item Gitとの連携方法検討
         \hfill
         (10%,+0%)(03/31)
   \item ファイル閲覧履歴取得方法検討
         \hfill
         (90%,+0%)(03/31)
   \item 履歴情報提示方法検討
         \hfill
         (85%,+5%)(03/31)
   \item DPS162発表スライド作成
         \hfill
         (100%,+30%)
   \item LastNote開発に関する項目
         \hfill
         \begin{enumerate}
          \item RestfulなMemberの実装
                \hfill
                (10%,+0%)(無期延期)
         \end{enumerate}
  \end{enumerate}

  \subsection{研究室関連}
  \begin{enumerate}
   \item 全体ミーティング
         \hfill
         (02/17)
   \item 部屋別対抗ボウリング
         \hfill
         (02/17)
   \item 乃村研円卓会議
         \hfill
         (02/19)
   \item 第108回計算機幹事打ち合わせ
         \hfill
         (02/20)
   \item 第169回GN検討打合せ
         \hfill
         (02/23)
   \item 第10回ノムニチ開発進捗報告会
         \hfill
         (02/23)
   \item もくもく会
         \hfill
         (02/25)
   \item 乃村研ミーティング
         \hfill
         (03/09)
  \end{enumerate}

  \subsection{大学院関連}
  \begin{enumerate}
   \item 第2回進路指導説明会
         \hfill
         (03/02)
  \end{enumerate}

  \subsection{就職活動関連}
  \begin{enumerate}
   \item 日本電気株式会社会社説明会
         \hfill
         (03/02)
   \item 株式会社日立製作所会社説明会
         \hfill
         (03/03)
   \item 業界研究セミナー in 東京
         \hfill
         (03/07)\\
         学研開催の業界研究セミナー in 東京に参加した.
         これは,俗に言う合同説明会である.
         自分の就職後を想像できる説明を聞けず残念だった.
         理工系向け合同説明会でもこのような状況であるため,
         合同説明会参加の意義はあまりないと感じられた.
         しかし,参加する意義があまりないと認識できた点は収穫であった.
         来年度就職活動をする方は1度は参加しておくことを勧める.
  \end{enumerate}

\section{KPT}
  \subsection{GOOD}
  \begin{enumerate}
   \item DTBのパフォーマンスを改善した
   \item 未読状態の本\cite{uenishi2013a}を読み始めた
   \item まずはやってみる精神が復調した
  \end{enumerate}

  \subsection{BAD}
  \begin{enumerate}
   \item DTBの履歴情報蓄積機能の実装を放置している
   \item 初めてのことを周囲に確認を取らずに行ってしまった
   \item 学会情報に登壇時間帯に関する情報を記載しておらず記録として不十分だった
   \item 初対面の方の顔を見て話せなかった
  \end{enumerate}

\section{近況報告}
就職活動の中で,いくつかの企業説明会に参加した.
新しい企業の説明会を聞くたび企業に興味が湧き,心が揺れる.
就職活動では推薦を取ろうと考えているが,なかなか推薦を取る企業を選べない.
推薦希望の締切りは3月31日であり,時間はある.
十分に考え,自分が良いと信じられる結論を得る.

\section{学会情報}
\begin{enumerate}
 \item 10th International Conference on Broadband and Wireless Computing, Communication and Applications(BWCCA2015)\\
       開催期間:平成27年11月4日(水) 〜 11月6日(金)\\
       開催場所:ポーランド クラクフ\\
       原稿締切:平成27年6月26日(金)
 \item 第23回 マルチメディア通信と分散処理ワークショップ(DPS2015)\\
       開催期間:平成27年10月14日(水) 〜 10月16日(金)\\
       開催場所:雲仙温泉 雲仙富貴屋 (長崎県雲仙市小浜町雲仙320)\\
       原稿締切:6月中旬
\end{enumerate}

\bibliographystyle{ipsjunsrt}
\bibliography{mybibdata}

\end{document}
