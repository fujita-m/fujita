\documentclass[fleqn, 14pt]{extarticle}
                    \usepackage{reportForm}
\usepackage[utf8]{inputenc}
\usepackage[T1]{fontenc}
\usepackage{fixltx2e}
\usepackage{graphicx}
\usepackage{longtable}
\usepackage{float}
\usepackage{wrapfig}
\usepackage[normalem]{ulem}
\usepackage{textcomp}
\usepackage{marvosym}
\usepackage{wasysym}
\usepackage{latexsym}
\usepackage{amssymb}
\usepackage{amstext}
\usepackage{hyperref}
\usepackage{comment}
\tolerance=1000
\subtitle{(2014年07月29日$\sim$2014年08月24日)}
\usepackage{strike}
\setcounter{section}{-1}
\author{乃村研究室B4\\藤田 将輝}
\date{2014年08月25日}
\title{記録書 No.7}
\hypersetup{
  pdfkeywords={},
  pdfsubject={},
  pdfcreator={Emacs 24.3.1 (Org mode 8.0.3)}}
\begin{document}

\maketitle




\section{前回ミーティングからの指導・指摘事項}
\label{sec-1}
\begin{enumerate}
\item 特になし
\newline
\hfill

\end{enumerate}




\section{実績}
\label{sec-2}

\subsection{研究関連}
\label{sec-2-1}
\begin{enumerate}
\item 研究テーマに関する項目
\hfill
\label{enum-research1}
\begin{enumerate}

\item 参考文献の読解
\hfill
\label{enum-1-A}
(50%,+0%)
\end{enumerate}
\item 開発に関する項目
\hfill
\label{enum-research2}
\begin{enumerate}

\item 自動ビルドスクリプトの作成
\hfill
\label{enum-2-A}
(90%,+0%)
\end{enumerate}


\end{enumerate}




\subsection{大学・大学院関連}
\label{sec-2-3}

\begin{enumerate}
\item オープンキャンパス
\hfill
\label{enum-univ2}
(08/08,09)
\item 平成27年度岡山大学大学院入学試験試
\hfill
\label{enum-univ3}
(08/21)
\end{enumerate}





\section{詳細および反省・感想}
\label{sec-3}
\subsection{大学関連}
\label{sec-3-1}

\begin{itemize}
\item[(\ref{enum-univ2})]
オープンキャンパスの1日目に参加した.
1日目はNEWグループの紹介であった.
OSとMintの説明の中で計算機をゲーム機に例えて説明していて,高校生にも理解しやすい説明だった.
来年は自分が発表するため,今回の発表の理解しやすい構成を参考にする.
\item[(\ref{enum-univ3})]
平成27年度岡山大学大学院入学試験を受験した.
受験する直前までは自信があったが,
本番になるととても緊張してしまった.
問題文を読んでいくうちに落ち着くことができたため,
持っている力は出しきれた.
\end{itemize}


\section{今後の予定}
\label{sec-4}
\subsection{研究関連}
\label{sec-4-1}

\begin{enumerate}
\item 研究テーマに関する項目
\hfill
\begin{enumerate}


\item 参考文献の読解
\hfill
(09/05)

\end{enumerate}
\item 開発に関する項目
\hfill
\begin{enumerate}

\item 自動ビルドスクリプトの作成
\hfill
(09/05)

\end{enumerate}
\item 第9回Newグループ開発打ち合わせ
\hfill
\label{enum-7}
(08/26)
\item 第259回New打ち合わせ
\hfill
\label{enum-3}
(09/04)
\end{enumerate}

\subsection{研究室関連}
\label{sec-4-2}

\begin{enumerate}

\item 2014年度研修会
\hfill
\label{enum-6}
(09/02,03)

\item 乃村研ミーティング
\hfill
\label{enum-7}
(09/08)


\item 第25回乃村杯
\hfill
\label{enum-8}
(09/08)

\end{enumerate}
\section{その他}
平成27年度岡山大学大学院入学試験を終え,研究を再開した.
一カ月半研究から離れていたため,研究内容を少し忘れてしまっている.
このため研究内容を思い出すところから始める.
これから本格的に研究が始まるため,真剣に勉強をする.


\end{document}
