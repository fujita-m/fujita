\documentclass[fleqn, 14pt]{extarticle}
                    \usepackage{reportForm}
\usepackage[utf8]{inputenc}
\usepackage[T1]{fontenc}
\usepackage{fixltx2e}
\usepackage{graphicx}
\usepackage{longtable}
\usepackage{float}
\usepackage{wrapfig}
\usepackage[normalem]{ulem}
\usepackage{textcomp}
\usepackage{marvosym}
\usepackage{wasysym}
\usepackage{latexsym}
\usepackage{amssymb}
\usepackage{amstext}
\usepackage{hyperref}
\tolerance=1000
\subtitle{(2014年04月11日$\sim$2014年04月27日)}
\usepackage{strike}
\setcounter{section}{-1}
\author{乃村研究室B4\\藤田 将輝}
\date{2014年04月28日}
\title{記録書 No.2}
\hypersetup{
  pdfkeywords={},
  pdfsubject={},
  pdfcreator={Emacs 24.3.1 (Org mode 8.0.3)}}
\begin{document}

\maketitle
\section{前回ミーティングからの指導・指摘事項}
\label{sec-1}
\begin{enumerate}
\item 手順書を作成するときは手順を知らない人が見て,理解できるように注意する.
\newline
\hfill
[4/15, 106,増田さん]
\end{enumerate}
\section{実績}
\label{sec-2}
\subsection{研究関連}
\label{sec-2-1}
\begin{enumerate}
\item 2014年度Newグループ新B4課題に関する項目
\hfill
\label{enum-research1}
\begin{enumerate}

\item Linuxカーネルへのシステムコールの実装
\hfill
\label{enum-1-A}
(100%,+100%)
\item システムコールの実装の手順書作成
\hfill
\label{enum-1-B}
(100%,+100%)
\item Mintの構築
\hfill
\label{enum-1-C}
(100%,+10%)

\end{enumerate}
\end{enumerate}


\subsection{研究室関連}
\label{sec-2-2}

\begin{enumerate}
\item 産学技術交流会(株式会社クレオフーガ様)
\hfill
\label{enum-laboratory1}
(04/16)
\item 第249回Newグループ打ち合わせ
\hfill
\label{enum-laboratory2}
(04/21)
\item 平成26年度B4英語勉強会
\hfill
\label{enum-laboratory3}
(04/25)
\end{enumerate}

\subsection{大学・大学院関連}
\label{sec-2-3}

\begin{enumerate}
\item 健康診断
\hfill
\label{enum-university1}
(04/14)
\end{enumerate}

\section{詳細および反省・感想}
\label{sec-3}
\subsection{研究関連}
\label{sec-3-1}

\begin{itemize}
\item[(\ref{enum-1-A})]
Linuxカーネルへシステムコールを実装した.
システムコールの実装方法を理解した.
\item[(\ref{enum-1-B})]
Linuxカーネルへシステムコールを実装する手順書を作成した.
各手順で何をしているかを理解できるように注意した.
先輩方の丁寧な添削のため,わかりやすい手順書を作成できた.
ご指摘いただいたことを,これからの文章作成にいかす.
\item[(\ref{enum-1-C})]
Mintを構築した.
1つの計算機上で3つのOSを同時に走行させた.
複数のOSを同時起動するには計算機資源を分配する必要がある.
特にデバイスの分割に苦戦した.
\end{itemize}

\subsection{研究室関連}
\label{sec-3-2}

\begin{itemize}
\item[(\ref{enum-laboratory1})]
産学技術交流会に参加した.
社会にサービスを提供している方々の生の声を聞くことができた.
職場も見学させていただき,とてもいい勉強になった.
\item[(\ref{enum-laboratory3})]
平成26年度B4英語勉強会に参加した.
TOEICの本番と同じ形式で問題を解いた.
目標の550点にまだまだ足りないため,まずは単語を覚えてスコアの上昇に努める.
また,試験時間は2時間であるため,2時間英語に取り組む集中力も身につける.
\end{itemize}

\section{今後の予定}
\label{sec-4}
\subsection{研究関連}
\label{sec-4-1}

\begin{enumerate}
\item 研究テーマに関する項目
\hfill
\begin{enumerate}

\item 研究テーマの決定
\hfill
(05/07)

\end{enumerate}
\end{enumerate}

\subsection{研究室関連}
\label{sec-4-2}

\begin{enumerate}
\item 第24回乃村杯
\hfill
\label{enum-3}
(04/28)
\item 第250回Newグループ打ち合わせ
\hfill
\label{enum-4}
(05/07)
\end{enumerate}
\subsection{大学関連}
\begin{enumerate}
\item カレッジTOEIC
\hfill
\label{enum-5}
(05/31)
\end{enumerate}
\section{その他}
\label{sec-5}
カーネル圧縮イメージであるbzImageのサイズを削減している.
カーネルの起動に関係のない機能を削ることで軽量化している.
現在はもとのbzImageのサイズの4852KBから2692KBまで軽量化できている.
さらなる軽量化を目指している.

\end{document}
