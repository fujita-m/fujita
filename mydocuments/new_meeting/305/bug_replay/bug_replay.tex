% Created 2013-12-20 金 04:52
\documentclass[12pt]{jsarticle}
\usepackage[dvipdfmx]{graphicx}
\usepackage{comment}
%\usepackage{setspace}
%%\date{\today}
%\title{}
\textheight = 25truecm
\textwidth = 18truecm
\topmargin = -1.5truecm
\oddsidemargin = -1truecm
\evensidemargin = -1truecm
\marginparwidth = -1truecm
\usepackage{nutils}
\def\theenumii{\Alph{enumii}}
\def\theenumiii{\alph{enumiii}}
\def\labelenumi{(\theenumi)}
\def\labelenumiii{(\theenumiii)}
%\setstretch{0.9}
\begin{document}


%\maketitle
%\tableofcontents

\begin{center}
%%%%%%%%%%%%%%%%%%%%%%%%%%%%%%%%%%%%%%%
%%%タイトル                         %%%
%%%%%%%%%%%%%%%%%%%%%%%%%%%%%%%%%%%%%%%
    {\LARGE 調査したバグの再現について}
\end{center}

\begin{flushright}
    2016/7/13\\
    藤田将輝
\end{flushright}
%%%%%%%%%%%%%%%%%%1章%%%%%%%%%%%%%%%%%%%
\section{はじめに}
\mbox{<new 304-04>}において,RTL8169の過去のバグについてをまとめた.
本資料では,\mbox{<new 304-04>}に挙げた3つのバグを本開発支援環境で
再現する際の設計について述べる.
具体的には,各バグを再現するにあたっての課題,対処,およびバグの確認方法について
述べる.

\section{目的}
本環境を用いてNICを用いず過去に発生したNICドライバのバグを再現する.
これにより,NICを用いた場合よりも
少ない工数でバグを再現できることを示す.

\section{再現するバグ}
再現するバグとその概要を以下に示す.
\begin{enumerate}
    \item チェックサムを確認するタイミングに関するバグ\\
        受信したパケットのチェックサムを確認するタイミングを誤っているため,
        チェックサムの確認が不要なパケットであってもすべてのパケットに対して
        チェックサムを確認してしまうバグである.
    \item 大きなサイズのパケットを受信した場合,マシンをクラッシュさせるバグ\\
        アロケートしたバッファサイズよりも大きなパケットの受信を許してしまうことにより,
        カーネル空間を侵害し,システムをクラッシュさせる可能性があるバグである.
    \item 特定のサイズのパケットを受信できないバグ\\
        パケットのフィルタリング機能とアロケートするバッファサイズの関係に不一致があることにより,
        受信可能であるはずのサイズのパケットまでフィルタにより破棄されてしまうバグである.
\end{enumerate}

\section{各バグを再現する際の開発支援環境の設計}
\subsection{概要}
各バグを再現するには,NICドライバをバグが報告されているバージョンに戻し,このバージョンに対して
開発支援環境を構築する必要がある.
それぞれのバグが発生するバージョンのNICドライバに開発支援環境を構築し,バグを再現する際に
発生する課題,対処,およびバグの確認方法についてを以降で示す.

\subsection{チェックサムを確認するタイミングに関するバグ}
\subsubsection{課題と対処}
本バグが発生するNICドライバのバージョンと開発支援環境を実装したNICドライバのバージョンとでは,
開発支援環境の構築に関して特筆すべき差異が無かった.
このため,現在構築しているNICドライバのバージョンと同様の改変を加えることにより,バグを再現できると考えられる.

\subsubsection{確認方法}
本バグはIP層でチェックサムの確認の必要がないパケットであっても
全てチェックサムを確認してしまうバグである.
本バグの確認方法を以下に示す.
\begin{description}
    \item[(確認方法)]上位層に送信される直前のソケットバッファをキャプチャし,{\tt skb->ip\_summed}が{\tt CHECKSUM\_NONE}
        であることを確認する.これにより,バグが発生していることが分かる.
\end{description}

\subsection{大きなサイズのパケットを受信した場合,マシンをクラッシュさせるバグ}
\subsubsection{課題}\label{kadai}
本バグを再現するにあたっての課題を以下に示す.
\begin{description}
    \item[(課題 1)] 受信バッファのアロケート方法の変更\\
        本バグが発生するNICドライバのバージョンと開発支援環境を構築しているNICドライバのバージョンとでは
        受信バッファのアロケート方法が異なっている.開発支援環境を構築しているNICドライバのバージョンでは,
        受信バッファのサイズは固定である.一方,本バグが発生するバージョンでは,指定したMTUによって受信バッファの
        サイズが変化する.開発支援環境では共有メモリに受信バッファを確保する.この際,開発支援OSでは,
        受信バッファのサイズは固定されたものとしてパケットを配置する.このため,本バグを再現するには,
        NICドライバが確保したバッファのサイズを開発支援OSが知る必要がある.
\end{description}

\subsubsection{対処}\label{taisyo}
課題に対しての対処について以下に示す.なお,課題番号と対処番号は対応している.
\begin{description}
    \item[(対処 1)] 共有メモリを用いたバッファサイズの通知\\
        NICドライバが受信バッファを確保した際,そのサイズを共有メモリを用いて開発支援OSに通知する.
        開発支援OSは通知されたバッファサイズに従って受信バッファのエントリを算出し,パケットを配置する.
        これにより,正しい位置にパケットを配置することができる.
\end{description}

\subsubsection{確認方法}
本バグは,アロケートした受信バッファサイズよりも大きいサイズのパケットを
受信することでシステムがダウンするバグである.
バグの確認方法について以下に示す.
\begin{description}
    \item[(確認方法)] 開発支援OSで,受信バッファよりも大きなパケットサイズを指定し,送信する.
        これにより,システムが停止し,バグが発生することを確認する.
\end{description}

\subsection{特定のサイズのパケットを受信できないバグ}
\subsubsection{課題}
本バグを再現するにあたっての課題を以下に示す.
\begin{description}
    \item[(課題 1)] 受信バッファのアロケート方法の変更\\
        \ref{kadai}項の(課題 1)と同じ.
    \item[(課題 2)] レジスタの動作の再現\\
        本バグは,ハードウェアのパケットフィルタの動作によって発生する.
        パケットフィルタはRxMaxSizeレジスタに受信可能なパケットサイズの最大値を指定することで,
        その値よりも大きなサイズのパケットを破棄する.
        開発支援環境ではNICハードウェアを用いないため,RxMaxSizeレジスタに値を指定しても,
        パケットはフィルタリングされない.
        このため,開発支援OSでパケットフィルタ機能を再現する必要がある.
\end{description}

\subsubsection{対処}
課題に対しての対処について以下に示す.なお,課題番号と対処番号は対応している.
\begin{description}
    \item[(対処 1)] 共有メモリを用いたバッファサイズの通知\\
        \ref{taisyo}項の(対処 1)と同じ.
    \item[(対処 2)] 共有メモリを用いたフィルタ機能の再現\\
        RxMaxSizeレジスタを共有メモリに配置し,開発支援OSと開発対象OSで参照可能にする.
        これにより,開発支援OSでRxMaxSizeの値を確認することができる.
        開発支援OSはパケットを受信バッファに格納する際,RxMaxSizeの値を確認し,
        この値よりも格納しようとしているパケットサイズが大きい場合,パケットを格納せず,破棄する.
\end{description}

\subsubsection{確認方法}
本バグは,指定したMTUと同じサイズのパケットがフィルタ機能によって破棄されてしまうものである.
バグの確認方法について以下に示す.
\begin{description}
    \item[(確認方法)] 
        開発支援OSでMTUと同じサイズのパケットを指定し,動作させた際,NICドライバがパケットを受信しないことを
        確認することで,バグを確認する.
\end{description}

\section{おわりに}
本資料では,各バグの再現についての設計を述べた.
本資料の設計に基づいてバグを再現し,再現にかかったコードの変更量等を
調査する.

\end{document}

%\begin{table}[htbp]
%    \caption{実験環境}
%    \label{kankyou}
%    \begin{center}
%        \begin{tabular}{l|l}   \hline \hline 
%            項目名      & 環境    \\ \hline
%            OS          & Fedora14 x86\_64(Mint 3.0.8)  \\ 
%            CPU         & Intel(R) Core(TM) Core i7-870 @ 2.93GHz \\ 
%            NICドライバ & RTL8169    \\ 
%            ソースコード& r8169.c \\ \hline
%
%        \end{tabular}
%    \end{center}
%\end{table}
%
%\insertfigure[0.8]{fig:frame}{fig1}{パケットの構成}{ipi route}


