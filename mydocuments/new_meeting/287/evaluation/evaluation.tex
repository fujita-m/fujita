% Created 2013-12-20 金 04:52
\documentclass[12pt]{jsarticle}
\usepackage[dvipdfmx]{graphicx}
\usepackage{comment}
%\usepackage{setspace}
%%\date{\today}
%\title{}
\textheight = 25truecm
\textwidth = 18truecm
\topmargin = -1.5truecm
\oddsidemargin = -1truecm
\evensidemargin = -1truecm
\marginparwidth = -1truecm
\usepackage{nutils}
\def\theenumii{\Alph{enumii}}
\def\theenumiii{\alph{enumiii}}
\def\labelenumi{(\theenumi)}
\def\labelenumiii{(\theenumiii)}
%\setstretch{0.9}
\begin{document}


%\maketitle
%\tableofcontents

\begin{center}
%%%%%%%%%%%%%%%%%%%%%%%%%%%%%%%%%%%%%%%
%%%タイトル                         %%%
%%%%%%%%%%%%%%%%%%%%%%%%%%%%%%%%%%%%%%%
    {\LARGE Mintを用いたデバッグ支援環境の評価(修正版)}
\end{center}

\begin{flushright}
    2015/10/13\\
    藤田将輝
\end{flushright}

%%%%%%%%%%%%%%%%%%%%%%%%%%%%%%%%%%%%%
\section{はじめに}
%%%%%%%%%%%%%%%%%%%%%%%%%%%%%%%%%%%%%

本資料は前回の資料で測定方法を誤っていたため,これを再考し,
再度測定を行った結果と考察を前回の資料に加えたものである.
また,本デバッグ支援環境におけるデバッグ支援機構がどの程度の
時間で処理をするかを測定し,これについても追記している.
さらに,実NICで測定を行い,この結果を掲載している.

%%%%%%%%%%%%%%%%%%%%%%%%%%%%%%%%%%%%%
\section{前回との差異}
%%%%%%%%%%%%%%%%%%%%%%%%%%%%%%%%%%%%%

前回の資料との差異を以下に示す.
\begin{enumerate}
    \item パケットサイズによるデバッグ支援機構の処理時間の測定\\
        本デバッグ支援環境ではデバッグ支援機構を動作させることで,デバッグ支援OSから
        デバッグ対象OSへ割り込みを発生させる.パケットのサイズを変更させた際の,
        デバッグ支援機構1処理の時間をそれぞれ測定した.
    \item 測定2の測定方法の再検討\\
        前回の測定2の測定方法はデバッグ支援機構を指定した間隔で連続で動作させ,どの程度
        の間隔とサイズならばNICドライバが正常にパケットを処理できるかを測定していた.
        しかし,この方法では,デバッグ支援機構の処理時間を下回る間隔では測定できない.
        このため,測定方法を再度検討し,測定した.
    \item 実NICでの測定\\
        本評価では,実NICとの比較がなかった.本デバッグ支援環境の有用性を示すため,
        実NICがどの程度の処理速度を実現しているかを測定し,比較した.
\end{enumerate}

%%%%%%%%%%%%%%%%%%%%%%%%%%%%%%%%%%%%%
\section{評価目的}
%%%%%%%%%%%%%%%%%%%%%%%%%%%%%%%%%%%%%

本評価の目的は,Mintを用いたデバッグ支援環境を使用して,デバッグ対象OSに
パケット受信割り込み処理をさせた際に,どの程度の割り込み間隔と
パケットのサイズならば正常にパケットを処理できるかを測定し,
実現できる処理速度を算出することにより,
実際のNICを再現することに十分な性能を実現できているかを評価することである.

%%%%%%%%%%%%%%%%%%%%%%%%%%%%%%%%%%%%%
\section{評価環境}
%%%%%%%%%%%%%%%%%%%%%%%%%%%%%%%%%%%%%

本評価における評価環境を表\ref{environment}に示す.

\begin{table}[h]
    \caption{測定環境}
    \label{environment}
    \begin{center}
        \begin{tabular}{l|l}   \hline \hline 
            項目名      & 環境                                    \\ \hline
            OS          & Fedora14 x86\_64(Mint 3.0.8)            \\ 
            CPU         & Intel(R) Core(TM) Core i7-870 @ 2.93GHz \\
            メモリ      & 2GB                                     \\
            Chipset     & Intel(R) 5 Series/3400                  \\
            NIC         & RTL8111/8168B                           \\
            NICドライバ & RTL8169                                 \\ 
            ソースコード& r8169.c                                 \\ \hline
        \end{tabular}
    \end{center}
\end{table}

%%%%%%%%%%%%%%%%%%%%%%%
\section{測定対象}
%%%%%%%%%%%%%%%%%%%%%%%

本評価で行う測定は,以下の5つである.

\begin{description}
    \item[【測定0】]
        パケットのサイズを変えた際のデバッグ支援機構の処理時間の変化
    \item[【測定1】]
        コアが反応可能なIPIの受信間隔の測定
    \item[【測定2】]
        NICドライバがパケットを処理可能なIPI送信間隔とパケットサイズの測定
    \item[【測定3】]
        デバッグ対象OS上で動作するUDPの受信プログラムがパケットを処理可能な
        IPI送信間隔とパケットサイズの測定
    \item[【測定4】]
        実NICでパケットを受信した際の処理速度の測定
\end{description}

%%%%%%%%%%%%%%%%%%%%%%%
\section{測定0}
%%%%%%%%%%%%%%%%%%%%%%%

%%%%%%%%%%%%%%%%%%%%%%%
\subsection{測定方法}
%%%%%%%%%%%%%%%%%%%%%%%

パケットのサイズを変化させ,デバッグ支援機構を動作させた際の,
処理時間の変化を測定した.
デバッグ支援機構の以下の処理にどの程度の時間がかかるかを測定した.
\begin{enumerate}
    \item パケットをNICの受信バッファに配置する.
    \item NICの受信ディスクリプタを受信状態に更新する.
    \item IPIを送信する.
\end{enumerate}
測定したパケットのサイズを以下に示す.
\begin{enumerate}
    \item 1KB
    \item 1.5KB
    \item 4KB
    \item 8KB
    \item 12KB
    \item 16KB
\end{enumerate}
この中で1.5KBはネットワークインタフェースにおけるデフォルトのMTUのサイズである.

%%%%%%%%%%%%%%%%%%%%%%%
\subsection{測定結果}\label{result-time}
%%%%%%%%%%%%%%%%%%%%%%%

測定結果を表\ref{table:time}に示し,分かったことを以下に示す.
\begin{enumerate}
    \item パケットのサイズが増加するに連れて,処理時間は増大している.
\end{enumerate}

\begin{table}[h]
    \caption{パケットのサイズを変化させた際のデバッグ支援機構の処理時間の変化}
    \label{table:time}
    \begin{center}
        \begin{tabular}{l|l}   \hline \hline 
            サイズ    & 処理時間   \\ \hline
            1KB       & 356ns          \\ 
            1.5KB     & 380ns          \\ 
            4KB       & 697ns          \\ 
            8KB       & 2200ns          \\ 
            12KB      & 3983ns          \\ 
            16KB      & 4797ns          \\ \hline
        \end{tabular}
    \end{center}
\end{table}


%%%%%%%%%%%%%%%%%%%%%%%
\section{測定1}
%%%%%%%%%%%%%%%%%%%%%%%

%%%%%%%%%%%%%%%%%%%%%%%
\subsection{測定方法}
%%%%%%%%%%%%%%%%%%%%%%%

IPIの間隔を調整し,連続でIPIを送信し,割り込みハンドラの動作確率を測定する.
具体的には,デバッグ支援OSからデバッグ対象OSに5000回連続でIPIを送信し,
デバッグ対象OSの割り込みハンドラを動作させた際に,カウンタをインクリメント
することで,動作した割り込みハンドラの回数を測定し,動作確率を算出する.
これらの動作を1サイクルとし,IPIの送信間隔を一定時間増加させながら何サイクルも行う.
これにより,全てのIPIにコアが反応できる送信間隔を調査する.

%%%%%%%%%%%%%%%%%%%%%%%
\subsection{測定結果}\label{result-ipi-limit}
%%%%%%%%%%%%%%%%%%%%%%%
\insertfigure[0.6]{fig:top-ipi}{fig21}{IPIの反応限界}{ipi route}
測定結果を図\ref{fig:top-ipi}に示し,分かったことを以下で説明する.

\begin{enumerate}
    \item IPIの送信間隔が0usから1usの間,割り込みハンドラの動作確率は一次関数
        的に増加する.
    \item IPIの送信間隔を1us以上にすると動作確率は100\%となる.
\end{enumerate}

%%%%%%%%%%%%%%%%%%%%%%%
\section{測定2}
%%%%%%%%%%%%%%%%%%%%%%%

%%%%%%%%%%%%%%%%%%%%%%%
\subsection{測定方法}\label{method-measure}
%%%%%%%%%%%%%%%%%%%%%%%

\insertfigure[0.6]{fig:flow}{fig24}{測定フロー}{ipi route}
本デバッグ支援環境を用いて,連続でパケットを送信した際のNICドライバでパケットを処理できる確率を測定する.
測定に用いたパケットはUDPパケットのEthernetフレームであり,パケットのサイズはEthernetフレームのサイズである.
また,パケットの受信成功とは,NICドライバがパケットをソケットバッファに格納することである.
測定を行う際の具体的な処理フローを図\ref{fig:flow}に示し,以下で説明する.
\begin{description}
    \item[デバッグ支援OS]~\\
        \begin{enumerate}
            \item パケットジェネレータを動作させる.
            \item NICドライバの受信バッファの全てのエントリ(256エントリ)にパケットを配置する.
            \item 受信ディスクリプタを更新する.
            \item IPIを送信する.
            \item 指定した時間だけ処理を待たせる.
            \item 5000回繰り返したかを判定する.
                5000回に満たなければ(3)へ.5000回であれば(7)へ.
            \item 共有メモリのカウンタを取得し,出力する.
            \item (3)〜(7)を1サイクルとして,待たせる時間を増加させながら
                指定したサイクル数になるまで繰り返す.
        \end{enumerate}
    \item[デバッグ対象OS]~\\
        \begin{enumerate}
        \renewcommand{\labelenumi}{(\Alph{enumi})}
            \item IPIを受信すると割り込みハンドラが動作する.
            \item 割り込みハンドラがパケット受信割り込み処理を呼び出し,処理が開始する.
            \item パケットを取得し,ソケットバッファに格納すると,カウンタをインクリメントする.
        \end{enumerate}
\end{description}

(3)〜(6)のループの中で,wait処理を除いた(3),(4)の処理時間は約100nsであった.
このため,IPIの送信間隔は正確であると言える.
これらの操作を以下の6つのサイズのパケットで行うことで,各パケットのサイズにおける受信可能な
動作間隔がわかる.
\begin{enumerate}
    \item 1KB
    \item 1.5KB
    \item 4KB
    \item 8KB
    \item 12KB
    \item 16KB
\end{enumerate}
さらに,1サイクルにかかった時間を測定する.この時間と,処理したデータ量からどの程度の処理速度を実現できているかが
分かる.

%%%%%%%%%%%%%%%%%%%%%%%
\subsection{測定結果}\label{result-driver-limit}
%%%%%%%%%%%%%%%%%%%%%%%

\insertfigure[0.7]{fig:driver-limit}{fig22}{NICドライバにおける受信成功確率}{ipi route}
測定結果を図\ref{fig:driver-limit}に示し,読み取れることを以下に示す.

\begin{enumerate}
    \item 0usから3usの間,どのサイズでもほとんどの割り込みでパケット受信に失敗している.
    \item 3usを超えると,一次関数的に受信成功確率は増加し始める.この際,パケットサイズが小さいほど
        傾きが急になっている.
    \item 受信成功確率が100\%になってからはどれだけ動作間隔を増加させても受信成功確率は100\%となる.
\end{enumerate}

%%%%%%%%%%%%%%%%%%%%%%%
\section{測定3}
%%%%%%%%%%%%%%%%%%%%%%%

%%%%%%%%%%%%%%%%%%%%%%%
\subsection{測定方法}
%%%%%%%%%%%%%%%%%%%%%%%

本デバッグ支援環境を用いて,デバッグ支援OSからデバッグ対象OSへパケットを送信した際の,
デバッグ対象OS上で動作するUDPの受信用プログラムでどの程度の間隔とパケットサイズならば,
パケットを受信できるかを測定する.また,デバッグ支援OSからデバッグ対象OS上のプログラムまでで
実現できている通信量を測定する.
具体的な方法は,\ref{method-measure}節とほぼ同様である.カウンタをインクリメントする箇所が
\ref{method-measure}節ではNICドライバがパケットを受信した時であったのに対し,本測定ではUDPの受信プログラムが
パケットを受信した時であることが差分である.
また,ここでの受信成功とはデバッグ対象OS上で動作するUDPの受信プログラムがパケットを受信することである.

%%%%%%%%%%%%%%%%%%%%%%%
\subsection{測定結果}\label{result-program-limit}
%%%%%%%%%%%%%%%%%%%%%%%

\insertfigure[0.7]{fig:program-limit}{fig23}{UDP受信プログラムにおける受信成功確率}{ipi route}

測定結果を図\ref{fig:program-limit}に示し,読み取れることを以下に示す.
\begin{enumerate}
    \item どのパケットサイズでもある時点まではほとんど受信に失敗している.
    \item ある時点から一次関数的に受信成功確率が増加している.
    \item 受信成功確率が増加し始めるある時点は,パケットサイズが増加するに連れて増大している.
    \item 受信成功確率が一次関数的に増加する際,パケットのサイズに関わらず,傾きは一定である.
    \item 受信成功確率が0回になるIPI送信間隔以上に間隔を増加させても受信成功確率は100\%となる.
\end{enumerate}

%%%%%%%%%%%%%%%%%%%%%%%
\section{測定4}
%%%%%%%%%%%%%%%%%%%%%%%

%%%%%%%%%%%%%%%%%%%%%%%
\subsection{測定方法}
%%%%%%%%%%%%%%%%%%%%%%%

2台の計算機を用いて,一方の計算機から他方の計算機へパケットを送信した際の,
処理速度を測定した.
2台の計算機はLANケーブルで接続されている.
具体的には以下の流れで測定を行った.
\begin{enumerate}
    \item 計算機0から計算機1へ1.5KBのUDPのEthernetフレームを連続で5000回送信する.
        1.5KBとは計算機1のNICのMTUである.
    \item 計算機1がパケットを受信する.この際,計算機1のNICドライバには
        1.5KBのパケットを受信するとカウンタを回すように改変を加えている.
    \item 計算機0で5000回パケットを送信する際に時間を測っており,5000回の送信を
        終えるとかかった時間を出力する.
    \item 5000回の送信を終えると,計算機1でカウンタの値を取得する.
    \item (3)と(4)の値,およびパケットのサイズである1.5KBから処理速度を算出する.
\end{enumerate}
また,同様の流れで,パケットが計算機1上で動作するUDPの受信プログラムまで届きいた際の
処理時間も測定した.

%%%%%%%%%%%%%%%%%%%%%%%
\subsection{測定結果}\label{result-nic}
%%%%%%%%%%%%%%%%%%%%%%%

測定結果を表\ref{table:nic-time}に示し,分かったことを以下に示す.
\begin{enumerate}
    \item NICドライバでの測定結果とプログラムでの測定結果は同じである.
\end{enumerate}

\begin{table}[h]
    \caption{NICドライバにおけるパケットのサイズと割り込み間隔の関係}
    \label{table:nic-time}
    \begin{center}
        \begin{tabular}{l|l|l|l}   \hline \hline 
            測定箇所   & サイズ      & 処理時間      & 処理速度        \\ \hline
            NICドライバ& 1.5KB       & 60378us       & 0.92Gbps        \\ 
            プログラム & 1.5KB       & 60378us       & 0.92Gbps        \\ \hline
        \end{tabular}
    \end{center}
\end{table}

%%%%%%%%%%%%%%%%%%%%%%%
\section{考察}
%%%%%%%%%%%%%%%%%%%%%%%

%%%%%%%%%%%%%%%%%%%%%%%
\subsection{測定0}
%%%%%%%%%%%%%%%%%%%%%%%

測定結果である\ref{result-time}節から,パケットのサイズが増加するに連れ,
デバッグ支援機構の処理時間も増加していると言える.
また,デバッグ支援機構を動作させて連続でパケットを処理させる際,
結果に示した表\ref{table:time}以上の間隔を取る必要があると言える.

%%%%%%%%%%%%%%%%%%%%%%%
\subsection{測定1}
%%%%%%%%%%%%%%%%%%%%%%%

測定結果である\ref{result-ipi-limit}節から,現在の測定環境ではIPIの送信間隔が約1us以上であれば
連続でIPIを送信した際に,全てのIPIに割り込みハンドラが動作できることが分かる.
したがって,本実験環境下では,連続で割り込みを発生させる際,最低でも1us以上の間隔は必要であると言える.

%%%%%%%%%%%%%%%%%%%%%%%
\subsection{測定2}
%%%%%%%%%%%%%%%%%%%%%%%

測定結果である\ref{result-driver-limit}節から,連続で割り込みを発生させた際,
パケットのサイズを増加させるほど全てのパケットを処理することにかかる時間が増加することが分かる.
なぜグラフのような概形になるかは考察できていない.
また,NICドライバで測定した受信成功回数,パケットのサイズ,および1サイクルにかかった時間から
各パケットにおける通信量を算出した.
各パケットサイズにおける初めて
受信成功確率が100\%になった際のIPI送信間隔,1サイクルにかかった時間,
総データ量,通信量,および実NICの測定結果を表\ref{table:driver}に示す.
\begin{table}[h]
    \caption{NICドライバにおけるパケットのサイズと割り込み間隔の関係}
    \label{table:driver}
    \begin{center}
        \begin{tabular}{l|l|l|l|l}   \hline \hline 
            サイズ   & 割り込み間隔      & 1サイクルの時間      & 総データ量   & 処理速度       \\ \hline
            1KB      & 4.3us            & 21689us              & 5000KB       & 1.7Gbps            \\ 
            1.5KB(NIC)& -               & 60378us              & 7324KB       & 0.92Gbps           \\
            1.5KB    & 4.4us            & 22203us              & 7324KB       & 2.5Gbps            \\ 
            4KB      & 4.6us            & 23195us              & 20000KB      & 6.5Gbps            \\ 
            8KB      & 5.5us            & 27788us              & 40000KB      & 10.9Gbps           \\ 
            12KB     & 5.9us            & 29770us              & 60000KB      & 15.3Gbps           \\ 
            16KB     & 6.4us            & 32491us              & 80000KB      & 18.7Gbps           \\ \hline
        \end{tabular}
    \end{center}
\end{table}

%%%%%%%%%%%%%%%%%%%%%%%
\subsection{測定3}
%%%%%%%%%%%%%%%%%%%%%%%

測定結果である\ref{result-program-limit}節から,連続で割り込みを発生させた際,
NICドライバでの結果と比べて,より多くの時間がかかっていることが分かる.
このことから,NICドライバですべてのパケットを処理できていても,
プログラムまでパケットが届くにはより多くの時間がかかることが分かる.
なぜグラフのような概形になるかは考察できていない.
また,UDPの受信プログラムで測定した受信成功回数,パケットのサイズ,および1サイクルにかかった時間から
各パケットサイズにおける通信量を算出した.
各パケットサイズにおける初めて
受信成功確率が100\%になった際のIPI送信間隔,1サイクルにかかった時間,
総データ量,通信量,および実NICの測定結果を表\ref{table:program}に示す.

\begin{table}[h]
    \caption{UDP受信プログラムにおけるパケットのサイズと割り込み間隔の関係}
    \label{table:program}
    \begin{center}
        \begin{tabular}{l|l|l|l|l}   \hline \hline 
            サイズ   & 割り込み間隔      & 1サイクルの時間      & 総データ量   & 処理速度       \\ \hline
            1KB      & 5.5us            & 27677us              & 5000KB       & 1.3Gbps            \\ 
            1.5KB(NIC)& -               & 60378us              & 7324KB       & 0.92Gbps           \\
            1.5KB    & 5.5us            & 27700us              & 7324KB       & 2.0Gbps            \\ 
            4KB      & 6.1us            & 30705us              & 20000KB      & 4.9Gbps            \\ 
            8KB      & 7.7us            & 38649us              & 40000KB      & 7.8Gbps            \\ 
            12KB     & 9.4us            & 47241us              & 60000KB      & 9.6Gbps            \\ 
            16KB     & 11.1us           & 55789us              & 80000KB      & 10.9Gbps           \\ \hline
        \end{tabular}
    \end{center}
\end{table}

%%%%%%%%%%%%%%%%%%%%%%%
\subsection{測定4}
%%%%%%%%%%%%%%%%%%%%%%%

測定結果を表\ref{table:driver},\ref{table:program}に挿入し,比較すると,
本デバッグ支援環境を用いると,実NICよりも非常に高速に処理できていることがわかる.
また,MTUを超えた大きなパケットも問題なく処理できている.

\section{評価}

上記の結果と考察から,NICドライバで最大約18.7Gbps,デバッグ対象OS上で動作するUDP受信プログラムで
最大約10.9Gbpsの処理速度を実現できることが分かった.
また,実機の測定から本実験環境では,処理速度が約1Gbpsであることがわかる.
また,PCI Express 1.1 の1レーンの転送量が2.5Gbpsであることから,実際のNICが実現できる最大の処理速度は
2.5Gbpsであると言える.
本デバッグ支援環境で実現できる処理速度と実際のNICが実現できる処理速度を比較すると,
本デバッグ支援環境は実際のNICの処理速度を大きく超えた処理速度を実現できていることがわかる.
このことから,本デバッグ支援環境を用いることで現在は実現できていない高処理速度の
NICのエミュレートができると考えられる.
高処理速度のNICやバスが開発されたと想定し,これらに対応するドライバを開発できる.

\section{おわりに}

本資料では,再考した測定方法とその結果,デバッグ支援機構の処理時間,
および実NICでの測定結果について記述した.
測定結果から本デバッグ支援環境は実NICに比べ,非常に高速な処理速度を実現できてると言える.

\end{document}

%\begin{table}[htbp]
%    \caption{実験環境}
%    \label{kankyou}
%    \begin{center}
%        \begin{tabular}{l|l}   \hline \hline 
%            項目名      & 環境    \\ \hline
%            OS          & Fedora14 x86\_64(Mint 3.0.8)  \\ 
%            CPU         & Intel(R) Core(TM) Core i7-870 @ 2.93GHz \\ 
%            NICドライバ & RTL8169    \\ 
%            ソースコード& r8169.c \\ \hline
%
%        \end{tabular}
%    \end{center}
%\end{table}
%
%\insertfigure[0.8]{fig:frame}{fig1}{パケットの構成}{ipi route}


