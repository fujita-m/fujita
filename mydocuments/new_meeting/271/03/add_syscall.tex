% Created 2013-12-20 金 04:52
\documentclass[12pt]{jsarticle}
\usepackage[dvipdfmx]{graphicx}
\usepackage{comment}
%\usepackage{setspace}
%%\date{\today}
%\title{}
\textheight = 25truecm
\textwidth = 18truecm
\topmargin = -1.5truecm
\oddsidemargin = -1truecm
\evensidemargin = -1truecm
\marginparwidth = -1truecm
\def\theenumii{\Alph{enumii}}
\def\theenumiii{\alph{enumiii}}
\def\labelenumi{(\theenumi)}
\def\labelenumiii{(\theenumiii)}
%\setstretch{0.9}
\begin{document}

%\maketitle
%\tableofcontents

\begin{center}
%%%%%%%%%%%%%%%%%%%%%%%%%%%%%%%%%%%%%%%
%%%タイトル                         %%%
%%%%%%%%%%%%%%%%%%%%%%%%%%%%%%%%%%%%%%%
{\LARGE システムコールの追加方法について}
\end{center}

\begin{flushright}
  2015/3/30\\
  藤田将輝
\end{flushright}
%%%%%%%%%%%%%%%%%%1章%%%%%%%%%%%%%%%%%%%
\section{はじめに}
本資料では,新B4課題の「(課題3)IPIを送信するシステムコールの実装と
実装手順書の作成」におけるシステムコールの追加手順を示すため,
Linux3.17におけるシステムコールの追加手順について
記述する.
\section{追加手順}\label{interrupt_handler}
\subsection{システムコール本体を用意}
システムコールの本体を arch/x86/kernel 以下に配置する.
システムコール本体のソースコードの例を以下に示す.
\subsection{システムコールテーブルへの追加}
新たに作成するシステムコールの名前をarch/x86/syscalls/syscall\_64.tbl
に追加する.
arch/x86/syscalls/syscall\_64.tblを以下のように変更する.
\begin{verbatim}
 310     64      process_vm_readv        sys_process_vm_readv
 311     64      process_vm_writev       sys_process_vm_writev
 312     common  kcmp                    sys_kcmp
 313     common  finit_module            sys_finit_module
+314     common  sys_my_syscall          sys_my_syscall
\end{verbatim}
\subsection{プロタイプ宣言を追加}
新たに作成するシステムコールのプロトタイプ宣言を追加する.
include/linux/syscalls.hを以下のように変更する.
\begin{verbatim}
 asmlinkage long sys_kcmp(pid_t pid1, pid_t pid2, int type,
                          unsigned long idx1, unsigned long idx2);
 asmlinkage long sys_finit_module(int fd, const char __user *uargs, int flags);
 
+asmlinkage long sys_my_syscall( char __user *buf, int count );
\end{verbatim}
\subsection{Makefileの編集}
Makefileを編集し,新たに作成するシステムコールをmake対象にする.
arch/x86/kernelを以下のように変更する.
\begin{verbatim}
 obj-$(CONFIG_PMC_ATOM)       += pmc_atom.o
+obj-y                        += my_syscall.o 
\end{verbatim}
\subsection{カーネルのビルド}
資料$<$271-02$>$を参考にカーネルのビルドを行う.
\section{動作確認}
\subsection{追加したシステムコールを呼び出すプログラムの作成}
新たに追加したシステムコールを呼び出すプログラムを任意のディレクトリに
作成する.
以下に,ソースコードの例を示す.
\begin{verbatim}
#include<stdio.h>
#include<unistd.h>
#include<sys/syscall.h>

int main(void)
{
    syscall(314);
    return 0;
}
\end{verbatim}
\subsection{プログラムの実行}
作成したプログラムをコンパイルし,実行する.

\section{おわりに}
本資料では,Linux3.17におけるシステムコールの追加手順について述べた.
\end{document}


