% Created 2013-12-20 金 04:52
\documentclass[12pt]{jsarticle}
\usepackage[dvipdfmx]{graphicx}
\usepackage{comment}
%\usepackage{setspace}
%%\date{\today}
%\title{}
\textheight = 25truecm
\textwidth = 18truecm
\topmargin = -1.5truecm
\oddsidemargin = -1truecm
\evensidemargin = -1truecm
\marginparwidth = -1truecm
\usepackage{nutils}
\def\theenumii{\Alph{enumii}}
\def\theenumiii{\alph{enumiii}}
\def\labelenumi{(\theenumi)}
\def\labelenumiii{(\theenumiii)}
%\setstretch{0.9}
\begin{document}


%\maketitle
%\tableofcontents

\begin{center}
%%%%%%%%%%%%%%%%%%%%%%%%%%%%%%%%%%%%%%%
%%%タイトル                         %%%
%%%%%%%%%%%%%%%%%%%%%%%%%%%%%%%%%%%%%%%
    {\LARGE 受信可能な最大パケットサイズ}
\end{center}

\begin{flushright}
  2015/8/3\\
  藤田将輝
\end{flushright}
%%%%%%%%%%%%%%%%%%1章%%%%%%%%%%%%%%%%%%%
\section{はじめに}
本資料では,本デバッグ支援環境において処理できる最大のパケットサイズの
調査について記述している.
受信ディスクリプタにパケットのサイズを表す部分があり,
これの最大値が16383である.
このため,最大で16383バイトを処理できると考え,実験を行った.
実験を行った結果,FCSの4バイトを除いた,16379バイトを処理できることがわかった.
\section{パケットサイズの算出}
NICドライバでパケットを受信した際,
パケット受信処理の関数内で,パケットのサイズを算出する.
パケットサイズは,受信ディスクリプタを参照することで算出する.
受信ディスクリプタ内にはパケットのサイズを保持する部分があり,
これを参照することで,パケットを算出している.
受信ディスクリプタのパケットサイズを表す部分は,14ビット割り当てられており,
これによってパケットのサイズを表している.
したがって,受信ディスクリプタが表せるパケットサイズの範囲は0〜16383バイトであると
考えられる.
このため,16383バイトのパケットを作成し,このパケットが正常に処理できるかどうかの
実験を行った.
また,NICドライバのソースコード内のパケットサイズを求める部分で,受信ディスクリプタ
が表すサイズから4を引いた値をパケットのサイズとして算出している.
これは,EthernetFrameのFCSを処理の対象外としているためだと考えられる.
したがって,実際に表せる最大のパケットサイズは16379バイトである.
FCSについての詳細は調査できていない.
\section{実験}
\subsection{実験環境}
本実験における実験環境を表1に示す.

\begin{table}[htbp]
\caption{実験環境}
\label{kankyou}
\begin{center}
\begin{tabular}{l|l}   \hline \hline 
項目名      & 環境    \\ \hline
OS          & Fedora14 x86\_64(Mint 3.0.8)  \\ 
CPU         & Intel(R) Core(TM) Core i7-870 @ 2.93GHz \\ 
NICドライバ & RTL8169    \\ 
ソースコード& r8169.c \\ \hline
           
\end{tabular}
\end{center}
\end{table}

\subsection{実験手順}
本実験では,表現可能な最大のサイズのパケットを
作成し,これをNICドライバで処理させ,正常に処理できるかを検証する.
本実験の実験手順を以下で説明する.
\begin{enumerate}
    \item パケットジェネレータにより,パケットを作成する.
        受信ディスクリプタで表現可能な最大のサイズのパケットを作成する.
    \item パケットを共有メモリに配置し,IPIを送信する.
    \item NICドライバがパケット受信処理を行う.
        デバッグ対象OS上で動作する,受信したUDPのpayloadを標準出力に
        出力するプログラムによってpayloadが表示されることを確認する.
\end{enumerate}

\subsection{作成したパケット}
作成したパケットを図\ref{fig:frame}に示し,説明する.
作成したパケットはEthernetFrameであるため,Ethernetヘッダ,IPv4ヘッダ,
およびUDPヘッダを含んでいる.
それぞれのサイズは,12バイト,20バイト,および8バイトである.
これらのヘッダのサイズは不変であるため,payloadを任意のサイズに設定することで,
パケットのサイズを変更する.
今回は,受信ディスクリプタで表現できる最大のサイズのパケットを作成するため,
payloadのサイズを表現できる最大のサイズである16383バイトからヘッダの42バイトを引いた16337バイトとしている.
また,payloadに16337個の「0」を配置することで実現している.

\insertfigure[0.8]{fig:frame}{fig1}{パケットの構成}{ipi route}

\subsection{実験結果}
実験を行った結果,デバッグ対象OS上で動作するプログラムが標準出力に
16337個の「0」を出力していたため,正常に処理できていると判断した.
また,最大よりも大きなパケットを処理させた際は,
計算機の電源が落ちたため,正常に処理できていないと判断した.
これらから,本デバッグ支援環境では1パケットで最大16379バイトのパケットを
処理できることを確認した.

\section{おわりに}
本資料では,本デバッグ環境で処理できるパケットの最大のサイズを検証した.
この結果,最大で16379バイトのパケットを処理できることを確認した.
今後は,連続でパケットを送信した際に,どの程度の速度やサイズであれば正常に処理できるか
を測定する.
\end{document}


