% Created 2013-12-20 金 04:52
\documentclass[12pt]{jsarticle}
\usepackage[dvipdfmx]{graphicx}
\usepackage{comment}
%\usepackage{setspace}
%%\date{\today}
%\title{}
\textheight = 25truecm
\textwidth = 18truecm
\topmargin = -1.5truecm
\oddsidemargin = -1truecm
\evensidemargin = -1truecm
\marginparwidth = -1truecm
\usepackage{nutils}
\def\theenumii{\Alph{enumii}}
\def\theenumiii{\alph{enumiii}}
\def\labelenumi{(\theenumi)}
\def\labelenumiii{(\theenumiii)}
%\setstretch{0.9}
\begin{document}


%\maketitle
%\tableofcontents

\begin{center}
%%%%%%%%%%%%%%%%%%%%%%%%%%%%%%%%%%%%%%%
%%%タイトル                         %%%
%%%%%%%%%%%%%%%%%%%%%%%%%%%%%%%%%%%%%%%
    {\LARGE NICドライバの追加・変更点}
\end{center}

\begin{flushright}
  2015/8/18\\
  藤田将輝
\end{flushright}
%%%%%%%%%%%%%%%%%%1章%%%%%%%%%%%%%%%%%%%
\section{はじめに}
本資料では,NICドライバの追加・変更点をまとめ,記述する.
\section{追加した関数}
本デバッグ支援環境において,NICドライバに追加した関数を以下に示す.
\begin{enumerate}
    \item fujita\_ipi\_irq()\\
        割り込みハンドラ(rtl8169\_interrupt\_fujita())を割り込みハンドラとして
        登録し,ベクタ番号100番にIRQ番号を登録する.
    \item rtl8169\_interrupt\_fujita()\\
        本デバッグ支援環境における割り込みハンドラである.
        NAPIによって登録されたポーリング関数(rtl8169\_poll())を呼び出す.
        これにより,パケット受信割り込み処理を行う.
\end{enumerate}
\section{変更した関数}
本デバッグ支援環境において,既存の関数に変更を加えた.
変更した関数名と,内容を以下に示す.
\begin{enumerate}
    \item rtl8169\_open()\\
        ネットワークインターフェースを起動させた際に実行され,受信バッファや受信ディスクリプタの初期化を行う関数である.
        変更点は以下の3点である.
        \begin{enumerate}
            \item 受信ディスクリプタの配列の先頭アドレスを共有メモリに変更.
            \item fujita\_ipi\_irq()を呼び出し,rtl8169\_interrupt\_fujita()を登録.
            \item この関数の最後に,外部と通信を行わないようにするため,rtl8169\_asic\_down()を呼び出し,通信を停止する.
        \end{enumerate}
    \item rtl8169\_alloc\_rx\_data()\\
        受信バッファのアドレスを決定する関数である.
        変更点は以下の1点である.
        \begin{enumerate}
            \item 受信バッファのアドレスであるmappingの値を,Mintの共有メモリのアドレスに変更.
        \end{enumerate}
    \item rtl8169\_rx\_interrupt()\\
        ポーリング関数(rtl8169\_poll())に呼ばれ,パケット受信割り込み処理を行う関数である.
        変更点は以下の1点である.
        \begin{enumerate}
            \item パケットのサイズを算出する際の値を変更した.
                変更前は,受信バッファの半分の8KBまでしかサイズとして算出できなかったため,
                受信バッファのサイズである16KBまでを算出できるよう変更した.
        \end{enumerate}
\end{enumerate}
\section{おわりに}
本資料では,本デバッグ支援環境におけるNICドライバの追加・変更点について記述した.
\end{document}


