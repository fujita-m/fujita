% Created 2013-12-20 金 04:52
\documentclass[12pt]{jsarticle}
\usepackage[dvipdfmx]{graphicx}
\usepackage{comment}
%\usepackage{setspace}
%%\date{\today}
%\title{}
\textheight = 25truecm
\textwidth = 18truecm
\topmargin = -1.5truecm
\oddsidemargin = -1truecm
\evensidemargin = -1truecm
\marginparwidth = -1truecm
\usepackage{nutils}
\def\theenumii{\Alph{enumii}}
\def\theenumiii{\alph{enumiii}}
\def\labelenumi{(\theenumi)}
\def\labelenumiii{(\theenumiii)}
%\setstretch{0.9}
\begin{document}

%\maketitle
%\tableofcontents

\begin{center}
%%%%%%%%%%%%%%%%%%%%%%%%%%%%%%%%%%%%%%%
%%%タイトル                         %%%
%%%%%%%%%%%%%%%%%%%%%%%%%%%%%%%%%%%%%%%
    {\LARGE パケット受信処理の再現}
\end{center}

\begin{flushright}
  2015/5/14\\
  藤田将輝
\end{flushright}
%%%%%%%%%%%%%%%%%%1章%%%%%%%%%%%%%%%%%%%
\section{はじめに}
本デバッグ支援環境において,擬似パケットを正常に処理できることを示すため,
本資料では,変更の加えられていない
NICドライバで破棄されずに上位層に送られるパケットを複製し,Mintを用いてこれを処理させることで
パケット受信処理を再現したことを記述する.
まず,2台の計算機を用いて,一方の計算機から他方の計算機へUDPパケットを送信する.
この際,受信側のNICドライバ中で受信したパケットをキャプチャする.
キャプチャしたパケットの内容を複写し,パケットを作成する.
ここで,Mintを用いて2つのOSを動作させる.一方のOSは作成したパケットをMintの共有メモリ
に配置し,IPIを他方のOSが占有するコアへ送信する.他方のOSが占有するコアがIPIを受信すると
配置したパケットに対し,パケット受信割り込み処理を行う.これらの処理により
パケット受信処理の再現を行う.また,本資料中のパケット受信処理とは
NICドライバがパケットをソケットバッファに格納し,アプリケーション層まで送る処理とする.


\section{パケットのキャプチャ}\label{cpt}
本実験では正常に処理されるパケットを複製し,これをIPIを契機としてNICドライバに処理させることで
パケット受信処理を再現する.本章ではパケットのキャプチャ方法について説明する.
一方の計算機から他方の計算機へ「Recieved Message」というメッセージを含んだUDPパケットを
送信する.受信側の計算機で受信したメッセージを標準出力に出力するプログラムを動作させ,
「Recieved Massage」と表示されることを確認した.この際,受信側計算機上で動作するNICドライバ
に,送信側計算機のMACアドレスからのパケットであれば,このパケットを指定したアドレスが指すメモリに
配置する操作を追加した.配置されたパケットの内容をカーネルのメッセージバッファに格納し,
内容を確認した.確認した内容からパケットを作成した.また,NICドライバはパケットの受信処理を
行う際,受信ディスクリプタという受信したパケットと受信バッファの情報を保持する構造体を参照し,
この情報からパケットを処理する.
パケットの受信処理を再現するには,この受信ディスクリプタの内容も必要である.
このため,受信ディスクリプタの内容も複写した.


\section{パケット受信処理再現の構成}\label{kankyou}
本章では,パケットの受信処理を再現する際の構成について説明する.
Mintを用いて2つのOSを動作させる.NICの動作を再現するOSと,パケット受信処理を再現するOSである.
これらのOS,複写したパケット,および複写した受信ディスクリプタを用いてパケット受信処理を
再現する.
NICの動作を再現するOSは以下の機能を持つ.
なお,以下の機能は全て1つのシステムコールによって実現している.
\begin{enumerate}
    \item パケットをMintの共有メモリに配置する機能
    \item 受信ディスクリプタを共有メモリから取得し,更新する機能
    \item 自身のもつコアから指定したコアへIPIを送信する機能
    \item 指定した間隔で指定した回数以上の機能を実行する機能
\end{enumerate}
パケット受信処理を再現するOSはNICを保持し,以下の機能を持つようにNICドライバを改変している.
\begin{enumerate}
    \item NICドライバの初期化処理中でMintの共有メモリをNICの受信バッファとする機能
    \item 受信ディスクリプタが示す受信バッファのアドレスをMintの共有メモリとする機能
    \item 受信ディスクリプタをMintの共有メモリに配置する機能
    \item 一方のOSから送信されたIPIによってパケット受信割り込み処理が動作する機能
\end{enumerate}

\section{パケット受信処理の再現}
\ref{kankyou}章で示した環境を用いて,パケット受信処理を再現する.
本章では,パケット受信処理の再現の流れと結果について説明する.
パケット受信処理の再現の流れを以下に示す.
\begin{enumerate}
    \item NICの動作を再現するOSは,複写したパケットをMintの共有メモリに配置する.
    \item NICの動作を再現するOSは,Mintの共有メモリから受信ディスクリプタを取得し,
        キャプチャした受信ディスクリプタの内容に更新する.
    \item NICの動作を再現するOSは,自身の占有するコアへIPIの送信要求を発行する.
    \item 要求を受けたコアは指定されたコアへIPIを送信する.
    \item 指定されたコアがIPIを受信すると,割り込みハンドラとして,
        NICドライバのパケット受信割り込み処理が動作する.
    \item (1)で配置したパケットが処理され,ソケットバッファに格納される.
    \item ソケットバッファが上位層に送られる.
\end{enumerate}
以上の操作を行い,\ref{cpt}章で示したものと同じプログラムを動作させると,
「Recieved Message」と出力された.これにより,パケット受信処理を
再現できたことを確認した.

\section{今後の課題}\label{kadai}
今後の課題として,以下の課題が考えられる.
\begin{enumerate}
    \item 任意のサイズと内容のパケット作成
    \item 短い間隔かつ連続でパケットを処理させた際に正常に処理できる間隔の限界の調査
    \item 実際のNICを用いた際と本環境を用いた際の最大通信量の比較
\end{enumerate}

\section{おわりに}
本資料では,正常に処理されるパケットを複製し,本環境で処理させることで,
パケットの受信処理を再現できることを示した.
これにより,本環境においてNICドライバのパケット受信割り込み処理を再現できることを示した.
今後は,\ref{kadai}章の課題について取り組む.
\end{document}


