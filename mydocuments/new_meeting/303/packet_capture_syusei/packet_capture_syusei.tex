% Created 2013-12-20 金 04:52
\documentclass[12pt]{jsarticle}
\usepackage[dvipdfmx]{graphicx}
\usepackage{comment}
%\usepackage{setspace}
%%\date{\today}
%\title{}
\textheight = 25truecm
\textwidth = 18truecm
\topmargin = -1.5truecm
\oddsidemargin = -1truecm
\evensidemargin = -1truecm
\marginparwidth = -1truecm
\usepackage{nutils}
\def\theenumii{\Alph{enumii}}
\def\theenumiii{\alph{enumiii}}
\def\labelenumi{(\theenumi)}
\def\labelenumiii{(\theenumiii)}
%\setstretch{0.9}
\begin{document}


%\maketitle
%\tableofcontents

\begin{center}
%%%%%%%%%%%%%%%%%%%%%%%%%%%%%%%%%%%%%%%
%%%タイトル                         %%%
%%%%%%%%%%%%%%%%%%%%%%%%%%%%%%%%%%%%%%%
    {\LARGE NICドライバ改変によるパケットキャプチャ方法(修正版)}
\end{center}

\begin{flushright}
    2016/6/15\\
    藤田将輝
\end{flushright}
%%%%%%%%%%%%%%%%%%1章%%%%%%%%%%%%%%%%%%%
\section{はじめに}
開発支援環境でパケットを擬似するにあたり,NICがメモリに配置したパケットをキャプチャし,
その構成を確認した.
本資料では,NICドライバを改変することによってパケットをキャプチャし,
パケットの内容を確認する方法について示す.
また,修正前の資料では,本資料の目的と図が誤っていたため,これを修正した.
さらに,実装したシステムコールを呼び出すユーザプログラムを示していなかったため,
システムコールのソースコードにコメントとして示し,これをGitHubにpushした.

\section{修正箇所}
本資料における前回資料からの修正点について以下に示し,説明する.
\begin{enumerate}
    \item 3章目的
        \begin{description}
            \item[(変更前)]開発支援環境を前提とした目的として書かれており,
                何故パケットキャプチャ機能が必要かが不明確であった.
            \item[(変更後)]ソケットバッファに格納される直前のパケットの構造を
                確認する目的であることを明確にした.
        \end{description}
    \item 5章パケットキャプチャ機構の処理流れ
        \begin{description}
            \item[(変更前)]通常のパケットの流れがわからず,加えた機能がどのように働いているかが
                不明確であった.
            \item[(変更後)]通常のパケットの流れを図で示し,加えた改変と元の流れを
                明確に分けて記述した.
        \end{description}
    \item 7章動作
        \begin{description}
            \item[(変更前)]実装したシステムコールについては,所在を示していたが,
                それを呼ぶためのユーザプログラムの所在が
                不明確であった.
            \item[(変更後)]ユーザプログラムについてはシステムコールのコメントとして示し,
                その所在を示した.
        \end{description}
\end{enumerate}

\section{目的}
ソケットバッファに格納する直前のパケットの状態を確認するため,
NICドライバを改変し,ドライバの層でパケットをキャプチャする機構を
作成する.

\section{実装環境}
パケットキャプチャ機構の実装環境を表\ref{kankyou}に示す.

\begin{table}[htbp]
    \caption{実験環境}
    \label{kankyou}
    \begin{center}
        \begin{tabular}{l|l}   \hline \hline 
            項目名      & 環境    \\ \hline
            OS          & Fedora14 x86\_64(Linux 3.0.8)  \\ 
            NICドライバ & RTL8169    \\ 
            ソースコード& r8169.c \\ \hline
        \end{tabular}
    \end{center}
\end{table}

\section{パケットキャプチャ機構の処理流れ}\label{syorinagare}
\insertfigure[0.5]{fig:packet-capture}{fig1}{パケットキャプチャの処理流れ}{ipi route}
NICドライバを改変し,あらかじめ確保した配列に最新10個のパケットを格納するものとする.
NICがパケットを受信してからパケットがAPに届くまでの処理流れとパケットキャプチャ機能が
パケットをキャプチャし,その内容をAPに複写するまでの処理流れについて図\ref{fig:packet-capture}に示し,以下で説明する.
\begin{enumerate}
    \item NICがパケットを受信し,パケットを受信バッファに格納する.
    \item NICドライバが動作し,受信バッファに格納されているパケットを
        ソケットバッファに複写する.
    \item ソケットバッファがプロトコルタックにより処理される.
    \item プロトコルスタックにより,各種ヘッダが外れたパケットがAPにより受信される.
\end{enumerate}

パケットキャプチャ機能は以下のように動作する.
\begin{enumerate}
\renewcommand{\labelenumi}{(\Alph{enumi})}
    \item カーネルが確保した配列に(2)と同じパケットを格納する.
    \item システムコールにより,ユーザ空間にパケットが複写され,内容を確認できる.
\end{enumerate}

\section{実装}
\ref{syorinagare}章で示したパケットキャプチャを実装するにあたり,以下の機能を実装した.
ソースコードは\mbox{GitHub}における,\mbox{fujita-m/Linux-3.0.8}の\mbox{packet\_capture}ブランチの
17326f3bbe35ad1e6a72286637d\newline d2df90d19f975 (コミットID)にある.
なお,システムコールの追加方法については「<new 249-06>Linuxカーネルへのシステムコール実装手順書」を
参考にすると良い.
\begin{enumerate}
    \item カーネルに配列を確保\\
        NICが受信したパケットを保存しておくため,カーネルに配列を確保する.
        今回は,受信した最新の10個のパケットを確保することとする.
        また,1つのパケットサイズの上限はMTUである1500Bとしている.
        この機能はarch/x86/kernel/sys\_packet\_capture.cに記述している.
    \item 確保した配列にパケットを格納\\
        NICドライバがパケットをソケットバッファに複写するタイミングで
        同じパケットを確保した配列に格納する.
        この機能はdrivers/net/r8169.cに記述している.
    \item システムコールによりパケットをユーザ空間に複写\\
        配列へのポインタを受け取り,保存しているパケットを複写する
        システムコールを実装する.
        これにより,受信した最新10個のパケットをユーザ空間で確認できる.
        この機能はarch/x86/kernel/sys\_packet\_capture.cに記述している.
\end{enumerate}

\section{動作}
実装したシステムコールを呼び出すアプリケーションを作成し,
実装したパケットキャプチャ機能を動作させた.
このアプリケーションはGitHubのfujita/Linux-3.0.8のおけるコミット
(4a46e1ba0766dad60be2be4fd85fbf91b01abfc9)内の
arch/x86/kernel/sys\_packet\_capture.cの先頭にコメントとして示している.
これをコンパイルし,実行すると以下の結果が得られた.
なお,キャプチャしたパケットの先頭から34Bまでを表示している.
\begin{verbatim}
--------------------------------------------------------------------
1  ----start----
2  20cf303e497a8851fb677c8608004500005c7f7c40008006c17fac1530e0ac153095
3  ----start----
4  20cf303e497a8851fb677c860800450000287f7940008006c1b6ac1530e0ac153095
5  ----start----
6  ffffffffffff480fcf3844a008060001080006040001480fcf3844a0ac1530c30000
7  ----start----
8  20cf303e497a8851fb677c8608004500005c7f7a40008006c181ac1530e0ac153095
9  ----start----
10 ffffffffffff8851fb6ece1f080600010800060400018851fb6ece1fac1530100000
11 ----start----
12 20cf303e497a8851fb677c860800450000287f7b40008006c1b4ac1530e0ac153095
13 ----start----
14 ffffffffffff8851fb6ecdd1080600010800060400018851fb6ecdd1ac15304c0000
15 ----start----
16 ffffffffffff8851fb677c78080600010800060400018851fb677c78ac1530e70000
17 ----start----
18 ffffffffffff74d4351405920806000108000604000174d435140592962e01050000
19 ----start----
20 ffffffffffff8851fb6ecdd308004500004e29e30000801187eaac1530a7ac15ffff
--------------------------------------------------------------------
\end{verbatim}
上記の結果を得た際,SSHを用いて実験用計算機と文書作成用計算機とで
通信を行なっていた.パケットのEtherヘッダには,宛先MACアドレスと
差出元MACアドレスが含まれており,キャプチャしたパケットのEtherヘッダには
実験用計算機と文書作成用計算機のMACアドレスが含まれていた(2,4,8,12行目の先頭から12B).
また,IPヘッダには宛先IPアドレスと差出元IPアドレスが含まれており,
キャプチャしたパケットのIPヘッダには実験用計算機と文書用計算機に
割り当てたIPアドレスが含まれていた(2,4,8,12行目の末尾から8B).
これらにより,正しくパケットをキャプチャできていると言える.

\section{おわりに}
本資料では,NICを改変することにより,パケットをキャプチャする方法について
示した.これを用いることで,パケットの構成を知ることができる.

\end{document}

%\begin{table}[htbp]
%    \caption{実験環境}
%    \label{kankyou}
%    \begin{center}
%        \begin{tabular}{l|l}   \hline \hline 
%            項目名      & 環境    \\ \hline
%            OS          & Fedora14 x86\_64(Mint 3.0.8)  \\ 
%            CPU         & Intel(R) Core(TM) Core i7-870 @ 2.93GHz \\ 
%            NICドライバ & RTL8169    \\ 
%            ソースコード& r8169.c \\ \hline
%
%        \end{tabular}
%    \end{center}
%\end{table}
%
%\insertfigure[0.8]{fig:frame}{fig1}{パケットの構成}{ipi route}


