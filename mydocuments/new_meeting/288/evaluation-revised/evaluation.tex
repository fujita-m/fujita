% Created 2013-12-20 金 04:52
\documentclass[12pt]{jsarticle}
\usepackage[dvipdfmx]{graphicx}
\usepackage{comment}
%\usepackage{setspace}
%%\date{\today}
%\title{}
\textheight = 25truecm
\textwidth = 18truecm
\topmargin = -1.5truecm
\oddsidemargin = -1truecm
\evensidemargin = -1truecm
\marginparwidth = -1truecm
\usepackage{nutils}
\def\theenumii{\Alph{enumii}}
\def\theenumiii{\alph{enumiii}}
\def\labelenumi{(\theenumi)}
\def\labelenumiii{(\theenumiii)}
%\setstretch{0.9}
\begin{document}


%\maketitle
%\tableofcontents

\begin{center}
%%%%%%%%%%%%%%%%%%%%%%%%%%%%%%%%%%%%%%%
%%%タイトル                         %%%
%%%%%%%%%%%%%%%%%%%%%%%%%%%%%%%%%%%%%%%
    {\LARGE デバッグ支援機構の処理流れと評価}
\end{center}

\begin{flushright}
    2015/10/28\\
    藤田将輝
\end{flushright}
%%%%%%%%%%%%%%%%%%%%%%%%%%%%%%%%%%%%%
\section{はじめに}
%%%%%%%%%%%%%%%%%%%%%%%%%%%%%%%%%%%%%

Mintを用いたデバッグ支援環境を評価する際に,各処理流れを明確にし,これらにかかる時間を
測定した.測定した各処理の時間を用いて,パケットの成功率を求める測定を行った.
各処理時間とパケットの成功率から本デバッグ支援環境について考察した.
本資料では,デバッグ支援環境における各処理とこれにかかる時間,どの程度のパケットサイズと
パケットの間隔ならばパケットの受信に成功するかを求め,これらについて考察したことを示す.

%%%%%%%%%%%%%%%%%%%%%%%%%%%%%%%%%%%%%
\section{測定環境}
%%%%%%%%%%%%%%%%%%%%%%%%%%%%%%%%%%%%%

測定を行った際の環境を表\ref{environment}に示す.

\begin{table}[h]
    \caption{測定環境}
    \label{environment}
    \begin{center}
        \begin{tabular}{l|l}   \hline \hline 
            項目名      & 環境                                    \\ \hline
            OS          & Fedora14 x86\_64(Mint 3.0.8)            \\ 
            CPU         & Intel(R) Core(TM) Core i7-870 @ 2.93GHz \\
            メモリ      & 2GB                                     \\
            Chipset     & Intel(R) 5 Series/3400                  \\
            NIC         & RTL8111/8168B                           \\
            NICドライバ & RTL8169                                 \\ 
            ソースコード& r8169.c                                 \\ \hline
        \end{tabular}
    \end{center}
\end{table}

%%%%%%%%%%%%%%%%%%%%%%%%%%%%%%%%%%%%%
\section{デバッグ支援環境の処理流れ}
%%%%%%%%%%%%%%%%%%%%%%%%%%%%%%%%%%%%%

%%%%%%%%%%%%%%%%%%%%%%%%%%
\subsection{目的}
%%%%%%%%%%%%%%%%%%%%%%%%%%

本デバッグ支援環境を用いて,どの程度のパケットサイズとパケットの間隔ならば
パケット受信に成功するかを測定する際,どの程度短い間隔でパケットを送信できるかは,
パケットのサイズが関係している.パケットのサイズ毎にパケットを送信,
または受信する処理でどのくらいの時間がかかっているかを知る必要がある.
このため,パケットの送信と受信処理の流れを示し,パケットのサイズによって
どの程度の時間がかかるかを測定する.

%%%%%%%%%%%%%%%%%%%%%%%%%%
\subsection{送信処理}
%%%%%%%%%%%%%%%%%%%%%%%%%%

%%%%%%%%%%%%%%%%%
\subsubsection{処理流れ}\label{flow-tx}
%%%%%%%%%%%%%%%%%
\insertfigure[0.3]{fig:flow-tx}{fig25}{パケット送信処理フロー}{ipi route}

パケット送信処理は,デバッグ支援OSのシステムコールとして実装しているデバッグ支援機構
が行う.デバッグ支援機構がパケットをNICの受信バッファに配置し,IPIを送信することで
パケットの送信を擬似的に行う.
デバッグ支援機構の処理中のIPIを送信するまでの処理を図\ref{fig:flow-tx}に示し,以下で説明する.
\begin{enumerate}
    \item パケットの格納\\
        作成したパケットをMintの共有メモリに作成したNICの受信バッファに格納する.
    \item 受信ディスクリプタの更新\\
        NICドライバはパケットの受信処理を行う際,受信ディスクリプタからパケットを
        受信しているかどうかを判断する.このため,受信ディスクリプタを更新し,
        受信状態にする.
    \item 待ち時間の作成\\
        デバッグ支援機構は指定した間隔でパケットを送信する機能を持っているため,
        IPIを送信する前に指定した秒数になるまで処理を待つ.
    \item IPIの送信\\
        デバッグ対象OSの保持するコアに向けてIPIを送信する.これにより,デバッグ対象OSの
        NICドライバが割り込みハンドラを動作させる.
\end{enumerate}

%%%%%%%%%%%%%%%%%
\subsubsection{測定}
%%%%%%%%%%%%%%%%%

\ref{flow-tx}項におけるIPIを送信するまでの時間を測定した.つまり,(1)と(2)にかかる時間を測定した.
RDTSCを用いて,(1)の処理の始めから,(2)の処理の最後までを256回計り,その平均値を処理時間とした.
測定した結果を,表\ref{time-until-IPI}に示す.

\begin{table}[h]
    \caption{IPIの送信までにかかる時間}
    \label{time-until-IPI}
    \begin{center}
        \begin{tabular}{l|l}   \hline \hline 
            パケットサイズ & かかった時間 \\ \hline
            1KB            & 302ns        \\ 
            1.5KB          & 326ns        \\ 
            4KB            & 683ns        \\ 
            8KB            & 1287ns       \\ 
            12KB           & 2132ns       \\ 
            16KB           & 3314ns       \\ \hline
        \end{tabular}
    \end{center}
\end{table}

結果から,パケットのサイズが大きいほど処理に時間がかかることが分かる.
また,表\ref{time-until-IPI}の間隔以下ではパケットを送信することができない.

%%%%%%%%%%%%%%%%%%%%%%%%%%
\subsection{受信処理}
%%%%%%%%%%%%%%%%%%%%%%%%%%

%%%%%%%%%%%%%%%%%
\subsubsection{処理流れ}\label{flow-rx}
%%%%%%%%%%%%%%%%%
\insertfigure[0.3]{fig:flow-rx}{fig26}{パケット受信処理フロー}{ipi route}

パケット受信処理は,デバッグ対象OSのNICドライバが行う.デバッグ支援OSが送信した
IPIを契機にして,割り込みハンドラが動作する.この際,割り込みハンドラ内では,
パケット受信割り込み処理を行わず,NAPIという機構を用いて,ポーリングでパケットの
処理を行わせるようにNICドライバのポーリング関数をNAPIが管理しているpoll\_listに
登録し,ソフトウェア割り込みを発生させる.
ソフトウェア割り込みを受けて,poll\_listから登録したポーリング関数を呼び出し,
パケット受信割り込みを行う.パケット受信割り込み処理が終了すると,poll\_listから
ポーリング関数を削除する.
パケット受信処理を図\ref{fig:flow-rx}に示し,以下で説明する.
\begin{enumerate}
    \item 割り込みハンドラの動作\\
        デバッグ支援OSから送信されたIPIを契機にして,割り込みハンドラが動作する.
        これにより,NICドライバのポーリング関数をpoll\_listに登録し,ソフトウェア割り込みを
        発せさせる.
    \item ポーリング関数の呼び出し\\
        ネットワーク層に位置するnet\_rx\_action()により,poll\_listから
        (1)で登録したポーリング関数を呼び出す.
    \item ポーリング関数の実行\\
        NICドライバのポーリング関数が実行され,受信割り込み処理関数であるrtl8169\_rx\_interrupt()
        を呼び出す.rtl8169\_rx\_interrupt()の処理が終了すると,poll\_listから
        ポーリング関数を削除する.
    \item 受信割り込み処理\\
        受信割り込み処理を行う.受信バッファにあるパケットを処理すると終了する.
\end{enumerate}


%%%%%%%%%%%%%%%%%
\subsubsection{測定}
%%%%%%%%%%%%%%%%%

\ref{flow-rx}項の処理流れの内,以下の2項目を測定した.
\begin{enumerate}
    \item 割り込みハンドラが開始してからポーリング関数が呼ばれるまでの時間の測定\\
        割り込みハンドラではパケット受信処理を行なっていないため,割り込みを挿入してから
        パケットが受信されるまでの時間を知るには,割り込みハンドラが
        呼ばれてから,ポーリング関数が開始するまでの時間を知る必要がある.
        このため,割り込みハンドラが開始してからポーリング関数が呼ばれるまでの時間を
        測定する.
        具体的には,RDTSCを用いて,\ref{flow-tx}項の(1)の始めから,(3)の始めまでの時間を測定した.
    \item 受信割り込み処理にかかる時間の測定\\
        どの程度のIPI送信間隔ならばパケットを受信できるかを知るためには,受信処理にどの程度の
        時間がかかるかを知る必要がある.このため,受信割り込み処理にかかる時間を測定する.
        具体的には,RDTSCを用いて,\ref{flow-rx}項の(4)の始めから,(4)のおわりまでの時間を測定した.
\end{enumerate}
(1)の測定結果は約3usであった.(2)の測定結果を表\ref{time-rx}に示す.

\begin{table}[h]
    \caption{パケットの受信にかかる時間}
    \label{time-rx}
    \begin{center}
        \begin{tabular}{l|l}   \hline \hline 
            パケットサイズ & かかった時間 \\ \hline
            1KB            & 384ns        \\ 
            1.5KB          & 350ns        \\ 
            4KB            & 683ns        \\ 
            8KB            & 1518ns       \\ 
            12KB           & 2065ns       \\ 
            16KB           & 3331ns       \\ \hline
        \end{tabular}
    \end{center}
\end{table}

結果から,割り込みハンドラの開始から約3usまでは処理が開始しないことが分かる.
また,パケットの処理はパケットのサイズが大きいほど時間がかかることが分かる.
表\ref{time-until-IPI}の値と類似していることから,memcpy()にかかる時間が大半を占めているのではないかと考えている.

%%%%%%%%%%%%%%%%%%%%%%%%%%%%%%%%%%%%%
\section{どの程度のパケットサイズと間隔ならば受信に成功するかの測定と考察}
%%%%%%%%%%%%%%%%%%%%%%%%%%%%%%%%%%%%%

%%%%%%%%%%%%%%%%%%%%%%%%%%
\subsection{測定対象}
%%%%%%%%%%%%%%%%%%%%%%%%%%

パケットのサイズとパケットの送信間隔を変化させて,どの程度のパケットサイズ,間隔ならば
パケットの受信に成功するかを測定する.
この測定の対象として,以下の2つがある.
\begin{enumerate}
    \item NICドライバ\\
        NICドライバでパケットを受信した際に,ソケットバッファに格納したことを
        成功と定義し,測定を行う.
    \item デバッグ対象OS上で動作するUDP受信プログラム\\
        デバッグ対象OS上で動作するUDP受信プログラムでメッセージを取得できたことを
        成功と定義し,測定を行う.
\end{enumerate}

%%%%%%%%%%%%%%%%%%%%%%%%%%
\subsection{NICドライバ}\label{result-driver}
%%%%%%%%%%%%%%%%%%%%%%%%%%
\insertfigure[0.5]{fig:result-driver}{fig27}{ドライバにおけるパケット受信成功確率}{ipi route}

6つのサイズのパケットを5000回連続で送信した際の受信成功率を計測する.
5000回は指定した一定の間隔で送信する.5000回連続で送信することを
1サイクルとし,間隔を1サイクル毎に増加させながら何サイクルも行う.
これらの操作により,どの程度のパケットサイズ,間隔ならば全てのパケットが受信に成功するかが分かる.
また,各パケットサイズにおける最短の間隔は,表\ref{time-until-IPI}の値を超えるものとしている.
結果を図\ref{fig:result-driver}に示し,以下で説明する.
図\ref{fig:result-driver}からどのサイズでも3μsまではほとんどパケットが受信できていないことが分かる.
この詳細についてはまだわかっていない.
3μs以降は,パケットが大きくなるほど傾きが緩やかになっていることが分かる.
これは,パケットが大きくなるほどパケット受信処理に時間がかかるためである.

%%%%%%%%%%%%%%%%%%%%%%%%%%
\subsection{デバッグ対象OS上で動作するUDP受信プログラム}
%%%%%%%%%%%%%%%%%%%%%%%%%%
\insertfigure[0.5]{fig:result-udp}{fig28}{プログラムにおけるパケット受信成功確率}{ipi route}

\ref{result-driver}節とほとんど同様の流れで測定を行う.
結果を図\ref{fig:result-udp}に示し,以下で説明する.
図\ref{fig:result-udp}から,パケットのサイズが増加するほど受信成功確率が上昇し始める時間が
増加していることが分かる.これは,パケットのサイズが増加するほどNICドライバで
すべてのパケットを処理できる時間が増加することに起因していると考えられる.
また,パケットのサイズによらず傾きは一定である.このことから,ドライバ以降の処理は,
パケットのサイズによらない処理をしていると考えられる.

%%%%%%%%%%%%%%%%%%%%%%%%%%%%%%%%%%%%%
\section{おわりに}
%%%%%%%%%%%%%%%%%%%%%%%%%%%%%%%%%%%%%

本資料では本デバッグ支援環境における送信処理と受信処理の処理流れ,および
これらにかかる時間を示した.また,この結果から,受信処理に関する考察を
行った.
\end{document}

%\begin{table}[htbp]
%    \caption{実験環境}
%    \label{kankyou}
%    \begin{center}
%        \begin{tabular}{l|l}   \hline \hline 
%            項目名      & 環境    \\ \hline
%            OS          & Fedora14 x86\_64(Mint 3.0.8)  \\ 
%            CPU         & Intel(R) Core(TM) Core i7-870 @ 2.93GHz \\ 
%            NICドライバ & RTL8169    \\ 
%            ソースコード& r8169.c \\ \hline
%
%        \end{tabular}
%    \end{center}
%\end{table}
%
%\insertfigure[0.8]{fig:frame}{fig1}{パケットの構成}{ipi route}


