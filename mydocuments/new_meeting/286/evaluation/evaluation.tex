% Created 2013-12-20 金 04:52
\documentclass[12pt]{jsarticle}
\usepackage[dvipdfmx]{graphicx}
\usepackage{comment}
%\usepackage{setspace}
%%\date{\today}
%\title{}
\textheight = 25truecm
\textwidth = 18truecm
\topmargin = -1.5truecm
\oddsidemargin = -1truecm
\evensidemargin = -1truecm
\marginparwidth = -1truecm
\usepackage{nutils}
\def\theenumii{\Alph{enumii}}
\def\theenumiii{\alph{enumiii}}
\def\labelenumi{(\theenumi)}
\def\labelenumiii{(\theenumiii)}
%\setstretch{0.9}
\begin{document}


%\maketitle
%\tableofcontents

\begin{center}
%%%%%%%%%%%%%%%%%%%%%%%%%%%%%%%%%%%%%%%
%%%タイトル                         %%%
%%%%%%%%%%%%%%%%%%%%%%%%%%%%%%%%%%%%%%%
    {\LARGE Mintを用いたデバッグ支援環境の評価}
\end{center}

\begin{flushright}
    2015/10/2\\
    藤田将輝
\end{flushright}

%%%%%%%%%%%%%%%%%%%%%%%%%%%%%%%%%%%%%
\section{はじめに}
%%%%%%%%%%%%%%%%%%%%%%%%%%%%%%%%%%%%%

本資料では,Mintを用いたデバッグ支援環境の評価について述べる.
本デバッグ支援環境ではIPIを用いて割り込みを発生させている.
したがって,まずコアがどの程度の短い間隔であればIPIを受信できるのかを測定した.
次に,本デバッグ支援環境を用いた際,NICドライバがどの程度の割り込み間隔であれば
パケットを正常に処理できるかを測定した.
最後に,本デバッグ支援環境を用いた際,デバッグ対象OS上で動作するUDPの受信プログラムが
どの程度の割り込み間隔であれば,パケットを正常に受信できるかを測定した.
これらの測定結果から本デバッグ支援環境を用いた際のスループットを算出し,
実際のNICと比較して,本デバッグ支援環境がNICドライバをデバッグすることに有益かどうかを
評価する.

%%%%%%%%%%%%%%%%%%%%%%%%%%%%%%%%%%%%%
\section{評価目的}
%%%%%%%%%%%%%%%%%%%%%%%%%%%%%%%%%%%%%

本評価の目的は,Mintを用いたデバッグ支援環境を使用して,デバッグ対象OSに
パケット受信割り込み処理をさせた際に,どの程度の割り込み間隔と
パケットのサイズならば正常にパケットを処理できるかを測定し,
実現できるスループットを算出することにより,
実際のNICを再現することに十分な性能を実現できているかを評価することである.

%%%%%%%%%%%%%%%%%%%%%%%%%%%%%%%%%%%%%
\section{評価環境}
%%%%%%%%%%%%%%%%%%%%%%%%%%%%%%%%%%%%%

本評価における評価環境を表\ref{environment}に示す.

\begin{table}[h]
    \caption{測定環境}
    \label{environment}
    \begin{center}
        \begin{tabular}{l|l}   \hline \hline 
            項目名      & 環境                                    \\ \hline
            OS          & Fedora14 x86\_64(Mint 3.0.8)            \\ 
            CPU         & Intel(R) Core(TM) Core i7-870 @ 2.93GHz \\
            メモリ      & 2GB                                     \\
            Chipset     & Intel(R) 5 Series/3400                  \\
            NICドライバ & RTL8169                                 \\ 
            ソースコード& r8169.c                                 \\ \hline
        \end{tabular}
    \end{center}
\end{table}

%%%%%%%%%%%%%%%%%%%%%%%
\section{測定対象}
%%%%%%%%%%%%%%%%%%%%%%%

本評価で行う測定は,以下の3つである.

\begin{description}
    \item[【測定1】]
        コアが反応可能なIPIの受信間隔の測定
    \item[【測定2】]
        NICドライバがパケットを処理可能なデバッグ支援機構の動作間隔とパケットサイズの測定
    \item[【測定3】]
        デバッグ対象OS上で動作するUDPの受信プログラムがパケットを処理可能な
        デバッグ支援機構の動作間隔とパケットサイズの測定
\end{description}

%%%%%%%%%%%%%%%%%%%%%%%
\section{測定1}
%%%%%%%%%%%%%%%%%%%%%%%

%%%%%%%%%%%%%%%%%%%%%%%
\subsection{測定方法}
%%%%%%%%%%%%%%%%%%%%%%%

IPIの間隔を調整し,連続でIPIを送信し,割り込みハンドラの動作確率を測定する.
具体的には,デバッグ支援OSからデバッグ対象OSに5000回連続でIPIを送信し,
デバッグ対象OSの割り込みハンドラを動作させた際に,カウンタをインクリメント
することで,動作した割り込みハンドラの回数を測定し,動作確率を算出する.
これらの動作を1サイクルとし,IPIの送信間隔を一定時間増加させながら何サイクルも行う.
これにより,全てのIPIにコアが反応できる送信間隔を調査する.

%%%%%%%%%%%%%%%%%%%%%%%
\subsection{測定結果}\label{result-ipi-limit}
%%%%%%%%%%%%%%%%%%%%%%%
\insertfigure[0.6]{fig:top-ipi}{fig21}{IPIの反応限界}{ipi route}
測定結果を図\ref{fig:top-ipi}に示し,分かったことを以下で説明する.

\begin{enumerate}
    \item IPIの送信間隔が0usから1usの間,割り込みハンドラの動作確率は一次関数
        的に増加する.
    \item IPIの送信間隔を1us以上にすると動作確率は100\%となる.
\end{enumerate}

%%%%%%%%%%%%%%%%%%%%%%%
\section{測定2}
%%%%%%%%%%%%%%%%%%%%%%%

%%%%%%%%%%%%%%%%%%%%%%%
\subsection{測定方法}\label{method-measure}
%%%%%%%%%%%%%%%%%%%%%%%

本デバッグ支援環境を用いて,連続でパケットを送信した際のNICドライバでパケットを処理できる確率を測定する.
測定に用いたパケットはUDPパケットのEthernetフレームであり,パケットのサイズはEthernetフレームのサイズである.
具体的には以下の方法でNICドライバでのパケットの受信成功確率を測定する.
\begin{enumerate}
    \item デバッグ支援機構の動作間隔を指定し,デバッグ支援機構を5000回動作させる.
    \item NICドライバでパケットを処理する際,共有メモリに配置したカウンタをインクリメントする.
    \item デバッグ支援機構を5000回動作させた後,デバッグ支援OSが共有メモリに配置したカウンタを
        取得し,成功回数を求める.この回数とデバッグ支援機構動作回数である5000回から受信成功率を算出する.
\end{enumerate}
この流れを1サイクルとし,1サイクル毎にデバッグ支援機構の動作間隔を100ns増加させながら何サイクルも行う.
これにより,各サイクルのデバッグ支援機構の動作間隔でのパケット受信成功確率を求められる.
また,これらの操作を以下の6つのサイズのパケットで行うことで,各パケットのサイズにおける受信可能な
動作間隔がわかる.
\begin{enumerate}
    \item 1KB
    \item 1500B
    \item 4KB
    \item 8KB
    \item 12KB
    \item 16KB
\end{enumerate}
さらに,1サイクルにかかった時間を測定する.この時間と,処理したデータ量からどの程度の通信量を実現できているかが
分かる.

%%%%%%%%%%%%%%%%%%%%%%%
\subsection{測定結果}\label{result-driver-limit}
%%%%%%%%%%%%%%%%%%%%%%%

\insertfigure[0.7]{fig:driver-limit}{fig20}{NICドライバにおける受信成功確率}{ipi route}
測定結果を図\ref{fig:driver-limit}に示し,読み取れることを以下に示す.

\begin{enumerate}
    \item 0usから3usの間,どのサイズでもほとんどの割り込みでパケット受信に失敗している.
    \item 3usを超えると,一次関数的に受信成功確率は増加し始める.この際,パケットサイズが小さいほど
        傾きが急になっている.
    \item 受信成功確率が100\%になってからはどれだけ動作間隔を増加させても受信成功確率は100\%となる.
\end{enumerate}

%%%%%%%%%%%%%%%%%%%%%%%
\section{測定3}
%%%%%%%%%%%%%%%%%%%%%%%

%%%%%%%%%%%%%%%%%%%%%%%
\subsection{測定方法}
%%%%%%%%%%%%%%%%%%%%%%%

本デバッグ支援環境を用いて,デバッグ支援OSからデバッグ対象OSへパケットを送信した際の,
デバッグ対象OS上で動作するUDPの受信用プログラムでどの程度の間隔とパケットサイズならば,
パケットを受信できるかを測定する.また,デバッグ支援OSからデバッグ対象OS上のプログラムまでで
実現できている通信量を測定する.
具体的な方法は,\ref{method-measure}節とほぼ同様である.カウンタをインクリメントする箇所が
\ref{method-measure}節ではNICドライバがパケットを受信した時であったのに対し,本測定ではUDPの受信プログラムが
パケットを受信した時であることが差分である.

%%%%%%%%%%%%%%%%%%%%%%%
\subsection{測定結果}\label{result-program-limit}
%%%%%%%%%%%%%%%%%%%%%%%

\insertfigure[0.7]{fig:program-limit}{fig19}{UDP受信プログラムにおける受信成功確率}{ipi route}

測定結果を図\ref{fig:program-limit}に示し,読み取れることを以下に示す.
\begin{enumerate}
    \item どのパケットサイズでもある時点まではほとんど受信に失敗している.
    \item ある時点から一次関数的に受信成功確率が増加している.
    \item 受信成功確率が増加し始めるある時点は,パケットサイズが増加するに連れて増大している.
    \item 受信成功確率が一次関数的に増加する際,パケットのサイズに関わらず,傾きは一定である.
    \item 受信成功確率が0回になるデバッグ支援機構の動作間隔以上に間隔を増加させても受信成功確率は100\%となる.
\end{enumerate}

%%%%%%%%%%%%%%%%%%%%%%%
\section{考察}
%%%%%%%%%%%%%%%%%%%%%%%

%%%%%%%%%%%%%%%%%%%%%%%
\subsection{測定1}
%%%%%%%%%%%%%%%%%%%%%%%

測定結果である\ref{result-ipi-limit}節から,現在の測定環境ではIPIの送信間隔が約1us以上であれば
連続でIPIを送信した際に,全てのIPIに割り込みハンドラが動作できることが分かる.
したがって,本実験環境下では,連続で割り込みを発生させる際,最低でも1us以上の間隔は必要であると言える.

%%%%%%%%%%%%%%%%%%%%%%%
\subsection{測定2}
%%%%%%%%%%%%%%%%%%%%%%%

測定結果である\ref{result-driver-limit}節から,連続で割り込みを発生させた際,
パケットのサイズを増加させるほど全てのパケットを処理することにかかる時間が増加することが分かる.
また,NICドライバで測定した受信成功回数,パケットのサイズ,および1サイクルにかかった時間から
各パケットにおける通信量を算出した.各パケットサイズにおける初めて
受信成功確率が100\%になった際のデバッグ支援機構の動作間隔,1サイクルにかかった時間,
総データ量,および通信量を表\ref{table:driver}に示す.
\begin{table}[h]
    \caption{NICドライバにおけるパケットのサイズと割り込み間隔の関係}
    \label{table:driver}
    \begin{center}
        \begin{tabular}{l|l|l|l|l}   \hline \hline 
            サイズ   & 割り込み間隔      & 1サイクルの時間      & 総データ量   & スループット       \\ \hline
            1KB      & 4.3us            & 21689us              & 5000KB       & 1.7Gbps            \\ 
            1500B    & 4.4us            & 22203us              & 7324KB       & 2.5Gbps            \\ 
            4KB      & 4.6us            & 23195us              & 20000KB      & 6.5Gbps            \\ 
            8KB      & 5.5us            & 27788us              & 40000KB      & 10.9Gbps           \\ 
            12KB     & 5.9us            & 29770us              & 60000KB      & 15.3Gbps           \\ 
            16KB     & 6.4us            & 32491us              & 80000KB      & 18.7Gbps           \\ \hline
        \end{tabular}
    \end{center}
\end{table}

%%%%%%%%%%%%%%%%%%%%%%%
\subsection{測定3}
%%%%%%%%%%%%%%%%%%%%%%%

測定結果である\ref{result-program-limit}節から,連続で割り込みを発生させた際,
NICドライバでの結果と比べて,より多くの時間がかかっていることが分かる.
このことから,NICドライバですべてのパケットを処理できていても,
プログラムまでパケットが届くにはより多くの時間がかかることが分かる.
また,UDPの受信プログラムで測定した受信成功回数,パケットのサイズ,および1サイクルにかかった時間から
各パケットサイズにおける通信量を算出した.各パケットサイズにおける初めて
受信成功確率が100\%になった際のデバッグ支援機構の動作間隔,1サイクルにかかった時間,
総データ量,および通信量を表\ref{table:program}に示す.

\begin{table}[h]
    \caption{UDP受信プログラムにおけるパケットのサイズと割り込み間隔の関係}
    \label{table:program}
    \begin{center}
        \begin{tabular}{l|l|l|l|l}   \hline \hline 
            サイズ   & 割り込み間隔      & 1サイクルの時間      & 総データ量   & スループット       \\ \hline
            1KB      & 5.5us            & 27677us              & 5000KB       & 1.3Gbps            \\ 
            1500B    & 5.5us            & 27700us              & 7324KB       & 2.0Gbps            \\ 
            4KB      & 6.1us            & 30705us              & 20000KB      & 4.9Gbps            \\ 
            8KB      & 7.7us            & 38649us              & 40000KB      & 7.8Gbps            \\ 
            12KB     & 9.4us            & 47241us              & 60000KB      & 9.6Gbps            \\ 
            16KB     & 11.1us           & 55789us              & 80000KB      & 10.9Gbps           \\ \hline
        \end{tabular}
    \end{center}
\end{table}

\section{評価}

上記の結果と考察から,NICドライバで最大約18.7Gbps,デバッグ対象OS上で動作するUDP受信プログラムで
最大約10.9Gbpsのスループットを実現できることが分かった.
PCI Express 1.1 の1レーンの転送量が2.5Gbpsであることから,実際のNICが実現できる最大のスループットは
2.5Gbpsであると言える.
本デバッグ支援環境で実現できるスループットと実際のNICが実現できるスループットを比較すると,
本デバッグ支援環境は実際のNICのスループットを大きく超えたスループットを実現できていることがわかる.
このことから,本デバッグ支援環境を用いることで現在は実現できていない高スループットの
NICのエミュレートができると考えられる.
高スループットのNICやバスが開発されたと想定し,これらに対応するドライバを開発できる.

\section{おわりに}

本資料では,本デバッグ支援環境の評価について述べた.
本デバッグ支援環境を用いると,NICドライバで最大18.7Gbps,プログラムで最大10.9Gbpsを
実現できる.
実際のNICは最大2.5Gbpsのスループットであるため,これの4〜7倍のスループットを実現できている.

\end{document}

%\begin{table}[htbp]
%    \caption{実験環境}
%    \label{kankyou}
%    \begin{center}
%        \begin{tabular}{l|l}   \hline \hline 
%            項目名      & 環境    \\ \hline
%            OS          & Fedora14 x86\_64(Mint 3.0.8)  \\ 
%            CPU         & Intel(R) Core(TM) Core i7-870 @ 2.93GHz \\ 
%            NICドライバ & RTL8169    \\ 
%            ソースコード& r8169.c \\ \hline
%
%        \end{tabular}
%    \end{center}
%\end{table}
%
%\insertfigure[0.8]{fig:frame}{fig1}{パケットの構成}{ipi route}


