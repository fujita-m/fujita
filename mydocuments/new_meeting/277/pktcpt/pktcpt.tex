% Created 2013-12-20 金 04:52
\documentclass[12pt]{jsarticle}
\usepackage[dvipdfmx]{graphicx}
\usepackage{comment}
%\usepackage{setspace}
%%\date{\today}
%\title{}
\textheight = 25truecm
\textwidth = 18truecm
\topmargin = -1.5truecm
\oddsidemargin = -1truecm
\evensidemargin = -1truecm
\marginparwidth = -1truecm
\usepackage{nutils}
\def\theenumii{\Alph{enumii}}
\def\theenumiii{\alph{enumiii}}
\def\labelenumi{(\theenumi)}
\def\labelenumiii{(\theenumiii)}
%\setstretch{0.9}
\begin{document}

%\maketitle
%\tableofcontents

\begin{center}
%%%%%%%%%%%%%%%%%%%%%%%%%%%%%%%%%%%%%%%
%%%タイトル                         %%%
%%%%%%%%%%%%%%%%%%%%%%%%%%%%%%%%%%%%%%%
    {\LARGE デバッグ支援機構の実装}
\end{center}

\begin{flushright}
  2015/6/10\\
  藤田将輝
\end{flushright}
%%%%%%%%%%%%%%%%%%1章%%%%%%%%%%%%%%%%%%%
\section{はじめに}
本資料では,山本が作成していたデバッグ支援機構のコードに機能を追加,改変し,
デバッグ支援機構を動作させたことを示す.
山本が作成していた機能として,IPIの送信機能とインターバルの生成機能がある.
自身が追加した機能として,パケットを配置する機能と受信ディスクリプタを更新する機能がある.
また,改変を行った機能としてインターバルを作成する機能がある.
パケットを作成する機能については完了していない.
このため,パケットの構造を調査し,この機能を実装する.

\section{デバッグ支援機構}
\subsection{デバッグ支援機構の処理流れ}
本デバッグ支援環境では,NICを用いずNICの割り込み処理を再現するため,NICの動作を再現するデバッグ支援機構を実装する.
デバッグ支援OSから任意のタイミングで割り込みを発生可能にするため,デバッグ支援機構はデバッグ支援OSのシステムコールとして実装する.
デバッグ支援機構の動作を図\ref{fig:kikou}に示し,以下で説明する.
\begin{enumerate}
    \item デバッグ支援OS上で動作するプロセスがシステムコールを発行する.
    \item デバッグ支援機構がNICドライバに処理させるパケットを作成する.
    \item デバッグ支援機構が作成したパケットをMintの共有メモリに配置する.
    \item デバッグ支援機構がMintの共有メモリに配置された受信ディスクリプタを更新する.
    \item デバッグ支援機構がデバッグ支援OSが占有するコアへIPIの送信要求を発行する.
\end{enumerate}
また,連続でパケットを送信する際,(2)〜(5)の動作を繰り返し実行し,(5)の実行後,指定したインターバルを作成する.
\subsection{機能}
2.1節の処理流れを実現するため,以下の機能を実現する必要がある.
\begin{description}
    \item[(機能1)] パケットを作成する機能
    \item[(機能2)] パケットをMintの共有メモリに配置する機能
    \item[(機能3)] 受信ディスクリプタを更新する機能
    \item[(機能4)] 指定したコアへのIPI送信要求を発行する機能
    \item[(機能5)] インターバルを作成する機能
\end{description}

\insertfigure[0.6]{fig:kikou}{fig1}{デバッグ支援機構の処理流れ}{ipi route}
\section{山本が実装した機能}
山本が実装していた機能は(機能4)と(機能5)である.
それぞれについて以下で説明する.
\begin{description}
    \item[(機能4)]指定したコアへのIPI送信要求を発行する機能\\
        {\tt apic\_icr\_write()}によって,IPI送信用のレジスタであるInterrupt Command Register(ICR)に値を書き込むことで
        IPIを送信する.この際,引数に送信先のコアIDとベクタ番号を指定する.
        これにより,NICドライバの割り込みハンドラが動作し,パケット受信割り込み処理を開始する.
    \item[(機能5)]インターバルを作成する機能\\
        {\tt ndelay()}関数を用いて,インターバルを実現している.しかし,{\tt ndelay()}では実際はズレが発生する.このため,
        より正確なインターバルを実現する必要がある.
\end{description}

\section{藤田が実装した機能}
3章の山本が作成したコードに藤田が機能を追加した.
追加した機能は(機能1),(機能2),および(機能3)である.また,山本が実装していた(機能5)を改変した.
追加と改変を行った機能について以下で説明する.
\begin{description}
    \item[(機能1)]パケットを作成する機能\\
        UDPパケットを作成する機能を作成した.しかし,現在はパケットの構成とヘッダの内容を正常に構成できていない.
        このため,デバッグ対象OSで動作するNICドライバが共有メモリに配置されたパケットを正常に処理できていない.
        パケットの構成とそのヘッダの内容を検討する必要がある.
    \item[(機能2)]パケットをMintの共有メモリに配置する機能\\
        {\tt memcpy()}により,作成したパケットをMintの共有メモリに配置した.現在は1つの割り込みに1つのパケットのみを配置し,処理させている.
    \item[(機能3)]受信ディスクリプタを更新する機能\\
        NICドライバが共有メモリに配置した受信ディスクリプタをデバッグ支援機構が取得し,更新する機能を実装した.
        受信ディスクリプタの受信状態を表すビットを立てることにより,NICドライバがパケットの受信処理を行う.
    \item[(機能5)]インターバルを作成する機能\\
        {\tt ndelay()}関数はズレが生じるため,for文で指定したクロック数が経過するまでループさせ,正確なインターバルを実現した.
\end{description}

\section{動作確認}
作成したデバッグ支援機構の動作確認の手順を以下に示す.
\begin{enumerate}
    \item Mintを用いてデバッグ支援OSとデバッグ対象OSを起動する.
    \item デバッグ支援OSでNICを起動する.この際,動作させるNICドライバは以下の改変を加えている.
        \begin{enumerate}
            \item 指定されたベクタ番号により動作する割り込みハンドラを追加登録している.
            \item Mintの共有メモリに受信バッファを作成する.
            \item Mintの共有メモリに受信ディスクリプタを配置する.
        \end{enumerate}
    \item デバッグ支援OS上でデバッグ支援機構を呼び出すプロセスを動作させる.
\end{enumerate}
これらの手順により,デバッグ対象OSのNICドライバの割り込みハンドラが動作した.
また,割り込みハンドラの処理中で,共有メモリからパケットを取り出し,ソケットバッファに
格納していることを確認した.
これらから,(機能2),(機能3),および(機能4)を実現できていることを確認した.
しかし,(機能1)において,パケットの構成とその内容を正常に構成できていないことが考えられるため,
ユーザ空間にパケットを送信できていない.

\section{課題}
今後の課題として,(機能1)パケットの作成を実現することである.
パケットのプロトコルはUDPとしている.
最終目標は,ユーザがIPアドレス,ポート番号,サイズを指定し,パケットを作成する機能を実装する.
現在は正常に処理されるパケットをキャプチャし,キャプチャしたパケットをNICドライバに処理させることで
ユーザ空間にパケットを送信できている.
キャプチャしたパケットの情報からヘッダの情報を確認し,正常に処理されるパケットを作成する機能を実装する.

\section{おわりに}
本資料では,デバッグ支援機構の実装と動作について述べた.
今後の課題として,UDPパケットを作成する機能を実装を行う.
\end{document}


