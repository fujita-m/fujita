% Created 2013-12-20 金 04:52
\documentclass[12pt]{jsarticle}
\usepackage[dvipdfmx]{graphicx}
\usepackage{comment}
%\usepackage{setspace}
%%\date{\today}
%\title{}
\textheight = 25truecm
\textwidth = 18truecm
\topmargin = -1.5truecm
\oddsidemargin = -1truecm
\evensidemargin = -1truecm
\marginparwidth = -1truecm
\usepackage{nutils}
\def\theenumii{\Alph{enumii}}
\def\theenumiii{\alph{enumiii}}
\def\labelenumi{(\theenumi)}
\def\labelenumiii{(\theenumiii)}
%\setstretch{0.9}
\begin{document}

%\maketitle
%\tableofcontents

\begin{center}
%%%%%%%%%%%%%%%%%%%%%%%%%%%%%%%%%%%%%%%
%%%タイトル                         %%%
%%%%%%%%%%%%%%%%%%%%%%%%%%%%%%%%%%%%%%%
{\LARGE パケットのキャプチャ}
\end{center}

\begin{flushright}
  2015/5/7\\
  藤田将輝
\end{flushright}
%%%%%%%%%%%%%%%%%%1章%%%%%%%%%%%%%%%%%%%
\section{はじめに}
本資料では,正常に処理されるパケットを複製し,IPIによってこれを処理させる流れについて示す.
\section{概要}
本資料における処理の流れを図\ref{fig:gaiyou}に示し,以下で説明する.
\begin{enumerate}
    \item 一方の計算機から他方の計算機にUDPを送信する.
    \item 受信側の計算機でパケットをキャプチャする.
    \item Mintを用いて2つのOSを起動する.
    \item キャプチャしたパケットを共有メモリに配置する.
    \item コア0からコア1へIPIを送信する.
    \item コア1がIPIを受信すると割り込みハンドラが動作し,
        パケットの受信割り込み処理を行う.
\end{enumerate}
\section{パケットのキャプチャ}
一方の計算機から他方の計算機にUDPを送信し,メモリ上に配置されたパケットをキャプチャする.

\section{おわりに}
本資料ではIPIとNICからの割り込みの違いを述べた.
また,その違いにより,NICドライバのパケット受信割り込み処理
に影響がないことを示した.
\end{document}


