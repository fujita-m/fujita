% Created 2013-12-20 金 04:52
\documentclass[12pt]{jsarticle}
\usepackage[dvipdfmx]{graphicx}
\usepackage{comment}
%\usepackage{setspace}
%%\date{\today}
%\title{}
\textheight = 25truecm
\textwidth = 18truecm
\topmargin = -1.5truecm
\oddsidemargin = -1truecm
\evensidemargin = -1truecm
\marginparwidth = -1truecm
\usepackage{nutils}
\def\theenumii{\Alph{enumii}}
\def\theenumiii{\alph{enumiii}}
\def\labelenumi{(\theenumi)}
\def\labelenumiii{(\theenumiii)}
%\setstretch{0.9}
\begin{document}


%\maketitle
%\tableofcontents

\begin{center}
%%%%%%%%%%%%%%%%%%%%%%%%%%%%%%%%%%%%%%%
%%%タイトル                         %%%
%%%%%%%%%%%%%%%%%%%%%%%%%%%%%%%%%%%%%%%
    {\LARGE パケットジェネレータの作成}
\end{center}

\begin{flushright}
  2015/7/27\\
  藤田将輝
\end{flushright}
%%%%%%%%%%%%%%%%%%1章%%%%%%%%%%%%%%%%%%%
\section{はじめに}
本デバッグ支援環境ではNICを用いずパケット受信処理を再現するため,
パケットを作成する必要がある.
本デバッグ支援環境において,パケットの作成はデバッグ支援OS上で動作する
プロセスによって行われる.
本デバッグ支援環境では,このプロセス中でlibnetというパケット作成用の
ライブラリを用いることで,パケットの作成を実現している.
本資料では,デバッグ支援OS上で動作するパケットを作成するプロセスが
正常に動作したことを示す.
\section{パケットジェネレータ}
本デバッグ支援環境では,デバッグ支援OS上で動作するパケットを作成し,
デバッグ支援機構を呼び出すプロセスをパケットジェネレータと呼ぶ.
パケットジェネレータの機能を以下に示し,説明する.
\begin{enumerate}
    \item パケットの作成\\
        デバッグ対象OSで動作するNICドライバが処理するパケットを作成する.
        IPアドレス,ポート番号,およびメッセージを指定し,パケットを作成できる.
        作成したパケットはEthernetFrameであり,Ethernetヘッダ,IPv4ヘッダ,およびUDPヘッダ
        を含んでいる.
    \item デバッグ支援機構の呼び出し\\
        デバッグ対象OSが占有しているコアID,NICドライバの割り込みハンドラが登録されているベクタ番号,
        割り込みの発生回数,インターバル,パケットの先頭アドレス,およびパケットのサイズを引数に
        デバッグ支援機構を呼び出す.デバッグ支援機構はシステムコールとして実装されている.
\end{enumerate}
\section{処理流れ}
パケットジェネレータにおいて,libnetを用いてパケットが作成されるまでの流れを図\ref{fig:libnet}に示し,以下で説明する.
\begin{enumerate}
    \item {\tt libnet\_init()}を使用し,libnetコンテキストを初期化する.
    \item {\tt libnet\_build\_udp()}を使用し,payloadにUDPヘッダを付与し,UDPパケットを作成する.
        この際,指定したポート番号が,ヘッダに適用される.
    \item {\tt libnet\_build\_ipv4}を使用し,UDPパケットにIPv4ヘッダを付与し,IPv4パケットを作成する.
        この際,指定したIPアドレスが,ヘッダに適用される.
    \item {\tt libnet\_build\_ethernet()}を使用し,IPv4パケットにEthernetヘッダを付与し,
        EthernetFrameを作成する.
    \item {\tt libnet\_adv\_cull\_packet()}を使用し,パケットを取り出す.
        また,この関数により,パケットのサイズも取得できる.
\end{enumerate}
以上の流れからパケットを作成し,作成したパケットをデバッグ対象OSのNICドライバに
処理させたところ,正常に処理されたことを確認した.
ここで正常な処理とは,デバッグ対象OS上で動作するUDPパケットを受信し,
UDPのpayloadのメッセージを標準出力に出力するプロセスにより,
メッセージが表示されることである.
\insertfigure[0.45]{fig:libnet}{fig1}{パケット作成流れ}{ipi route}
\section{課題}
今後の課題として,パケットのサイズと送信間隔を
変更し,どの程度正常にパケットが処理されるかを測定することがある.
1回の試行で送信する回数と,間隔,およびサイズを検討する.
\section{おわりに}
本資料では,実装したパケットジェネレータがどのような処理流れで
パケットを作成するかを示した.
また,作成したパケットがデバッグ対象OS上で動作するプロセスにおいて
正常に処理されることを確認した.
今後は,本デバッグ支援環境においてどの程度連続でパケットを処理できるか
を測定し,評価する.
\end{document}


