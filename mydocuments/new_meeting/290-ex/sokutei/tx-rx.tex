% Created 2013-12-20 金 04:52
\documentclass[12pt]{jsarticle}
\usepackage[dvipdfmx]{graphicx}
\usepackage{comment}
%\usepackage{setspace}
%%\date{\today}
%\title{}
\textheight = 25truecm
\textwidth = 18truecm
\topmargin = -1.5truecm
\oddsidemargin = -1truecm
\evensidemargin = -1truecm
\marginparwidth = -1truecm
\usepackage{nutils}
\def\theenumii{\Alph{enumii}}
\def\theenumiii{\alph{enumiii}}
\def\labelenumi{(\theenumi)}
\def\labelenumiii{(\theenumiii)}
%\setstretch{0.9}
\begin{document}


%\maketitle
%\tableofcontents

\begin{center}
%%%%%%%%%%%%%%%%%%%%%%%%%%%%%%%%%%%%%%%
%%%タイトル                         %%%
%%%%%%%%%%%%%%%%%%%%%%%%%%%%%%%%%%%%%%%
    {\LARGE パケットの送受信処理時間の測定}
\end{center}

\begin{flushright}
    2015/11/25\\
    藤田将輝
\end{flushright}
%%%%%%%%%%%%%%%%%%1章%%%%%%%%%%%%%%%%%%%
\section{はじめに}
本デバッグ支援環境において,送受信処理でどの程度
時間がかかるかを測定した.この時間を正しく測定することにより,
本環境を使用した際の受信可能なIPIの間隔,サイズがわかる.
本資料では,本デバッグ支援環境を用いた際の送信処理と受信処理流れ,
それぞれの処理時間を各パケットサイズで測定した際の結果を示す.

\section{送受信処理流れ}
\subsection{送信処理}
本デバッグ支援環境における送信処理の流れを以下に示す.
\begin{enumerate}
    \item Mintの共有メモリにあるNICの受信ディスクリプタを
        受信済み状態に更新する.これにより,NICが受信バッファから
        パケットを取得できる状態となる.
    \item NICの受信バッファにあらかじめ用意していたパケットを
        格納する.この際,複数のパケットは配置せず,1つのパケットのみ
        を配置する.また,パケットとは,UDPのフレームである.
    \item 指定した間隔になるまで処理を待たせる.これは,rdtscで(1)の
        処理の始めのタイムスタンプを取り,そこから指定した時間になるまで
        ループを回すことで実現している.
    \item 共有メモリに配置している割り込みの禁止/許可を示すフラグを
        確認し,許可状態であればIPIを送信する.禁止状態であれば
        IPIを送信せず処理を終了する.
\end{enumerate}
上記の処理を指定した回数ループさせることにより,連続でパケットを送信する.

\subsection{受信処理}
本デバッグ支援環境における受信処理の流れを以下に示す.
\begin{enumerate}
    \item IPIを受け,割り込みハンドラが動作する.Mintの共有メモリに配置している
        割り込みの禁止/許可を表すフラグを更新し,割り込み禁止状態にする.
        このあと,ソフト割り込みを発生させる.
    \item ソフト割り込みのハンドラが動作し,NICドライバのポーリング関数を呼び出す.
    \item NICドライバのポーリング関数が呼び出され,パケット受信処理関数を呼び出す.
    \item パケット受信処理を行う.NICの受信バッファからパケットを取り出し,
        ソケットバッファに格納し,上位層へ送信する.
\end{enumerate}

\section{測定}
\subsection{測定環境}
測定環境を表\ref{}に示す.

\subsection{測定方法}
\subsubsection{送信処理}
各パケットサイズにおける送信処理の処理時間を測定する.
具体的には,rdtscを用いて,\ref{}節の(1)の処理の開始から(2)の処理の終了までの
差分を取ることで,処理時間を測る.
パケットサイズは以下の6つである.
\begin{enumerate}
    \item 1KB
    \item 1.5KB
    \item 4KB
    \item 8KB
    \item 12KB
    \item 16KB
\end{enumerate}

\subsubsection{受信処理}
各パケットサイズにおける受信処理の処理時間を測定する.
パケットを処理している部分は\ref{}節の(4)であるため,
(1)の処理の開始から(3)の処理の終了までの時間の測定と(4)の開始から終了までの
時間の測定にわけて測定を行った.
パケットサイズは\ref{}項と同じ6つである.

\subsection{結果}
\subsubsection{送信処理}
\insertfigure[0.8]{fig:tx-1024}{fig16}{1KBのパケットにおける送信処理時間}{ipi route}
\insertfigure[0.8]{fig:tx-1500}{fig17}{1.5KBのパケットにおける送信処理時間}{ipi route}
\insertfigure[0.8]{fig:tx-4096}{fig18}{4KBのパケットにおける送信処理時間}{ipi route}
\insertfigure[0.8]{fig:tx-8192}{fig19}{8KBのパケットにおける送信処理時間}{ipi route}
\insertfigure[0.8]{fig:tx-12244}{fig20}{12KBのパケットにおける送信処理時間}{ipi route}
\insertfigure[0.8]{fig:tx-16384}{fig21}{16KBのパケットにおける送信処理時間}{ipi route}
各パケットにおける測定結果を図1〜6に示す.
図からどのサイズでも一定間隔毎に飛び値が発生していることがわかる.
このため,これらの処理時間を定める際は,この飛び値を省いたもので平均を
取ることが考えられる.

\subsubsection{受信処理A}
\insertfigure[0.8]{fig:rx-from}{fig23}{割り込みが入ってからパケット受信処理が開始するまでの時間}{ipi route}
各パケットサイズにおける測定結果を図7に示す.
図からパケットサイズによらずこの値は一定であることが分かる.


\subsubsection{受信処理B}
\insertfigure[0.8]{fig:tx-16384}{fig9}{1KBのパケットにおける送信処理時間}{ipi route}
\insertfigure[0.8]{fig:tx-16384}{fig10}{1.5KBのパケットにおける送信処理時間}{ipi route}
\insertfigure[0.8]{fig:tx-16384}{fig11}{4KBのパケットにおける送信処理時間}{ipi route}
\insertfigure[0.8]{fig:tx-16384}{fig12}{8KBのパケットにおける送信処理時間}{ipi route}
\insertfigure[0.8]{fig:tx-16384}{fig13}{12KBのパケットにおける送信処理時間}{ipi route}
\insertfigure[0.8]{fig:tx-16384}{fig14}{16KBのパケットにおける送信処理時間}{ipi route}
\insertfigure[0.8]{fig:tx-16384}{fig2}{最初の処理を省いた1KBのパケットにおける送信処理時間}{ipi route}
\insertfigure[0.8]{fig:tx-16384}{fig3}{最初の処理を省いた1.5KBのパケットにおける送信処理時間}{ipi route}
\insertfigure[0.8]{fig:tx-16384}{fig4}{最初の処理を省いた4KBのパケットにおける送信処理時間}{ipi route}
\insertfigure[0.8]{fig:tx-16384}{fig5}{最初の処理を省いた8KBのパケットにおける送信処理時間}{ipi route}
\insertfigure[0.8]{fig:tx-16384}{fig6}{最初の処理を省いた12KBのパケットにおける送信処理時間}{ipi route}
\insertfigure[0.8]{fig:tx-16384}{fig7}{最初の処理を省いた16KBのパケットにおける送信処理時間}{ipi route}
各パケットにおける測定結果を図8〜13に示す.
図から最初の処理に多大な時間がかかっていることがわかる.
このため,最初の処理を除いたものを図14〜19に示す.
図からどのサイズでも一定間隔毎に飛び値が発生していることがわかる.
このため,これらの処理時間を定める際は,この飛び値を省いたもので平均を
取ることが考えられる.

\section{おわりに}
本資料ではパケットサイズによる送受信処理時間の変化を測定したことを示した.

\end{document}

%\begin{table}[htbp]
%    \caption{実験環境}
%    \label{kankyou}
%    \begin{center}
%        \begin{tabular}{l|l}   \hline \hline 
%            項目名      & 環境    \\ \hline
%            OS          & Fedora14 x86\_64(Mint 3.0.8)  \\ 
%            CPU         & Intel(R) Core(TM) Core i7-870 @ 2.93GHz \\ 
%            NICドライバ & RTL8169    \\ 
%            ソースコード& r8169.c \\ \hline
%
%        \end{tabular}
%    \end{center}
%\end{table}
%
%\insertfigure[0.8]{fig:frame}{fig1}{パケットの構成}{ipi route}
