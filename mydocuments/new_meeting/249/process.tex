% Created 2013-12-20 金 04:52
\documentclass[12pt]{jsarticle}
\usepackage[dvipdfmx]{graphicx}
\usepackage{comment}
%%\date{\today}
%\title{}
\textheight = 25truecm
\textwidth = 18truecm
\topmargin = -1.5truecm
\oddsidemargin = -1truecm
\evensidemargin = -1truecm
\marginparwidth = -1truecm
\def\theenumii{\Alph{enumii}}
\def\theenumiii{\alph{enumiii}}
\def\labelenumi{(\theenumi)}
\def\labelenumiii{(\theenumiii)}
\begin{document}

%\maketitle
%\tableofcontents

\begin{center}
%%%%%%%%%%%%%%%%%%%%%%%%%%%%%%%%%%%%%%%
%%%タイトル                         %%%
%%%%%%%%%%%%%%%%%%%%%%%%%%%%%%%%%%%%%%%
{\LARGE Linuxカーネルへのシステムコール実装手順書}
\end{center}

\begin{flushright}
  2014/4/21\\
  藤田将輝
\end{flushright}

%%%%%%%%%%%%%%%%%%%%%%%%%%%%%%%%%%%%%%%
%%%はじめに                         %%%
%%%%%%%%%%%%%%%%%%%%%%%%%%%%%%%%%%%%%%%
\section{はじめに}
本手順書ではLinuxカーネルへのシステムコール実装の手順を
示す.以下に章構成を示す.
\begin{description}
\item[第1章]はじめに
\item[第2章]実装環境
\item[第3章]カーネルのソースコードの取得
\item[第4章]システムコールの実装手順
\item[第5章]注意すべき点
\item[第6章]おわりに
\end{description}


%%%%%%%%%%%%%%%%%%%%%%%%%%%%%%%%%%%%%%%
%%%実装環境                         %%%
%%%%%%%%%%%%%%%%%%%%%%%%%%%%%%%%%%%%%%%
\section{実装環境}
実装環境を表1に示す.
\begin{table}[htbp]
\caption{実装環境}
\label{kankyou}
\begin{center}
\begin{tabular}{|l|l|}   \hline 
項目名      & 環境    \\ \hline \hline
OS          & Fedora14  \\ \hline
カーネル    & Linuxカーネル 3.0.8 64bit   \\ \hline
CPU         & Intel(R) Core(TM) Core i7-870 @ 2.93GHz \\ \hline
メモリ      & 2GB \\  \hline
           
\end{tabular}
\end{center}
\end{table}
%%%%%%%%%%%%%%%%%%%%%%%%%%%%%%%%%%%%%%%
%%%システムコールの実装手順         %%%
%%%%%%%%%%%%%%%%%%%%%%%%%%%%%%%%%%%%%%%
\section{カーネルのソースコードの取得}
Gitを用いて,カーネルのソースコードを取得する.
Gitはバージョン管理ツールであり,git clone [リポジトリのURL]を実行すると指定した
リポジトリを取得する.
git checkout -b [ブランチ名] [タグ名]を実行すると指定したブランチ名のブランチを作成し,切り替える.
また,タグを指定すると,特定の状態からブランチを作成できる.
ブランチとはGitにおける開発ラインのことである.
以下に実行例を示す.
\begin{verbatim}
$ git clone git://git.kernel.org/pub/scm/linux/kernel/git/ \
  stable/linux-stable.git Linux-3.0.8
$ cd Linux-3.0.8/
$ git checkout -b linux3.0.8 v3.0.8  
\end{verbatim}
実行例ではLinux-3.0.8にリポジトリを取得し,
linux3.0.8というブランチを作成し,切り替えている.
\section{システムコールの実装手順}
\subsection{カレントディレクトリ}
カレントディレクトリは/home/fujita/git/Linux-3.0.8とする.
\subsection{システムコールの追加}
システムコールの追加手順について以下で説明する.
書き加えた行には+を付与している.
\begin{enumerate}
\item システムコール本体の作成

arch/x86/kernel以下にシステムコール本体を作成する.
例としてtest.cというファイルを作成し,sys\_testという名前の
システムコールを作成する.
これは,整数を引数にとり,この整数の2倍の値を返すものである.
以下にソースコードを示す.

\begin{verbatim}
1  int sys_test(int arg1)
2  {
3    return arg1 + arg1;
4  } 
\end{verbatim}

\item ヘッダ・ファイルの編集

unistd\_64.hとsyscalls.hを編集する.
\begin{enumerate}
\item unistd\_64.hの編集

arch/x86/include/asm/unistd\_64.hに追加するシステムコールの名前とシステムコール番号をつけ,
書き加える.
例ではシステムコール番号として,\_\_NR\_testを309と定義し,sys\_testを追加している.
システムコール番号は既存のシステムコール番号と重複しないようにする.
以下に例を示す.

\begin{verbatim}
 682     #define __NR_setns                              308
 683     __SYSCALL(__NR_setns, sys_setns)
+684     #define __NR_test                               309
+685     __SYSCALL(__NR_test, sys_test)
\end{verbatim}

\item syscalls.hの編集

arch/x86/include/asm/syscalls.hにシステムコールのプロトタイプ宣言を
追加する.
以下に例を示す.

\begin{verbatim}
 43     asmlinkage int sys_get_thread_area(struct user_desc __user *);
+44     extern int sys_test(int arg1);
\end{verbatim}

\end{enumerate}
\item Makefileの編集

arch/x86/kernel/Makefileにシステムコールのオブジェクトファイル名を追加する.
以下に例を示す.

\begin{verbatim}
 117     obj-$(CONFIG_OF)    += devicetree.o
+118     obj-y               += test.o
\end{verbatim}

\end{enumerate}
\subsection{カーネルの再構築}
カーネルの再構築の手順を以下に示す.
\begin{enumerate}
\item カーネルの設定ファイルの作成

カーネルの設定ファイル.configを作成する.
\begin{enumerate}
\item ncurses-develのインストール

make menuconfigを実行するために必要なncurses-develを
インストールする.
ncurses-develはカーネルの設定画面を表示するために
必要なパッケージである.
以下に実行例を示す.


\begin{verbatim}
$ sudo yum install ncurses-devel
\end{verbatim}


\item make menuconfigの実行

以下のコマンドでカーネルを設定する.


\begin{verbatim}
$ make menuconfig
\end{verbatim}


コマンドを実行するとカーネルの設定画面に切り替わる.
Fedora14では標準ファイルシステムがext4であり,
ext4に対応するようにカーネルを設定する.
File system項目で以下のように設定する.
組み込む項目には[$\ast$]を付ける.
なお,組み込むか否かはスペースキーの押下で変更する.


\begin{verbatim}
<*>The Extend 4 (ext4) filesystem
[*]  Use ext4 for ext2/ext3
[*]  Ext4 extended attributes
[*]    Ext4 POSIX Access Contorol Lists
[*]    Ext4 Security Labels
\end{verbatim}


設定後,Escキーを入力しホーム画面に戻る.
Exitを選択し,終了する.
すべての設定が終了するとカレントディレクトリ以下に
.configが作成される.
\end{enumerate}
\item bzImageの作成

bzImageはLinuxカーネルであるvmlinuxを
圧縮した形式のファイルである.
以下にbzImageを作成する実行例を示す.
なお,j8はコンパイルを並列処理し,
コンパイル時間を短縮するオプションである.

\begin{verbatim}
$ make bzImage -j8
\end{verbatim}

この操作を行うことで,bzImageがarch/x86/boot以下に作成される.
さらにカーネルが使用するシンボルテーブルであるSystem.mapが
カレントディレクトリ以下に作成される.
\item カーネルモジュールのコンパイル,インストール

カーネルモジュールはカーネルの機能を拡張するもので,
必要に応じてカーネルにロード,アンロードできる.
例ではインストールの実行を終えるとDEPMOD 3.0.8+と表示される.
この3.0.8+はモジュールがインストールされたディレクトリ名である.
これは/lib/modules以下に作成される.
以下にカーネルモジュールをコンパイルし,インストールするコマンドを示す.

\begin{verbatim}
$ make modules
$ sudo make modules_install
\end{verbatim}

実行結果例を以下に示す.

\begin{verbatim}
  INSTALL sound/usb/snd-usbmidi-lib.ko
  INSTALL sound/usb/usx2y/snd-usb-us122l.ko
  INSTALL sound/usb/usx2y/snd-usb-usx2y.ko
  DEPMOD  3.0.8+
\end{verbatim}

\item カーネルのインストール

(1)で作成したbzImageとSystem.mapに名前を付けて/boot以下にコピーする.
例では作成したbzImageとSystem.mapにそれぞれ
vmlinuz-3.0.8-fujita,System.map-3.0.8-fujitaという名前を付ける.
カーネルをインストールする実行例を以下に示す.

\begin{verbatim}
$ sudo cp arch/x86/boot/bzImzge /boot/vmlinuz-3.0.8-fujita
$ sudo cp System.map /boot/System.map-3.0.8-fujita
\end{verbatim}

\item 初期RAMディスクの生成

初期RAMディスクとは実際のルートファイルシステムが使用できるようになる前に
マウントされる初期ルートファイルシステムである.
以下に初期RAMディスクを作成するコマンドを示す.

\begin{verbatim}
$ sudo mkinitrd /boot/initrd-3.0.8-fujita.img 3.0.8+
\end{verbatim}

この3.0.8+の部分はmake moduls\_installを実行したときに表示される
DEPMODに記されているディレクトリを指定する.
3.0.8+の中にはカーネルモジュールがインストールされており,これをもとに初期RAMディスク
を作成する.
例では初期RAMディスクinitrd-3.0.8-fujita.imgを作成している.
\item ブートローダの設定

再構築したカーネルを起動するために,GRUBの設定ファイルにエントリを追加する.
GRUBとはブートローダであり,ブート処理を行うものである.
/boot/grub/menu.lstを編集することで,設定を
変更できる.
編集にはルート権限が必要である.
以下に追加するエントリの例を示す.

\begin{verbatim}
1    title Linux-3.0.8-fujita
2            root (hd0,0)
3            kernel /vmlinuz-3.0.8-fujita                             \
4            ro root=UUID=56a552db-4819-4fa4-8b8e-5f76afb7edd7        \
5            rd_NO_LUKS rd_NO_LVM rd_NO_MD rd_NO_DM LANG=en_US.UTF-8  \
6            SYSFONT=latarcyrheb-sun16 KEYTABLE=jp106 rhgb quiet  
7            initrd /initrd-3.0.8-fujita.img
\end{verbatim}

1行目のtitleはカーネルの選択画面で表示されるエントリ名であり,任意のエントリ名を入力する.
2行目のrootは/bootのパーティションを指定する.
書式は(ハードディスクの番号,パーティション番号)となっている.
パーティションとはハードディスクの分割された個々の領域のことである.
例では1番目のハードディスクの1番目のパーティションを指定している.
3行目のkernelは起動するカーネルを指定する.
今回の例では(3)で名前を付けたvmlinuz-3.0.8-fujitaを指定している.
4,5,6行目はブートオプションである.
4行目のro rootにはルートのパーティションのUUIDを指定する.
UUIDは一意な識別子である.
5,6行目は既存のエントリと同様に設定する.
7行目は起動時に使用する初期RAMディスクを指定する.
例では(4)で作成したinitrd-3.0.8-fujita.imgを指定している.
%titleは任意の名前を付ける.rootは最初にhdの番号,2つ目に...kernelには生成したカーネルを指定する.
%UUIDには1つ目のHDのUUIDを指定する.最後の行のinitrdには先ほど生成した初期RAMディスクを指定する.
%今回の例では"initrd-3.0.8-fujita"を指定する.

\item 再起動

すべての設定を終えた後,以下のコマンドで計算機を再起動する.

\begin{verbatim}
$ sudo reboot
\end{verbatim}

\end{enumerate}
\subsection{テスト}

実装したシステムコールを使ったプログラムを作成し,
実行することでシステムコールをテストする.
今回実装したシステムコールは整数を引数にとり,その整数の2倍の値を返す.
よって,4を引数にした場合,8が出力されればシステムコールの実装の成功となる.
システムコールの呼び出しにはsyscall関数を用いる.
このとき,syscall関数に1つ目の引数として,実装したシステムコールのシステムコール番号,
2つ目の引数として,実装したシステムコールの引数を与える.
テストプログラムを以下に示す.

\begin{verbatim}
1  #include<stdio.h>
2  #include<sys/syscall.h>
3  
4  main()
5  {
6          printf("%d\n",syscall(309,4));
7  }
\end{verbatim}

%このプログラムをコンパイルし,実行する.
%コマンドと実行結果を以下に示す.

%\begin{verbatim}
%$ gcc test.c
%$ ./a.out
%8
%\end{verbatim}

%8が出力されているため,成功である.
\section{注意すべき点}
カーネルの再構築時に注意すべき点はbzImageや初期RAMディスクを
インストールする際に生成されるディレクトリや
ファイルを上書きせずに,名前を変えて別のカーネルとして
保存することである.
これにより,
既存の環境を破壊することを防ぐ.
%%%%%%%%%%%%%%%%%%%%%%%%%%%%%%%%%%%%%%%
%%%おわりに                         %%%
%%%%%%%%%%%%%%%%%%%%%%%%%%%%%%%%%%%%%%%
\section{おわりに}
本手順書ではLinuxカーネルへのシステムコールの実装手順を示した.
\end{document}
