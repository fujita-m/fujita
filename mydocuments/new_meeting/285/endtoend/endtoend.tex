% Created 2013-12-20 金 04:52
\documentclass[12pt]{jsarticle}
\usepackage[dvipdfmx]{graphicx}
\usepackage{comment}
%\usepackage{setspace}
%%\date{\today}
%\title{}
\textheight = 25truecm
\textwidth = 18truecm
\topmargin = -1.5truecm
\oddsidemargin = -1truecm
\evensidemargin = -1truecm
\marginparwidth = -1truecm
\usepackage{nutils}
\def\theenumii{\Alph{enumii}}
\def\theenumiii{\alph{enumiii}}
\def\labelenumi{(\theenumi)}
\def\labelenumiii{(\theenumiii)}
%\setstretch{0.9}
\begin{document}


%\maketitle
%\tableofcontents

\begin{center}
%%%%%%%%%%%%%%%%%%%%%%%%%%%%%%%%%%%%%%%
%%%タイトル                         %%%
%%%%%%%%%%%%%%%%%%%%%%%%%%%%%%%%%%%%%%%
    {\LARGE デバッグ支援環境におけるパケット受信の測定}
\end{center}

\begin{flushright}
    2015/9/18\\
    藤田将輝
\end{flushright}
%%%%%%%%%%%%%%%%%%1章%%%%%%%%%%%%%%%%%%%
\section{はじめに}
本資料では,デバッグ支援環境を用いて,
デバッグ支援OSからデバッグ対象OSにUDPパケットを連続で送信した際に,
デバッグ対象OS上で動作する受信プログラムがどの程度の割り込み間隔,および
パケットサイズならば,パケットを全て受信できるかについて測定したことを示す.
測定の目的は,本デバッグ支援環境の動作を確認し,どの程度の通信量を実現できるかを調査することである.
前回の資料<No.284-07>では,デバッグ支援OSで作成したパケットがデバッグ対象OSのNICドライバで
ソケットバッファに格納された数を計測した.
これに対し,本資料では,デバッグ支援OSで作成したパケットがデバッグ対象OS上で動作する
UDPの受信プログラムによって受信された数を計測している.

\section{測定}
\subsection{測定目的}
本測定は,本デバッグ支援環境がどの程度の通信量を実現して動作できるかを
調査することである.
また,前回の資料で示した,NICドライバで処理できる割り込み間隔と比較し,
差異があればこの差異について考察することである.
これらのため,デバッグ対象OSに繰り返し割り込みを発生させ,パケット受信割り込み処理を
させた際の,受信に失敗した回数を計測する.
また,繰り返し割り込みを発生させた際の時間を計測し,計測した時間と,パケットの受信に成功した
データ量から通信量を計算する.
なお,受信に失敗するとは,割り込みを発生させたにも関わらず,デバッグ対象OS上で動作する
UDPの受信プログラムがパケットを受信できなかったことを指す.

\subsection{測定環境}
本測定における測定環境を表に示す.
\begin{table}[htbp]
    \caption{実験環境}
    \label{kankyou}
    \begin{center}
        \begin{tabular}{l|l}   \hline \hline 
            項目名      & 環境    \\ \hline
            OS          & Fedora14 x86\_64(Mint 3.0.8)  \\ 
            CPU         & Intel(R) Core(TM) Core i7-870 @ 2.93GHz \\ 
            NICドライバ & RTL8169    \\ 
            ソースコード& r8169.c \\ \hline
        \end{tabular}
    \end{center}
\end{table}

\subsection{測定方法}
本測定では,条件を変更させながら,繰り返し割り込みを発生させることで,
デバッグ対象OSで複数回パケット受信割り込み処理をさせ,パケットが受信できなかった数を測定する.
ここで,パケットが受信できなかったとは,デバッグ対象OS上で動作する受信プログラムが
recv()でパケットを受け取れなかったことを指す.
具体的な動作は,以下の動作を3000回繰り返すことを1サイクルとして,条件を変えながら500サイクル行う.
なお,この動作はデバッグ支援OSが実装しているデバッグ支援機構によって行う.
\begin{enumerate}
    \item パケットをMintの共有メモリ上にあるNICの受信バッファに配置する.
    \item Mintの共有メモリ上にあるNICの受信ディスクリプタを取得し,受信済み状態に更新して,
        Mintの共有メモリ上に再配置する.
    \item デバッグ対象OSのコアへIPIを送信する.
    \item 指定したインターバルとなるまで処理を待つ.
    \item (1)に戻る.
\end{enumerate}
上記(1)〜(4)の動作1回につき,デバッグ対象OSではパケット受信割り込み処理が発生し,
デバッグ対象OS上で動作するUDPの受信プログラムが1パケットを受信する.

\subsection{失敗回数の測定}
失敗回数の測定は,デバッグ支援OSとデバッグ対象OSの2つのOSが参照できるカウンタを用いることで
行っている.
デバッグ対象OS上で動作するrecv()がパケットを受信すると,共有メモリに配置しているカウンタを
インクリメントする.1サイクルが終了すると,デバッグ支援OSがカウンタを取得し,ループ回数の3000回との差分
を取ることで,受信に失敗した回数を測定している.

\subsection{測定流れと測定箇所}
\insertfigure[0.7]{fig:flow}{fig2}{処理フロー}{ipi route}
スループットを算出するため,1サイクルにかかった時間を計測している.
1サイクルにかかった時間と受信に成功したデータ量からスループットを計算する.
本測定における処理フローと時間の測定箇所を図\ref{fig:flow}に示す.
デバッグ支援OSにおける処理流れを以下で説明する.
\begin{enumerate}
    \item デバッグ支援OS上でパケットジェネレータを動作させる.
    \item システムコールによりデバッグ支援機構が呼び出され,パケットを配置する.
    \item Mintの共有メモリに配置している受信ディスクリプタを更新し,再配置する.
    \item デバッグ対象OSの占有するコアへIPIを送信する.
    \item 指定した時間の間隔になるまで処理を待たせる.
    \item 3000回ループすれば,次に進み,3000回に満たなければ(2)に戻る.
    \item 共有メモリに配置しているカウンタを取得し,3000回と比較し,失敗回数を求める.
    \item 指定したサイクル数に達していれば終了し,指定したサイクル数に満たなければ(2)に戻る.
\end{enumerate}
デバッグ対象OSにおける処理流れを以下で説明する.
\begin{enumerate}
        \renewcommand{\labelenumi}{(\Alph{enumi})}
    \item IPIを受けて割り込みハンドラが動作する.
    \item パケット受信割り込み処理が動作する.
    \item 受信プログラムのrecv()がパケットを受信する.
    \item システムコールにより,カウンタを更新し,共有メモリに配置する.
\end{enumerate}
以上のフローの中で,時間を計測した箇所は(2)から(6)のループを抜けた直後である.


\subsection{測定手順}
測定手順について以下に示し,説明する.
\begin{enumerate}
    \item Mintを用いて,デバッグ支援OSとデバッグ対象OSを動作させる.
    \item デバッグ対象OS上でUDPの受信プログラムを動作させる.
    \item デバッグ支援OS上で1サイクルのループ回数,1サイクルごとに増加させる時間,サイクル数,およびパケットのサイズを引数に取り,
        パケットジェネレータを動作させる.
    \item (割り込みの間隔,失敗回数,1サイクルに要した時間)という形式でデータをファイルに出力する.
\end{enumerate}

\subsection{測定結果}
\insertfigure[0.9]{fig:end2end}{fig1}{受信プログラムのパケット受信におけるサイズと割り込み間隔の関係}{ipi route}
パケットサイズ毎のパケット受信失敗回数と割り込み間隔の測定結果を図\ref{fig:end2end}に示す.
また,各サイズ毎の最初に受信失敗回数が0回になった時の割り込み間隔,1サイクルにかかった時間,およびスループットについて,
表\ref{result}に示す.
\begin{table}[htbp]
    \caption{パケットのサイズと割り込み間隔の関係}
    \label{result}
    \begin{center}
        \begin{tabular}{l|l|l|l}   \hline \hline 
            サイズ(KB)  & 割り込み間隔(ns)  & 1サイクルの時間(us)  & スループット(Gbps) \\ \hline
            1           & 5400              & 16319                & 1.4                \\ 
            4           & 6300              & 19037                & 4.8                \\ 
            8           & 7800              & 23536                & 7.7                \\ \hline
        \end{tabular}
    \end{center}
\end{table}

\section{考察}
測定結果からわかることを以下に示す.
\begin{enumerate}
    \item パケットのサイズによらず,ある時点までは受信に失敗した回数は一定である.
        この時点をtとする.
    \item 時点tを超えるまでは,ほとんど受信に失敗している.
    \item tを超えると受信失敗回数は一次関数的に減少し,割り込み間隔を増加させると
        受信失敗回数は0回になる.一次関数的に減少する際,パケットのサイズに関わらず,
        傾きは一定である.
    \item パケットのサイズを増加させると,tの値も増加する.この際,
        パケットのサイズとこれに対応するtの値は一次関数的な関係を持っている.
    \item 実現できる最大のスループットは,パケットサイズが8KBの時で,7.7Gbpsである.
\end{enumerate}
本測定の測定結果は,デバッグ対象OS上で動作するUDPの受信プログラム内でrecv()が受信したパケットの数から
受信失敗回数を求めている.
これに対し,前回の測定では,デバッグ対象OSが保持するNICドライバ内で,ソケットバッファに格納した
パケットの数から受信失敗回数を求めた.
これらの測定結果を比較して,わかったことを以下に示す.
\begin{enumerate}
    \item NICドライバで測定した各パケットサイズの初めて受信失敗回数が0回になった際の割り込み間隔と,受信プログラムにおける時点tがほぼ一致する.
        これを示すため,NICドライバで測定した各パケットサイズの初めて受信失敗回数が0回になった際の割り込み間隔と,受信プログラムにおける時点tを表に示す.
        \begin{table}[htbp]
            \caption{初めて受信失敗回数が0回になった際の割り込み間隔と時点tの値}
            \label{jiten}
            \begin{center}
                \begin{tabular}{l|l|l}   \hline \hline 
                    サイズ(KB)  & 割り込み間隔(ns)  & 時点t(ns)  \\ \hline
                    1           & 4300              & 4200   \\ 
                    4           & 4500              & 4500   \\ 
                    8           & 5400              & 5400   \\ \hline
                \end{tabular}
            \end{center}
        \end{table}
\end{enumerate}

\section{おわりに}
本資料では,デバッグ対象OS上で動作するUDPの受信プログラムにおいて,
どの程度の間隔,パケットサイズならばパケットを受信できるかについて測定したことを示した.
前回の測定と今回の測定から,本デバッグ環境を用いて,NICドライバでは最大10Gbpsを,
受信プログラムでは最大7.7Gbpsで動作することを確認した.

\end{document}

%\begin{table}[htbp]
%    \caption{実験環境}
%    \label{kankyou}
%    \begin{center}
%        \begin{tabular}{l|l}   \hline \hline 
%            項目名      & 環境    \\ \hline
%            OS          & Fedora14 x86\_64(Mint 3.0.8)  \\ 
%            CPU         & Intel(R) Core(TM) Core i7-870 @ 2.93GHz \\ 
%            NICドライバ & RTL8169    \\ 
%            ソースコード& r8169.c \\ \hline
%
%        \end{tabular}
%    \end{center}
%\end{table}
%
%\insertfigure[0.8]{fig:frame}{fig1}{パケットの構成}{ipi route}


