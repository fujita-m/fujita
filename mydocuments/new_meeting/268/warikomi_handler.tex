% Created 2013-12-20 金 04:52
\documentclass[12pt]{jsarticle}
\usepackage[dvipdfmx]{graphicx}
\usepackage{comment}
%\usepackage{setspace}
%%\date{\today}
%\title{}
\textheight = 25truecm
\textwidth = 18truecm
\topmargin = -1.5truecm
\oddsidemargin = -1truecm
\evensidemargin = -1truecm
\marginparwidth = -1truecm
\def\theenumii{\Alph{enumii}}
\def\theenumiii{\alph{enumiii}}
\def\labelenumi{(\theenumi)}
\def\labelenumiii{(\theenumiii)}
%\setstretch{0.9}
\begin{document}

%\maketitle
%\tableofcontents

\begin{center}
%%%%%%%%%%%%%%%%%%%%%%%%%%%%%%%%%%%%%%%
%%%タイトル                         %%%
%%%%%%%%%%%%%%%%%%%%%%%%%%%%%%%%%%%%%%%
{\LARGE 割り込みハンドラについての実験}
\end{center}

\begin{flushright}
  2015/1/6\\
  藤田将輝
\end{flushright}
%%%%%%%%%%%%%%%%%%1章%%%%%%%%%%%%%%%%%%%
\section{はじめに}
IPIによって動作し,共有メモリからパケットを取得し,
受信処理をする割り込みハンドラを作成している.
現在はIPIを受信すると,共有メモリから
メッセージを取得する機能を実現できた.
本資料では実際のNICドライバのパケット受信の流れと,
本機構でのパケット受信の流れを説明し,
割り込みハンドラについての実験について述べる.
\section{実験環境}
%本機構にける最終目標について図1に示し,以下で説明する.
%\begin{description}
%    \item[(機能1)] 共有メモリにパケットを格納する.
%    \item[(機能2)] コア0にIPI送信要求を出し,コア0からコア1へIPIを送信する.
%    \item[(機能3)] コア1がIPIを受信するとNICドライバが
%        共有メモリからパケットを取得する.
%\end{description}
%本資料で述べるのは機能3についてである.
%実際のNICドライバの受信処理流れと,本機構における処理流れを比較するため
%次章では本来のNICドライバの受信処理の流れについて述べる.
実験環境は表1のようになっている.
\begin{table}[htbp]
\caption{実験環境}
\label{kankyou}
\begin{center}
\begin{tabular}{|l|l|}   \hline 
項目名      & 環境    \\ \hline \hline
OS          & Fedora14 x86\_64(Mint 3.0.8)  \\ \hline
CPU         & Intel(R) Core(TM) Core i7-870 @ 2.93GHz \\ \hline
NICドライバ & RTL8169    \\ \hline
           
\end{tabular}
\end{center}
\end{table}

\section{進捗状況}
本実験を経た進捗状況について表2に示し以下で説明する.
本実験で可能になったことは以下の2つである.
\begin{enumerate}
    \item IPIにより動作するNICドライバの割り込みハンドラを
        作成できた.これによりNICドライバのプライベート構造体
        を参照可能になった.
    \item パケットを用意出来れば,共有メモリから
        パケットを取得し,パケットの受信処理が可能になった.
\end{enumerate}

\begin{table}[htbp]
\caption{進捗状況}
\label{kankyou}
\begin{center}
\begin{tabular}{l|l}   \hline 
項目名                          & 進捗状況    \\  \hline
パケットジェネレータの作成      & 20\% +0\%  \\ 
IPIの送信                       & 100\% +0\% \\ 
割り込みハンドラの作成          & 80\% +30\%    \\ 
           
\end{tabular}
\end{center}
\end{table}

\section{実際のNICドライバの受信処理流れ}
実際のNICドライバの受信処理流れを以下で説明する.
\begin{enumerate}
    \item NICからNICドライバへ割り込みが発生する.
    \item 割り込みハンドラである{\tt rtl8169\_interrupt()}により
        NIC(デバイス)のエラーチェックを行う.
    \item {\tt rtl8169\_interrupt()}により割り込みが禁止され,
        ネットワーク層に処理が移る.
    \item ネットワーク層の{\tt net\_rx\_action()}により
        NICドライバのポーリングルーチンである{\tt rtl8169\_poll()}
        が呼び出される.
    \item {\tt rtl8169\_poll()}により,パケット受信処理を行う
        {\tt rtl8169\_rx\_interrupt()}が呼び出される.
    \item {\tt rtl8169\_rx\_interrupt()}により,パケットの受信処理を行う. 
\end{enumerate}
次章では本機構におけるパケット受信処理流れを説明する.
\section{本機構におけるパケット受信処理流れ}
本機構におけるパケット受信処理流れについて以下で説明する.
\begin{enumerate}
    \item IPIによりNICドライバへ割り込みが発生する.
    \item 自作した割り込みハンドラである{\tt rtl8169\_interrupt\_fujita()}
        が動作する.この際,本機構ではNICを用いないため,
        デバイスのエラーチェックは必要としない.
        したがって自作した割り込みハンドラではデバイスのエラーチェックは
        行わない.
    \item {\tt rtl8169\_interrupt\_fujita()}により,{\tt rtl8169\_rx\_interrupt()}
        が呼び出される.
    \item {\tt rtl8169\_rx\_interrupt()}により,パケットの受信処理を行う.
\end{enumerate}
\section{割り込みハンドラにおける実験}
前章の処理流れを実現するために実験を行った.
IPIにより,{\tt rtl8169\_interrupt\_fujita()}を動作させ,
{\tt rtl8169\_rx\_interrupt()}を呼び出し,
この{\tt rtl8169\_rx\_interrupt()}内で,共有メモリからメッセージを取り出し,
表示させるという実験である.
本実験における流れについて以下で説明する.
\begin{enumerate}
    \item 割り込み元OSから共有メモリにメッセージを格納する.
    \item 割り込み先OSを起動し,NICドライバを起動する.
    \item 割り込み元OSから割り込み先OSへIPIを送信する.
    \item 割り込み先OSのNICドライバで割り込みハンドラである
        {\tt rtl8169\_interrupt\_fujita()}が動作する.
    \item {\tt rtl8169\_interrupt\_fujita()}により,{\tt rtl8169\_rx\_interrupt()}
        が動作する.
    \item {\tt rtl8169\_rx\_interrupt()}内で共有メモリからメッセージを取り出し,
        カーネルのメッセージバッファに表示する.
\end{enumerate}

\section{課題}
今後の課題について以下に示す.
\begin{enumerate}
    \item パケットの構造について調査する.
    \item パケットジェネレータを作成する.
    \item リングバッファに対応させ,複数のパケットの
        受信処理をできるようにする.
    \item 受信状態を更新できるようにする.
\end{enumerate}
\section{おわりに}
本資料では本機構における受信処理流れについて述べ,
その実験について説明した.
今後は課題に取り組み,実装をすすめる.

\end{document}


