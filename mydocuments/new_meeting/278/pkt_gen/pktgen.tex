% Created 2013-12-20 金 04:52
\documentclass[12pt]{jsarticle}
\usepackage[dvipdfmx]{graphicx}
\usepackage{comment}
%\usepackage{setspace}
%%\date{\today}
%\title{}
\textheight = 25truecm
\textwidth = 18truecm
\topmargin = -1.5truecm
\oddsidemargin = -1truecm
\evensidemargin = -1truecm
\marginparwidth = -1truecm
\usepackage{nutils}
\def\theenumii{\Alph{enumii}}
\def\theenumiii{\alph{enumiii}}
\def\labelenumi{(\theenumi)}
\def\labelenumiii{(\theenumiii)}
%\setstretch{0.9}
\begin{document}


%\maketitle
%\tableofcontents

\begin{center}
%%%%%%%%%%%%%%%%%%%%%%%%%%%%%%%%%%%%%%%
%%%タイトル                         %%%
%%%%%%%%%%%%%%%%%%%%%%%%%%%%%%%%%%%%%%%
    {\LARGE パケットジェネレータの実装}
\end{center}

\begin{flushright}
  2015/6/19\\
  藤田将輝
\end{flushright}
%%%%%%%%%%%%%%%%%%1章%%%%%%%%%%%%%%%%%%%
\section{はじめに}
本資料では,デバッグ支援環境において,NICドライバに処理させるパケットを作成する機能を実装したことを示す.
本デバッグ支援環境ではNICを用いずに任意のタイミングで割り込み処理を発生させるため,パケットを擬似する必要がある.
このため,デバッグ支援OS上で動作し,パケットを作成するプログラム(以下,パケットジェネレータ)を実装した.
プロトコルはUDPとしている.現在は任意のメッセージを指定し,パケットを作成できる.
このパケットを本環境により処理させたところ,
デバッグ支援OS上で動作するUDPの受信プログラムでパケットを受信できることを確認した.
\section{パケットジェネレータ}
パケットジェネレータはデバッグ支援OS上で動作するAPとして実装した.
作成するパケットはUDPであり,Etherフレームを擬似している.
パケットジェネレータを動作させるとパケットを作成し,作成したパケットとこのサイズを引数にデバッグ支援機構を呼び出す.
また,パケットを作成する際,任意のメッセージを入力し,これをデバッグ対象OSに送信できる.
\section{パケット配送の処理流れ}
\subsection{概要}
パケットジェネレータを使用し,デバッグ支援機構を呼び出すことで,デバッグ対象OSへパケットを配送する.
これを確認するため,デバッグ対象OS上でUDPの受信プログラムを動作させ,このプログラムが正常にパケットを受信し,
指定したメッセージを出力するかどうかを実験した.
以降でこの実験を行う際の環境の構成と,実験の流れについて説明する.
\subsection{環境構成}
本デバッグ環境はMintを用いてデバッグ支援OSとデバッグ対象OSの2つのOSを動作させる.
また,デバッグ支援OSはデバッグ支援機構を保持し,デバッグ対象OSは改変したNICドライバを保持している
本実験を行う環境の構築手順について以下で説明する.
\begin{enumerate}
    \item デバッグ支援OSからデバッグ対象OSを起動する.
    \item デバッグ対象OSでネットワークインタフェースを起動する.
    \item デバッグ対象OSでUDPの受信プログラムを動作させる.
\end{enumerate}
\subsection{パケットの配送}
構築した環境を用いて実験を行う.
パケットジェネレータを用いて,パケットを作成し,これが正常に処理されることを実験する.
パケットジェネレータを動作させてから,
デバッグ対象OSの画面上にメッセージが表示されるまでの流れを図\ref{fig:nagare}に示し,以下で説明する.
\begin{enumerate}
    \item デバッグ支援OSでパケットジェネレータを動作させる.
    \item パケットジェネレータにより,パケットが作成され,デバッグ支援機構が呼び出される.
    \item デバッグ支援機構は作成されたパケットを共有メモリに配置する.
    \item デバッグ支援機構がデバッグ対象OSに割り込みを発生させる.
    \item 割り込みハンドラが動作し,NICドライバはパケットを共有メモリからソケットバッファに格納する.
    \item NICドライバはソケットバッファを上位層に送信する.
    \item デバッグ対象OS上で動作するUDPの受信プログラムがUDPパケットを受け取る.
    \item デバッグ対象OS上で動作するUDPの受信プログラムがメッセージを画面に出力する.
\end{enumerate}
\insertfigure[0.6]{fig:nagare}{fig1}{パケットの配送流れ}{ipi route}
\section{実装}
\subsection{処理概要}
パケットジェネレータは指定したメッセージに定義したヘッダを付与することで
Etherフレームを擬似している.
パケットジェネレータを動作させると,3つのヘッダを定義し,
これに適切な値を割り当てることでEtherフレームを作成している.
以降で,定義したヘッダと,その内容について述べる.
\subsection{定義したヘッダ}
NICドライバが処理をするのはEtherフレームである.
このため,Etherフレームを擬似する必要がある.
現在は,正常に動作するパケットをキャプチャし,その内容から値を決定している.
Etherフレームの擬似を行うため,以下の3つのヘッダを定義した.
\begin{enumerate}
    \item Etherヘッダ
    \item IPv4ヘッダ
    \item UDPヘッダ
\end{enumerate}

\subsection{各ヘッダの情報}
定義した各ヘッダにおけるメンバの役割について以下に示し,説明する.
\begin{enumerate}
    \item Etherヘッダ
        \begin{enumerate}
            \item 宛先MACアドレス(6byte)\\
                宛先のMACアドレスを表す.
            \item 送信元MACアドレス(6byte)\\
                送信元のMACアドレスを表す.
            \item IPのバージョン(2byte)\\
                次に続くIPヘッダのバージョンを表す.IPv4ヘッダならば0x0800である.
        \end{enumerate}
    \item IPv4ヘッダ
        \begin{enumerate}
            \item バージョン(4bit)\\
                IPプロトコルのバージョンを表す.IPv4ならば0x4である.
            \item ヘッダ長(4bit)\\
                データを除いたヘッダ部分のみのサイズを表す.4byte単位で表しており,0x5ならば20byteとなる.
            \item サービスタイプ(1byte)\\
                IPパケットの優先度などを表す.現在はほとんど使われておらず,意味を持っていないことが多い.
            \item データグラム長(2byte)\\
                IPパケット全体の長さを表す.
            \item ID(2byte)\\
                フラグメンテーションが起きた際の識別に使用される.毎回ランダムな値が格納される.
            \item フラグ(3bit)\\
                フラグメンテーションの際に使用される.パケットがまだ続くか否かを識別する.
            \item フラグメントオフセット(13bit)\\
                フラグメントされたパケットがIPパケットのどの位置かを識別するために使用される.
            \item TTL(1byte)\\
                パケットの寿命を表す.
            \item プロトコル番号(1byte)\\
                次に続くプロトコルの情報を表す.
            \item チェックサム(2byte)\\
                IPヘッダのチェックサムを表す.計算方法として1の補数演算を利用する.
            \item 送信元IPアドレス(4byte)\\
                送信元のIPアドレスを表す.
            \item 宛先IPアドレス(4byte)\\
                宛先のIPアドレスを表す.
        \end{enumerate}
    \item UDPヘッダ
        \begin{enumerate}
            \item 送信元ポート(2byte)\\
                送信元のポートを表す.
            \item 宛先ポート(2byte)\\
                宛先のポートを表す.
            \item サイズ(2byte)\\
                UDPパケット全体の長さを表す.
            \item チェックサム(2byte)\\
                UDPパケットのチェックサムを表す.
                計算方法はIPパケットと同様に1の補数演算を利用する.
        \end{enumerate}
\end{enumerate}
\section{パケットジェネレータの動作}
パケットジェネレータを動作させると,パケットが作成され,デバッグ支援機構が呼び出される.
パケットを作成する際のプログラムの動作について以下に示し,説明する.
\begin{enumerate}
    \item メッセージを指定し,プログラムを起動する.
    \item 各ヘッダを定義し,値を割り当てる.この際,割り当てる値は,キャプチャしたパケットを参考にしている.
    \item ヘッダの末尾に作成したメッセージを配置する.
    \item ヘッダ全体のサイズと作成したメッセージのサイズを足しあわせたものをパケットのサイズとする.
    \item パケットとパケットのサイズを引数にデバッグ支援機構を呼び出す.
\end{enumerate}
\section{課題}
現在は,ヘッダの情報をキャプチャしたパケットから指定している.
このため全ての情報が静的に割り当てられている.
これらの情報の内,IPアドレス,ポート番号,サイズ,およびチェックサムはユーザの指定,
またはユーザが指定したメッセージのようなデータから計算されるべきである.
したがってこれらをユーザの入力によって変化させるよう改変する.
\section{おわりに}
本資料ではパケットジェネレータについて示した.
動作について,デバッグ対象OS上で動作するUDPの受信プログラムが指定したメッセージを出力したことから,
正常に処理されていることを確認した.
また,今後の課題として,ユーザの入力によってヘッダの値を自動的に決定するよう,パケットジェネレータを
改変する.
\end{document}


