% Created 2013-12-20 金 04:52
\documentclass[12pt]{jsarticle}
\usepackage[dvipdfmx]{graphicx}
\usepackage{comment}
%\usepackage{setspace}
%%\date{\today}
%\title{}
\textheight = 25truecm
\textwidth = 18truecm
\topmargin = -1.5truecm
\oddsidemargin = -1truecm
\evensidemargin = -1truecm
\marginparwidth = -1truecm
\def\theenumii{\Alph{enumii}}
\def\theenumiii{\alph{enumiii}}
\def\labelenumi{(\theenumi)}
\def\labelenumiii{(\theenumiii)}
%\setstretch{0.9}
\begin{document}

%\maketitle
%\tableofcontents

\begin{center}
%%%%%%%%%%%%%%%%%%%%%%%%%%%%%%%%%%%%%%%
%%%タイトル                         %%%
%%%%%%%%%%%%%%%%%%%%%%%%%%%%%%%%%%%%%%%
{\LARGE IPIからの割り込みとNICからの割り込みの違いについて}
\end{center}

\begin{flushright}
  2015/4/6\\
  藤田将輝
\end{flushright}
%%%%%%%%%%%%%%%%%%1章%%%%%%%%%%%%%%%%%%%
\section{はじめに}
本資料では,IPIによる割り込みとNIC(デバイス)による割り込みの
違いについて示す.
両者の割り込み発生から割り込みハンドラの動作までの処理流れを
示し,違いを示す.
\section{両者の違いについて}
\subsection{IPIとNICの割り込み通知の差}
IPIとNICの割り込み通知の差は,割り込み要因を示すベクタ番号
をコアに通知するまでの流れにある.
コアがベクタ番号を通知してからの処理は同様の処理を行う.
\subsection{IPIによる割り込み通知}
IPIによる割り込み通知の中でベクタ番号をコアに通知するまでの流れを以下に示す.
        \begin{enumerate}
            \item ユーザがIPI送信用のレジスタに
                送信先のPhysical APIC IDと割り込み
                要因を示すベクタ番号を指定し,IPIを送信する.
            \item 送信先のコアにベクタ番号が渡される.
        \end{enumerate}
\subsection{NICによる割り込み通知}
NICによる割り込み通知の中でベクタ番号をコアに通知するまでの流れを以下に示す.
        \begin{enumerate}
            \item I/O APICのピンに割り込みが通知される.
            \item ピンの番号がI/O APICにより,ベクタ番号に変換される.
            \item I/O APICがコアにベクタ番号を通知する.
        \end{enumerate}
\section{ベクタ番号を通知されてから割り込みハンドラが動作するまでの処理流れ}
コアがベクタ番号を通知されてから割り込みハンドラが動作するまでの処理流れを
以下に示す.
\begin{enumerate}
    \item
        コアはベクタ番号nを受け取った後,Interrupt Descriptor Table(IDT) 
        のn番目のエントリに登録された割り込みゲートを呼び出す.
    \item 
        割り込みゲートはベクタ番号を引数に,デバイスからの割り込みを処理する
        関数 do\_IRQ()を呼び出す.
    \item 
        do\_IRQ()はコアに対応するベクタ管理表を用い,ベクタ番号nから
        IRQ番号pを求める.
    \item
        OSは求めたIRQ番号pに対応する割り込み処理を行う.
\end{enumerate}
\section{IPIとNICからの割り込みの差によるパケット受信処理の影響}
IPIとNICからの割り込みの差は,ベクタ番号をコアへ通知する
までの流れのみである.
したがって,本研究で扱うNICドライバのパケット受信割り込み処理
には影響がないと考えられる.
\section{おわりに}
本資料ではIPIとNICからの割り込みの違いを述べた.
また,その違いにより,NICドライバのパケット受信割り込み処理
に影響がないことを示した.
\end{document}


