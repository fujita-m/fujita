% Created 2013-12-20 金 04:52
\documentclass[12pt]{jsarticle}
\usepackage[dvipdfmx]{graphicx}
\usepackage{comment}
%\usepackage{setspace}
%%\date{\today}
%\title{}
\textheight = 25truecm
\textwidth = 18truecm
\topmargin = -1.5truecm
\oddsidemargin = -1truecm
\evensidemargin = -1truecm
\marginparwidth = -1truecm
\usepackage{nutils}
\def\theenumii{\Alph{enumii}}
\def\theenumiii{\alph{enumiii}}
\def\labelenumi{(\theenumi)}
\def\labelenumiii{(\theenumiii)}
%\setstretch{0.9}
\begin{document}


%\maketitle
%\tableofcontents

\begin{center}
%%%%%%%%%%%%%%%%%%%%%%%%%%%%%%%%%%%%%%%
%%%タイトル                         %%%
%%%%%%%%%%%%%%%%%%%%%%%%%%%%%%%%%%%%%%%
    {\LARGE 2016年度研究計画(藤田)}
\end{center}

\begin{flushright}
    2016/3/7\\
    藤田将輝
\end{flushright}
%%%%%%%%%%%%%%%%%%1章%%%%%%%%%%%%%%%%%%%
\section{はじめに}
本資料は,藤田の2016年度研究計画を示す.
2016年度の前期はNICドライバにおけるパケット送信処理についての検討と
実装についてを進める.後期は,実装したパケット送信処理の評価と
バグの再現について評価する.また,第136回OS研(2016年11月開催予定)
の原稿を執筆する.最後に修士論文を執筆する.

\section{研究テーマ}
\begin{description}
    \item[(題目)]Mintオペレーティングシステムを用いたOS開発支援手法(仮)
    \item[(概要)]OS機能を実現する処理の中でも割込処理は非同期に発生するため,
        再現が困難であり,開発工数の増加を招く.これを解決する手法として,
        仮想計算機を用いた開発支援間手法が提案されている.しかし,仮想化による
        処理オーバヘッドのため短い間隔や一定間隔の割込発生が困難である.
        そこで,Mintを用いて開発支援OSと開発対象OSを独立に走行させる手法を提案する.
        提案手法は,既存手法と比較してハイパーバイザが存在しないため,
        短い間隔や一定間隔の割込を発生可能である.
\end{description}

\section{課題の進捗状況}
\subsection{完了済みの課題}
完了済みの課題を以下に示す.なお,課題番号と通番は別資料「研究課題一覧(藤田)(2016年3月7日)」
に対応している.
\begin{description}
    \item[(課題番号1-B)]IPI送受信時の処理の調査(通番4〜8)
    \item[(課題番号1-C)]Linuxのテストの調査(通番12)
    \item[(課題番号1-D)]NICドライバを対象とした開発支援環境の実装(通番15〜20)
    \item[(課題番号1-E)]評価(通番23〜25)
\end{description}

\subsection{未完了の課題}
未完了の課題を以下に示す.なお,課題番号と通番は別資料「研究課題一覧(藤田)(2016年3月7日)」
に対応している.
\begin{description}
    \item[(課題番号1-A)]割込処理が発生する経路の調査(通番1〜3)
    \item[(課題番号1-C)]Linuxのテストの調査(通番10〜11)\\
        Linuxにおける,ドライバのテストにはどのようなものがあるかを調査する.
        これにより,困難であるテストを明確にする.
        また,報告されているバグについても調査し,提案手法でバグを再現する.
    \item[(課題番号1-D)]NICドライバを対象とした開発支援環境の実装(通番1〜2,21)\\
        現時点では,パケット受信割込処理のみを対象としている.
        今後は,パケット送信処理においても検討し,処理を実装する.
    \item[(課題番号1-E)]評価(通番1,26〜28)\\
        現在は,NICドライバにおいての通信処理性能を評価している.
        今後はアプリケーションレイヤまでを評価対象とし,通信処理性能を評価する.
        また,報告されているバグを再現することで,デバッグにも使用できることを示す.
        さらに,仮想計算機を用いるとどの程度の影響が出るかを評価する.
\end{description}

\section{2016年度の予定}
2016年度の予定について図1に示し,以下で説明する.また,藤田の研究課題一覧を別紙「研究課題一覧(藤田)(2016年3月7日)」
に示す.別紙の表中の課題と対応する予定には,対応する課題番号を記載している.
なお,図1中の予定の通番は,以下で示す予定の通番と対応している.
\insertfigure[1]{fig:frame}{fig1}{2016年度研究計画}{ipi route}
\begin{enumerate}
    \item 割込処理が発生する経路の調査(課題番号1-A)\\
        割込ピン,MSI,およびIPIそれぞれについての割込処理発生までの経路について調査する.
        これは,3月中に完了させる.
    \item Linuxのテストの調査(課題番号1-C)\\
        Linuxのテストについて調査する.主にドライバのテストについて調査し,一般には困難なテストを明確にする.
        また,ドライバのコミット等を調査し,バグの報告を調査する.
        これは,4月中に完了させる.
    \item NICドライバを対象とした開発支援環境の実装(1-D)
        \begin{enumerate}
            \item NICドライバの解析(1-D-a)\\
                NICの送信処理を再現するため,NICドライバを解析し,送信処理を解析する.
                これは,5月中に完了する.
            \item NICハードウェアを用いないNICハードウェアの認識(1-D-b)\\
                本環境はNICを用いずに構築するという方針がある.現在,ネットワークインタフェースの起動時に
                NICを用いて起動し,NICドライバの初期化処理時にNICハードウェアを停止させ,使用しないようにしている.
                今後は,起動時にもNICを用いずネットワークインタフェースを起動し,ドライバの開発を可能にする.
                これに関して,デバイス認識について調査し,デバイスを用いずにデバイスを認識可能にする.
                これは,6月中に完了させる.
            \item パケット送信完了割込の生成(1-D-i)\\
                パケット送信処理を再現する機構を実装する.これにより,開発対象ドライバを用いて,OS間でネットワーク通信の再現が可能になる.
                これは,7月中に完了させる.
        \end{enumerate}
    \item 評価(1-E)
        \begin{enumerate}
            \item コード改変量についての評価(1-E-a)\\
                開発支援環境の構築におけるコード改変量について評価する.MintにおけるLinuxの改変量は考慮しない.
                これは,8月中に完了する.
            \item End-to-End通信における,プロトコルスタックの影響(1-E-e)\\
                現在は,NICドライバにおける通信処理性能を評価している.ここでは,アプリケーションレイヤまでを対象とし,
                NICドライバでの結果とどの程度の差異が発生するかを評価する.
                これは,9月中に完了する.
            \item バグの再現(1-E-f)\\
                報告されているドライバのバグを本環境で再現することにより,本環境の有用性を評価する.
                これは,10月中に完了させる.
            \item 仮想計算機を用いた割込処理の開発支援環境の評価(1-E-g)\\
                仮想計算機を用いて,本環境と同様の処理を行い,仮想化による処理オーバヘッドが
                どの程度のものとなるかを評価する.
                これは,12月中に完了する.
        \end{enumerate}
    \item 第139回OS研原稿執筆\\
        10月中旬締め切り
    \item 修士論文執筆\\
        2月上旬締め切り
\end{enumerate}

\section{おわりに}
本資料では,藤田の2016年度研究計画を示した.

\end{document}

%\begin{table}[htbp]
%    \caption{実験環境}
%    \label{kankyou}
%    \begin{center}
%        \begin{tabular}{l|l}   \hline \hline 
%            項目名      & 環境    \\ \hline
%            OS          & Fedora14 x86\_64(Mint 3.0.8)  \\ 
%            CPU         & Intel(R) Core(TM) Core i7-870 @ 2.93GHz \\ 
%            NICドライバ & RTL8169    \\ 
%            ソースコード& r8169.c \\ \hline
%
%        \end{tabular}
%    \end{center}
%\end{table}
%
%\insertfigure[0.8]{fig:frame}{fig1}{パケットの構成}{ipi route}


