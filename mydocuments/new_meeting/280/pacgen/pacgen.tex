% Created 2013-12-20 金 04:52
\documentclass[12pt]{jsarticle}
\usepackage[dvipdfmx]{graphicx}
\usepackage{comment}
%\usepackage{setspace}
%%\date{\today}
%\title{}
\textheight = 25truecm
\textwidth = 18truecm
\topmargin = -1.5truecm
\oddsidemargin = -1truecm
\evensidemargin = -1truecm
\marginparwidth = -1truecm
\usepackage{nutils}
\def\theenumii{\Alph{enumii}}
\def\theenumiii{\alph{enumiii}}
\def\labelenumi{(\theenumi)}
\def\labelenumiii{(\theenumiii)}
%\setstretch{0.9}
\begin{document}


%\maketitle
%\tableofcontents

\begin{center}
%%%%%%%%%%%%%%%%%%%%%%%%%%%%%%%%%%%%%%%
%%%タイトル                         %%%
%%%%%%%%%%%%%%%%%%%%%%%%%%%%%%%%%%%%%%%
    {\LARGE パケットジェネレータの作成(途中経過)}
\end{center}

\begin{flushright}
  2015/7/16\\
  藤田将輝
\end{flushright}
%%%%%%%%%%%%%%%%%%1章%%%%%%%%%%%%%%%%%%%
\section{はじめに}
libnetにより作成したパケットがNICドライバで正常に処理されていないことを確認した.
本資料では,その原因の調査の途中経過について述べる.
正常に処理されるパケットとlibnetによって作成されたパケットを比較し,
原因の調査を行った.
この結果,IPv4ヘッダに差異が多く見られた.
IPv4ヘッダのみを正常に処理されるパケットと同じ値にしたところ,
正常に処理されていることを確認した.

\section{パケットジェネレータ}
本デバッグ支援環境においてパケットジェネレータはデバッグ支援OS上で動作する
アプリケーションとして実装する.
パケットジェネレータはデバッグ対象OSのNICドライバが処理するパケットを作成し,
システムコールとして実装しているデバッグ支援機構を呼び出す機能を持つ.
本資料において,パケットが正常に処理されるとは,
パケットジェネレータを起動してから,デバッグ対象OS上のUDPを
受信し,メッセージを表示するアプリケーションがメッセージを表示するまで
の処理を行うこととする.
パケットジェネレータの処理流れを図\ref{fig:nagare}に示し,以下で説明する.
%\begin{enumerate}
%    \item 送信先のIPアドレス,メッセージ,パケットを送信する回数,および連続でパケットを送信する際のインターバルを引数に
%        パケットジェネレータを動作させる.
%    \item 指定されたIPアドレスとメッセージからパケットを作成する.
%        パケットの作成には,libnetというライブラリを用いる.
%    \item 作成したパケット,パケットのサイズ,パケットを送信する回数,および連続でパケットを送信する際のインターバル
%        を引数に,システムコールとして実装したデバッグ支援機構を呼び出す.
%\end{enumerate}
\begin{enumerate}
    \item デバッグ支援OSでパケットジェネレータを動作させる.
    \item パケットジェネレータにより,パケットが作成され,デバッグ支援機構が呼び出される.
    \item デバッグ支援機構は作成されたパケットを共有メモリに配置する.
    \item デバッグ支援機構がデバッグ対象OSに割り込みを発生させる.
    \item 割り込みハンドラが動作し,NICドライバはパケットを共有メモリからソケットバッファに格納する.
    \item NIC ドライバはソケットバッファを上位層に送信する.
    \item デバッグ対象OS上で動作するUDPの受信プログラムがUDPパケットを受け取る.
    \item デバッグ対象OS上で動作するUDPの受信プログラムがメッセージを画面に出力する.
\end{enumerate}

\section{作成したパケットの処理}
libnetにより作成したパケットをデバッグ対象OSのNICドライバに処理させた.
具体的には,以下の流れで,処理を行った.
\begin{enumerate}
    \item パケットを共有メモリに作成したNICドライバの受信バッファに配置する.
    \item 受信ディスクリプタを更新し,デバッグ対象OSの占有するコアへIPIを送信する.
    \item デバッグ対象OSの占有するコアがIPIを受信すると,NICドライバの割り込みハンドラが動作する.
    \item 割り込みハンドラにより,共有メモリからパケットを取得し,パケット受信割り込み処理を行う.
\end{enumerate}
以上の流れで作成したパケットを処理させた結果,正常に処理されていなかった.
具体的には,デバッグ対象OS上で動作するUDPのメッセージを取得し,表示するアプリケーションに
何も表示されていなかった.
これにより,パケットは正常に処理されていないと判断した.

\insertfigure[0.6]{fig:nagare}{fig1}{正常な処理流れ}{ipi route}
\section{原因の調査}
libnetにより作成したパケットが正常に処理されない原因について
調査している.本章では,この途中経過について説明する.
なお,正常に処理されないパケットと,正常に処理されるパケットの
値を比較することで調査を行った.
作成したパケットには,以下の3つのヘッダが付与されている.
\begin{enumerate}
    \item Etherヘッダ
    \item IPv4ヘッダ
    \item UDPヘッダ
\end{enumerate}
これらのヘッダによってパケットの処理が決定するため,これらのヘッダに
不具合の原因があると考えた.
また,EtherヘッダとUDPヘッダについてはヘッダを構成する要素が少なく,
作成したパケットと正常に処理されるパケットで大きな差異がなかったため,
IPv4ヘッダの構成に不備があると考え,調査を行った.
正常に動作するパケットのヘッダと,正常に動作しないパケットのヘッダで
異なっていたのは以下の要素である.
\begin{enumerate}
    \item データグラム長\\
        IPパケット全体のサイズを示す要素である.
    \item IDフィールド\\
        フラグメントが発生した際に使用される識別子をもつ要素である.
    \item フラグフィールド\\
        フラグメントを行うか否かを示す要素である.
    \item チェックサム\\
        IPv4ヘッダのチェックサムの要素である.
    \item 送信元IPアドレス\\
        送信元のIPアドレスを表す要素である.
\end{enumerate}
これらの値を全て正常に動作するパケットと同様の値に
変更すると,パケットは正常に処理されることを確認した.
どの要素が原因で正常に動作しないのかは特定できていない.
%データグラム長についてはペイロードが異なっている場合,
%それぞれでサイズは変化するため,原因ではないと考えている.
%IDフィールドについては,フラグメントが発生した際の識別子として
%使用されるものであり,作成したパケットではフラグメントが発生しないため,
%原因ではないと考えている.
%チェックサムはIPv4ヘッダの値によって変化するため,原因ではないと考えている.
今後,どの要素が不具合の原因となっているか調査し,
その理由について考察する.

\section{おわりに}
本資料では,libnetによって作成されたパケットが正常に処理されない
原因についての調査について述べた.
今後はIPv4ヘッダのどの要素が原因かを特定し,
その理由について調査し,考察する.
その後,パケットジェネレータを完成させる.
\end{document}

