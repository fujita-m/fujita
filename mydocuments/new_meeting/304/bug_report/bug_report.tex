% Created 2013-12-20 金 04:52
\documentclass[12pt]{jsarticle}
\usepackage[dvipdfmx]{graphicx}
\usepackage{comment}
%\usepackage{setspace}
%%\date{\today}
%\title{}
\textheight = 25truecm
\textwidth = 18truecm
\topmargin = -1.5truecm
\oddsidemargin = -1truecm
\evensidemargin = -1truecm
\marginparwidth = -1truecm
\usepackage{nutils}
\def\theenumii{\Alph{enumii}}
\def\theenumiii{\alph{enumiii}}
\def\labelenumi{(\theenumi)}
\def\labelenumiii{(\theenumiii)}
%\setstretch{0.9}
\begin{document}


%\maketitle
%\tableofcontents

\begin{center}
%%%%%%%%%%%%%%%%%%%%%%%%%%%%%%%%%%%%%%%
%%%タイトル                         %%%
%%%%%%%%%%%%%%%%%%%%%%%%%%%%%%%%%%%%%%%
    {\LARGE RTL8169の過去のバグについて}
\end{center}

\begin{flushright}
    2016/6/30\\
    藤田将輝
\end{flushright}
%%%%%%%%%%%%%%%%%%1章%%%%%%%%%%%%%%%%%%%
\section{はじめに}
開発支援環境を用いて,過去あったNICドライバのバグを再現することを考えている.
LinuxカーネルはGitを用いて開発されており,開発過程が全て記録として残っている.
本開発環境での開発対象であるNICドライバ(RTL8169)の開発過程も記録されており,
どのようなバグが修正されてきたかを確認可能である.
そこで,この記録から修正されたバグを調査した.
本資料は,調査したバグについてまとめたものである.

\section{調査目的}
本研究における開発支援環境はNICドライバを対象としており,NICを用いずにNICの動作を再現するものである.
これを用いて,報告されたバグを再現できれば,本環境がデバッグにおいて有用である可能性を示せる.
そこで,過去に修正されたバグを調査し,現在の開発支援環境で再現できるかを考察する.
また,現在の開発支援環境にどのような機能を追加すればバグを再現するかを考察する.
これらの目的からバグについて調査した.

\section{調査方法}
本開発支援環境の開発対象としているRTL8169はLinuxカーネルの一部としてGitを用いて
開発されている.このため,開発工程が全て記録として残っている.
この記録はある機能を追加した,あるバグを修正した,といった作業の粒度で記録されており,
そのメッセージが残されている.
そこで,本開発環境が対象としているバージョンの最新の更新記録から過去3年間の
バグを調査した.
バグを修正したメッセージには「fix: 」という文字列が含まれているため,
この文字列を含むメッセージの記録を調査した.
この際,ハードウェアに依存した修正や詳細が記述されていない修正は調査対象外とした.

\section{発見したバグ}
\subsection{一覧}
Linux3.0.8の最新更新記録から過去3年間のバグを調査した.
なお,詳細の記載されていないバグやハードウェアに関するバグは省いている.
その一覧を以下に示す.
\begin{enumerate}
    \item チェックサムを確認するタイミングに関するバグ
    \item 大きなサイズのパケットを受信した場合,マシンをクラッシュさせるバグ
    \item 特定のサイズのパケットを受信できないバグ
\end{enumerate}

\subsection{チェックサムを確認するタイミングに関するバグ}
\subsubsection{バグが発生する機能の概要}
NICドライバはパケットを受信した後,パケットをソケットバッファに格納する.
その後,NICドライバは受信ディスクリプタを確認する.TCPかUDPパケットを受信していることを確認すると
そのパケットが格納されているソケットバッファの構造体のメンバの1つである{\tt skb->ip\_summed}
というIP層でのチェックサムが必要であるか否かを示すフラグを更新する.これにより,IP層でチェックサムが必要でない状態にする.

\subsubsection{バグ内容}
本バグでは,ソケットバッファにパケットを格納する前に{\tt skb->ip\_summed}を更新してしまう.
この後,ソケットバッファにパケットを格納する.この際,新しくソケットバッファを作成し,新しいソケットバッファに
パケットを格納する.これにより,{\tt skb->ip\_summed}の更新が無かったことになる.
したがって,チェックサムの不要なパケットであってもIP層でチェックサムを確認することになる.

\subsection{大きなサイズのパケットを受信した場合,マシンをクラッシュさせるバグ}
\subsubsection{バグが発生する機能の概要}
NICは受信パケットの最大サイズとしてRxMaxSizeというレジスタを持っている.
これより大きいパケットはフィルタし,破棄するようになっている.

\subsubsection{バグ内容}
本バグでは,RxMaxSizeを固定的に16384としていた.
NICドライバがアロケートするバッファサイズは{\tt tp->rx\_buf\_sz}に格納されているが,
このサイズはMTUによって決定されるため,16384より小さくなる可能性がある.
このため,アロケートしたバッファサイズよりも大きいパケットを受信できる状態となる.
アロケートしたバッファサイズよりも大きいパケットを受信すると他のカーネル領域を侵害し,
クラッシュする可能性がある.

\subsection{特定のサイズのパケットを受信できないバグ}
\subsubsection{バグが発生する機能の概要}
NICドライバはネットワークインタフェースの起動時,MTUを指定できる.
このMTU値は{\tt dev->mtu}に格納されている.
NICドライバはこのMTU値を用いて{\tt tp->rx\_buf\_sz}という受信バッファのサイズを決定する.
このメンバはハードウェアのフィルタリングに関係しており,{\tt tp->rx\_buf\_sz}より大きい
サイズのパケットはハードウェアにより破棄される.
ユーザはMTUをIPパケットの最大サイズだと意識して値を指定する.

\subsubsection{バグ内容}
本バグでは,{\tt tp->rx\_buf\_sz}を決定する際,{\tt dev->mtu}の値とデフォルトの受信バッファサイズである
1536を比較し,{\tt dev->mtu}の方が大きければ,{\tt dev->mtu}の値にether net ヘッダのサイズを付与したサイズを
{\tt tp->rx\_buf\_sz}とする.{\tt dev->mtu}の方が小さければデフォルト値である1536とする.
ここで,指定したMTUはIPパケットサイズだと意識しているため,{\tt tp->rx\_buf\_sz}を決定する際,
デフォルト値と比較するのはフレームサイズである必要がある.しかし,本バグで比較しているのは
IPパケットサイズとして指定したMTUとデフォルト値である.これでは,指定したMTUに従ってパケットを受信しようとすると
ハードウェアによって破棄される可能性がある.例えば,ユーザがMTUとして1536を指定した場合を考える.
{\tt dev->mtu}には1536が格納され,{\tt tp->rx\_buf\_sz}を決定する際,{\tt dev->mtu}とデフォルト値である1536を比較する.
同じ値であるため,{\tt tp->rx\_buf\_sz}は1536となる.このため,1537B以上のパケットはハードウェアにより破棄される.
しかし,ユーザは1536BまでのIPパケットを受信できると意図している.1536BのIPパケットはフレームにすると\mbox{1536 + 22(ether net ヘッダ等) = 1558}
となり,受信できない.

\section{考察}
\subsection{チェックサムを確認するタイミングに関するバグ}
このバグは,バグが発生するバージョンに戻すことで,容易に再現が可能であると考えられる.
なぜなら,全てのパケットにバグの効果が適用されるためである.
本研究における開発支援環境を用いなくても,ソケットバッファが作成された段階で
ソケットバッファ構造体の要素の値を確認するだけでよい.
このため,このバグは,本開発支援環境のデバッグ支援対象としては使用できないと考えられる.

\subsection{大きなサイズのパケットを受信した場合,マシンをクラッシュさせるバグ}
このバグを再現しようとすると,2台の計算機を用いる必要がある.
また,アロケートしたバッファサイズ以上のパケットを受信する必要がある.
本開発支援環境を用いると1台の計算機で2つのOSを動作できる.
また,任意にパケットのサイズを指定できる.
これにより,このバグを再現できると考えられる.
このため,このバグは本開発支援環境で再現が容易になるバグだと考えられる.
ただし,このバグが発生するドライバのバージョンは本研究で実装したバージョンと
バッファのアロケート方法が違っているため,これを考慮する必要がある.

\subsection{特定のサイズのパケットを受信できないバグ}
このバグを再現しようとすると,2台の計算機を用いる必要がある.
また,MTUを指定してこれに近いサイズのパケットを受信する必要がある.
本開発支援環境を用いると,1台の計算機上で2つのOSを動作できる.
また,パケットのサイズの指定も可能である.
このため,このバグは本開発支援環境で再現が容易になるバグだと考えられる.
ただし,このバグが発生するドライバのバージョンは本研究で実装したバージョンと
バッファのアロケート方法が違っているため,これを考慮する必要がある.

\section{おわりに}
本資料では,過去に修正されたNICドライバのバグについてまとめた.
今後は,バグを再現するための具体的な課題や対処について検討する.

\end{document}

%\begin{table}[htbp]
%    \caption{実験環境}
%    \label{kankyou}
%    \begin{center}
%        \begin{tabular}{l|l}   \hline \hline 
%            項目名      & 環境    \\ \hline
%            OS          & Fedora14 x86\_64(Mint 3.0.8)  \\ 
%            CPU         & Intel(R) Core(TM) Core i7-870 @ 2.93GHz \\ 
%            NICドライバ & RTL8169    \\ 
%            ソースコード& r8169.c \\ \hline
%
%        \end{tabular}
%    \end{center}
%\end{table}
%
%\insertfigure[0.8]{fig:frame}{fig1}{パケットの構成}{ipi route}


