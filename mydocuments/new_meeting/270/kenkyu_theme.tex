% Created 2013-12-20 金 04:52
\documentclass[12pt]{jsarticle}
\usepackage[dvipdfmx]{graphicx}
\usepackage{comment}
%\usepackage{setspace}
%%\date{\today}
%\title{}
\textheight = 25truecm
\textwidth = 18truecm
\topmargin = -1.5truecm
\oddsidemargin = -1truecm
\evensidemargin = -1truecm
\marginparwidth = -1truecm
\def\theenumii{\Alph{enumii}}
\def\theenumiii{\alph{enumiii}}
\def\labelenumi{(\theenumi)}
\def\labelenumiii{(\theenumiii)}
%\setstretch{0.9}
\begin{document}

%\maketitle
%\tableofcontents

\begin{center}
%%%%%%%%%%%%%%%%%%%%%%%%%%%%%%%%%%%%%%%
%%%タイトル                         %%%
%%%%%%%%%%%%%%%%%%%%%%%%%%%%%%%%%%%%%%%
{\LARGE 2015年度Newグループ研究テーマ(案)}
\end{center}

\begin{flushright}
  2015/3/3\\
  藤田将輝
\end{flushright}
%%%%%%%%%%%%%%%%%%1章%%%%%%%%%%%%%%%%%%%
\section{はじめに}
本資料では,2015年度Newグループテーマ(案)について記述する.
\section{Newグループ研究テーマ}
現在のNewグループ研究テーマと課題について別紙の表“ Mint課題一覧”
に示し,以下で説明する.
Newグループの研究テーマは以下の7つに大別される.
\begin{description}
    \item[(テーマ1)] Mintの基本機能\\
        マルチコア計算機上で複数のカーネルが独立に走行するMintを実現する.
    \item[(テーマ2)] 用途別のOSの混載\\
        Linux系のOSにMint方式による変更を加えることで,複数のカーネルが
        1台の計算機上で独立に動作することを示す.
    \item[(テーマ3)] 自由な起動と終了\\
        マルチコア計算機上で複数のLinux系OSが目的に応じて起動し,
        終了することができる.
    \item[(テーマ4)] 資源分配\\
        Mint方式で動作するLinux系OSに効率良く計算機資源を分配する.
    \item[(テーマ5)] OS間の連携\\
        Mint方式を用いて,同一計算機で走行する複数OSノード間で連携し,
        処理を行う.
    \item[(テーマ6)] VMとの比較\\
        1台の計算機上で複数のOSが同行する方式として仮想計算機方式が
        広く利用されている.
        そこで,Mint方式とVMを比較することで,Mintの有用性について
        明確にする.
    \item[(テーマ7)] Mintの利用\\
        Mintを有効に利用した環境を考察し,作成する.
\end{description}

\section{来年度在籍する学生の研究テーマ}
2015年度のNewグループに所属するMasterが担当している研究テーマを以下に示す.
なお,テーマの担当者は右に記載する.
\begin{description}
    \item[(テーマ1)] Mintの基本機能
        \begin{enumerate}
            \item 多コアプロセッサへのMintの適用(中村)\\
                単一のOSで多くのコアを効率良く利用することは困難である.
                そこで,複数のOSでコアを分割統治する方法の1つにVM(Virtual Machine)方式が
                あるが,これには仮想化によるオーバヘッドが発生する.ここで,
                Mintを用いて複数のコアを分割統治する.しかし,
                Mintは各Linuxが最少構成としてメモリを一部,コアを1つ以上,
                ファイルシステムを持つためのI/Oデバイスを1つ以上占有する.
                また,1台の計算機に搭載可能なI/Oデバイスの数には物理的な制約がある.
                この問題への対処として,ファイルシステムを持つためのI/Oデバイスを占有しない
                OSノードの走行を実現する.
        \end{enumerate}
    \item[(テーマ4)] 資源分配
        \begin{enumerate}
            \item 割り込みルーティングの変更によるデバイス移譲(増田)\\
                現在MintはLoadable Kernel Module(LKM)を利用したデバイス
                の移譲が可能である.しかし,LKMを利用したデバイスの
                移譲時間はミリ秒単位のオーダとなっている.このため,
                タイムスライス間隔でのデバイス移譲は不可能である.
                そこで,MSIを利用することで,マイクロ秒単位でのデバイス
                移譲を可能にする.
        \end{enumerate}
    \item[(テーマ7)]Mintの利用
        \begin{enumerate}
            \item デバッグ支援環境の作成(藤田)\\
                Mintを用いて,OSのデバッグ支援環境を実現する.
                既存のOSのデバッグ支援環境として仮想計算機を用いたものがある.
                しかし,仮想計算機を用いたものはハイパーバイザへの処理遷移に伴う
                処理負荷が発生する.このため,実割り込みの再現が困難になる.
                MintはOS間の独立性が保たれているため,一方の処理負荷を受けずに
                割り込み処理の再現が可能になる.
        \end{enumerate}
\end{description}
\section{新B4の研究テーマ}
2014年度Newグループに所属するBachelorに割り振る研究テーマ案を以下に示す.なお,
各テーマにおいて,2013年度に担当者が確定していたものは,担当者名を右に記載する.
\begin{description}
    \item[(テーマ2)]用途別OSの混載
        \begin{enumerate}
            \item 組み込み環境への対応\\
                1台の計算機上で複数のOSを同時走行させる手法として,仮想計算機方式がある.
                仮想計算機方式を用いる場合,ハイパーバイザの処理によるオーバヘッドが発生する.
                このため,仮想計算機方式はリアルタイム性が要求される組み込み環境には
                適していない.一方Mintは,実計算機と同じ環境で複数OSを
                同時走行させることができる.そこで,組み込み環境でMintを実現する.
        \end{enumerate}
    \item[(テーマ3)]自由な機動と終了
        \begin{enumerate}
            \item OSノード環境の高速な複製手法の実現\\
                Mintにおいて,すでに起動しているOSノードの環境を高速に複製する手法を実現する.
                OSノード環境の高速な複製手法を実現するためには,
                以下の課題を完了させる必要がある.
                \begin{enumerate}
                    \item すでに走行しているカーネルの指定領域へのコピー
                    \item 初期化ルーチンの変更
                    \item コアの走行環境の再現
                \end{enumerate}
        \end{enumerate}
    \item[(テーマ4)]資源分配
        \begin{enumerate}
            \item Mintにおける資源の可視化\\
                Mintにおいて,CPUコア,メモリ,および入出力デバイスを各OSノードが
                どのように占有しているのかをユーザに提示する機能を実現する.この際,
                CPUコアとメモリの資源管理インタフェースと連携する.
            \item Mintにおけるデバイス認識の動的変更\\
                現在,Mint方式ではあるOSノードが占有するデバイスを他OSノードから隠蔽すること
                で,デバイスの分配を実現している.デバイスの隠蔽をOSノード走行中に動的に
                変更することで,デバイスの動的な分配を実現する.
            \item USBデバイスへの対応\\
                現在,Mintにおいて各OSノードがUSBデバイスを占有して使用することを
                想定していない.このため,MintでUSBデバイスを使用可能にする.
        \end{enumerate}
    \item[(テーマ5)]OS間の連携
        \begin{enumerate}
            \item Mintにおけるヘルスチェック機能の設計\\
                Mintにおけるヘルス機能チェックを設計する.Mintにおける
                ヘルスチェック機能を完成させるためには,以下の5つの課題を完了させる必要がある.
                \begin{enumerate}
                    \item コアを一時停止する機能の実装
                    \item 一時停止したコアを再開する機能の実装
                    \item ハッシュ値を取得する機能の実装
                    \item サービス用OSを起動中に検査する機能の実装
                    \item LKM検査機能の実装
                \end{enumerate}
                以上の5つの課題の内,(A)コアを一時停止する機能の実装と(B)一時停止したコアを再開
                する機能の実装は完了している.また,必須の機能ではないが,TPM(Trusted Platform
                Module)を使用して後続OSを起動し,ヘルスチェック機能の正当性を向上させる必要が
                ある.
            \item OSノード間の高速通信機能の実現\\
                Mintにおいて,OSノード間の高速な通信機能を実現する.具体的には,
                Mint独自のデバイスドライバとして,仮想NICドライバを作成することで,
                同一計算機上のOSノード間で高速な通信機能を実現する.
        \end{enumerate}
    \item[(テーマ7)]Mintの利用
        \begin{enumerate}
            \item Xen modefied Mint\\
                近年,ハイパーバイザの複数走行が望まれている.
                このため,ホストOS型の仮想化技術の多段化により,
                ハイパーバイザの複数走行が実現されている.
                しかし,Xenのようなハイパーバイザ型の
                複数走行はハイパーバイザの変更が必要であるため,
                実現されていない.
                ハイパーバイザ型の仮想化技術はホストOS型に比べて,
                性能が高い特徴を持つ.
                そこで,最下段の走行方式にMintを用いて,Xenの
                ハイパーバイザを動作させる.
                これにより,より性能が高いOSを多く動作できる.
        \end{enumerate}
\end{description}
\section{完了した研究テーマ}
前年度までに完了した研究テーマをいかに示す.なお,担当者については研究テーマの右に記載する.
\begin{description}
    \item[(テーマ1)]Mintの基本機能
        \begin{enumerate}
            \item メモリ分配機能(粟田)\\
                各OSノードが占有するメモリ領域を操作し,各OSに特定のメモリ領域を占有させる.
            \item CPUコア分割機能(粟田)\\
                各OSノードが占有するCPUコア以外のコアを使用しないように制限する.
            \item 入出力デバイス占有機能(粟田・千崎)\\
                各OSノードにおいて,任意の入出力デバイスを選択し,占有させる.また,
                占有していない入出力デバイスを使用しないようにする.
            \item OSノードの起動(粟田)\\
                最初に起動したOSノードからシステムコールを発行することでほかのOSノードを
                起動する.
        \end{enumerate}
    \item[(テーマ2)]用途別OSの混載
        \begin{enumerate}
            \item Linux系OSの混載(中原)\\
                32bitLinuxと64bitLinuxを混載する.
        \end{enumerate}
    \item[(テーマ3)]自由な起動と終了
        \begin{enumerate}
            \item Kexecを用いたOSノードの起動(中原)\\
                Kexecを用いて最初に起動したOSノードから他のOSノードを起動する.
            \item 単一のカーネルイメージを用いた起動(中原)\\
                同一のLinuxカーネルを起動する際に複数のカーネルイメージを用意することなく
                カーネルを起動する.
            \item ディスクレスブート機能(天野)\\
                初期RAMディスクを変更することで,ディスクを占有しないOSノードを起動する.
            \item 起動の並列化(中原)\\
                複数OSノードを並列に起動する.
        \end{enumerate}
    \item[(テーマ4)] 資源分配
        \begin{enumerate}
            \item Mintにおける実メモリ分配(宮崎)\\
                Mintでは,カーネルメイク時に実メモリ分配を決め打ちで指定しておき,
                OSノード起動時にOSノード毎に処理を分岐している.よって,
                OSノードメモリ構成を柔軟に設定可能にする.
            \item Mintにおける実メモリ移譲(宮崎)\\
                Mintにおいて,Linuxのメモリホットプラグ機能を利用することで
                MintのOSノード間で実メモリ移譲を可能にする.
            \item MintにおけるCPUコア移譲(池田騰)\\
                Mintにおいて,LinuxのCPUホットプラグ機能を利用することで,
                MintのOSノード間でCPUコアを移譲可能にする.
        \end{enumerate}
    \item[(テーマ6)]VMとの比較
        \begin{enumerate}
            \item VMとの性能比較(池田剛,仲尾)\\
                VM方式との性能比較を行う.
                具体的には,XenとKVMで比較する.
            \item MintにおけるKVMの評価(仲尾)\\
                同一計算機上に複数のOSを走行させる方式として,VM方式が広く利用されている.
                このため,Mint方式とVM方式の処理性能を比較することで,
                Mintの性能について明らかにする.また,Mint方式のOSノードは通常のLinux
                として動作することから,Mint方式で動作するOSノードにVM方式を導入した
                場合の処理性能を明らかにする.
        \end{enumerate}
    \item[(テーマ7)]Mintの利用
        \begin{enumerate}
            \item KVM on Mint(仲尾)\\
                近年,複数のOSが走行する技術の多段化が望まれている.
                しかし,既存技術による多段化はOSの性能低下が大きい.
                このため,最下段の走行方式にMintを用いることで,
                性能低下を抑制できることを示す.
                具体的には,KVMをMintを用いて複数走行させた際の性能を
                評価した.
        \end{enumerate}
\end{description}
\section{保留するテーマ}
前年度までの研究テーマにおいて,保留するテーマについて以下に示す.
なお,担当者については研究テーマの右に記載する.
\begin{description}
    \item[(テーマ2)]用途別OSの混載
        \begin{enumerate}
            \item Linux系OSの混載(北川)\\
                MintにおいてLinuxをベースとするOSを混載させる.この際,
                LinuxベースのOSを混載するために必要な改修箇所と改修理由を明らかにする.
                また,OS混載の更なる研究としてARMアーキテクチャ上にMintを適用し,
                AndroidとLinuxの混載を目指す.
                しかし,ARM用のKexecが正常に動作できることを
                確認していないため,保留とする.
        \end{enumerate}
\end{description}
\section{おわりに}
本資料では,2015年度Newグループテーマ(案)について記述した.
\end{document}


